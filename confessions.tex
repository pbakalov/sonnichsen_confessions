\documentclass[a5paper,12pt]{book}
\usepackage{hyperref}
\usepackage{geometry}

\hyphenation{nest-ling}
\geometry{margin=1in}

\begin{document}

\frontmatter
\title{Confessions of a Macedonian Bandit}
\author{Albert Sonnichsen}
\date{}

\maketitle

\newpage

\newpage
\begin{center}
    \vspace*{\fill}
    First published by Duffield and Company\\
    New York, 1909\\
    \vspace*{\fill}
\end{center}


\tableofcontents

\mainmatter

%CONFESSIONS OF A MACEDONIAN BANDIT

%By ALBERT SONNICHSEN
%Author of “Ten Months a Captive among Filipinos”, “Deep Sea Vagabonds”, etc.

%NEW YORK
%DUFFIELD & COMPANY
%1909

%Copyright, 1909, by Duffield & Company
%THE PREMIER PRESS

%CONTENTS.
%
%Chapter I. By Legal Passport. Page 9
%Chapter II. Joining the Chetas. Page 19
%Chapter III. With Luka's Outlaws. Page 25
%Chapter IV. The Swamp of Karafferia. Page 30
%Chapter V. The Burning of Nici. Page 53
%Chapter VI. Through the Mountains of Vodinsko. Page 65
%Chapter VII. Becoming Turkish Subjects. Page 83
%Chapter VIII. On the Road to Market. Page 93
%Chapter IX. On the Intrigues of a Prince. Page 102
%Chapter X. Monastir City. Page 111
%Chapter XI. Meeting Officials of the Underground Republic. Page 117
%Chapter XII. A Lonesome Walk. Page 124
%Chapter XIII. Back in Yeni Mahli. Page 132
%Chapter XIV. Massacre of a Cheta. Page 144
%Chapter XV. Among the Masses. Page 150
%Chapter XVI. A Voyvoda with a Hobby. Page 160
%Chapter XVII. Petrush “The Just”. Page 167
%Chapter XVIII. Krusty of Ressen. Page 190
%Chapter XIX. Yeni Mali Again. Page 194
%Chapter XX. A Picnic with a Climax. Page 214
%Chapter XXI. “Asker!”. Page 221
%Chapter XXII. Back to Vodensko. Page 232
%Chapter XXIII. A Royal Emissary. Page 243
%Chapter XXIV. Confession of the Melancholy Brigand. Page 253

%LIST OF ILLUSTRATIONS.
%
%Hristo Tchernopeef. Frontispiece
%Luka Ivanoff, Vodensko voyvoda. Page 24
%Apostol, chief of Enedjie Vardar, whom Abdul Hamid believed to be the leading spirit of the revolutionary movement. Page 34
%In the Swamps of Karafferia; Luka’s Camp. Page 44
%Tashko, voyvoda of Demir Hizzar, and his pillbox camera. Page 50
%We frequently bathed in Lake Prespa. Page 60
%Krusty Trikov, before his trip into Monastir. Page 100
%Map, showing route followed by the author. Page 120
%A jolly time on the heights above Lake Prespa. Page 140
%Petrush gathers testimony for one of his cases. Page 166
%The Ochridsko cheta; Petrush in middle, author to left of him. Page 180
%We disembarked in an inlet half way down the eastern shore. Page 188
%Krusty, of Ressen and Sandy. Page 190
%Disguised as Peasants; the day after the flight from Monastir. Page 200
%Miss Stone’s Captors. Page 214
%Hristo Tchernopeef holding a conference with a village local committee. Page 250
%
%CONFESSIONS OF A MACEDONIAN BANDIT

%CHAPTER I. 

\chapter{BY LEGAL PASSPORT}
\pagestyle{plain} % removes headers & keeps page numbers

My coming into Turkey was scrupulously legal. The Turkish diplomatic agent in Sofia did not hesitate to vise my passport, though for two years I had been knocking about the small Bulgarian capital associating with men of doubtful reputation in the eyes of consular officials.

“You are a member of some traveling troupe?” he observed, regarding my clean-shaven face. The suggestion was a good one. What consular official has not helped a stranded actor on his way to that rendezvous of bankrupt vaudeville artists—Cairo?

“Yes,” I said, “I shall probably go to Cairo from Salonica.”

So, on the evening of February 28th, 1906, I crossed the frontier and traveled through Adrianople down to Salonica on the Aegean Sea, meeting with no worse obstacle than the confiscation of a paper-bound novel.

For ten days I saw official Turkey in Europe as a dozen good books describe it. In Salonica journalism became my vocation, and I made the round of local institutions and notabilities, mostly Greek, it seemed, though the city is Jewish by a vast majority. Then, one morning, I set out for the interior on a limited Turkish passport, obtained through the American consul, also a Greek.

The railway to Monastir, leaving Salonica, enters a vast, swampy plain. On the left the crags of Mt. Olympus, towering naked above the timber line, are so near that you can distinguish the rock fissures. The mountains continue inland in a succession of bare peaks. This I observed casually, for my interest was centered on the swamps to our right.

We crossed wide tracts of brown bullrushes, catching here and there a glimpse of open water. Sometimes the metallic clacking of the wheels on the track broke into a hollow roar as the train rolled over trestlework. In the distance, beyond what seemed an open lake, a streak of stunted forest appeared, from which rose black, naked limbs of dead trees, silhouetted against the pale sky above the low horizon. Circling flocks of blackbirds, crows, or ravens, only accentuated the desolate lifelessness of the aspect. So the swamps of Karafferia appear from the coach windows of the Monastir railway.

On the left, again, at almost regular intervals, we passed small clusters of minarets rising from white domes among groups of glistening white buildings, nestling in hollows among the rising foothills; Karafferia, Negush, and finally, as the road swings into the first rise of hills, Voden. Here was the Macedonia of antiquity. Here, too, centered the bitterest of the present strife that makes of modern Macedonia the bloodiest field of battle of our own times.

My passport gave me a day’s stop-over privilege in any town along the route. I alighted in Voden with only a dress-suit case half full of old clothes and money in my pocket amounting to less than the price of a good dinner. There was nothing about me to confiscate, and I passed out of the station into the crowd of local inhabitants, gathered to inspect new arrivals.

There was a rush of boys to relieve me of my suit case, but I held fast to it until I caught the eye of a stout youth wearing a white fez and skirting the outer circle of his competitors. He instantly butted through the crowd and got hold of my luggage. The others fell back, and the white-fezzed boy followed me at a few paces. I started out along the mile of open road that led to the town.

“I am taking you to the Greek hotel,” he said in Bulgarian. “I was told to tell you this; after dinner, first thing, the kaimakam. Then the Greek school, and, if there is time, the Greek bishop. At four I shall be lounging in front of the hotel.”

He did the business as expertly as any experienced conspirator. We passed through the narrow, crooked main street and came to the inn, where my porter, in his zeal to act his part realistically, demanded so big a fee that I was much embarrassed, for I was ill equipped to play the part of the generous stranger. In sight of all the onlookers i gave the boy, and he took, not too graciously, every last para I had, though it left me penniless.

The hotel was little more than a roadside hahn. Voden needed nothing more pretentious. The lower floor was a cafe, where some Turkish officers, a few civilians, and a priest or two drank coffee or plum brandy and gamed at cards or dominoes. A stranger was rare enough to rouse keen curiosity. The landlord was supposed to be a Greek, but I observed that he addressed his servants in Bulgarian, and, as I learned later, he knew no more Greek than I. Similarly in Caesar's day there must have been many a Roman who did not know his Latin.
After dinner, I went out into the general mass of low, stone buildings which pass as streets in that country. I wore a caped overcoat, common enough in that part of the Orient, but the storekeepers, seated cross-legged on low platforms inside their open-front shops, passing slovenly soldiers and idlers and peasants coming in to market, all stared. Although Voden is officially a Greek town, the remarks I heard passed around me were either in Bulgarian or distinctly Turkish in sound. As the necessity of asking my way took hold of me, I became aware of my embarrassing predicament. Of Turkish I knew almost nothing. Of Bulgarian I must betray no knowledge. My futile efforts to locate the kaimakam’s headquarters brought about me a mixed crowd of Moslems and Christians. 

Some uttered solitary French words; all seemed desirous of entering into some sort of verbal intercourse with me. Mentally I cursed the local committee for not having supplied me with an interpreter. Once I was on the point of addressing them in purposely mutilated Bulgarian, hoping it would pass as Russian. I refrained as I caught the remark, “Let’s find one of the Russian gendarmerie officers; they can talk French and German.” 

“No, let us take him to Dr. Filtcheff,” suggested another, “he speaks German.” The doctor must have thought they meant to mob his office, for he came out rather hurriedly; a tall, fair man of refined appearance. He eyed me coldly for a moment, then deliberately turned his back and spoke to the crowd. 

“Who told you I knew German?” he demanded curtly in Bulgarian. “I am busy. Take him to the station-master; he’s an Austrian.’’ 

The crowd took the rebuke in better humor than I did, and hustled me off, fully determined now that they must see me through my troubles before they went about their own affairs again. One of them, a Turkish sergeant, speaking broken Bulgarian, declared that if the station-master wouldn’t do, he was going to take me out to a bey who lived in the suburbs, and who had been in Paris and knew all the languages spoken in Europe. By good chance we met the station-master in the next street, and when he and I began conversing quite comfortably in German, the crowd, which was quite a big one by this time, gave a loud murmur of satisfaction, as with a duty undertaken and well done, and dispersed. 

The Austrian was so overjoyed to meet an “auslander” that he proposed at once to take me to the only place in town where good bottled beer could be had. This proposal being accomplished in fact, we next visited the kaimakam, who was such a courteous, jolly little fellow that I felt regretful of the trouble I was going to cause him before another day should pass. 

From the kaimakam we went to the Greek school. The Austrian, who could not conceive of any foreigner but a journalist visiting Voden, took it upon himself to supply me with material for what he thought would be an unbiased, comprehensive article on the situation in Macedonia. Quite confidentially, unofficially, as between man and man, he let me in on the fact that the government was pretty much of a joke. It was the Greek Church that kept the administrative machinery running in the splendid fashion that must already have impressed me. Fine people, the Greeks—quite a jolly crowd. 

“If only the Bulgars would keep quiet,” he remarked, “all would be perfect.” 

“So they are the trouble makers,” I replied. "Still, I suppose they are in such a minority they can’t do much.” 

"Well,” he admitted, "it’s hard nowadays to know who is Greek and who is Bulgar.” 

"By language, I should suppose.” 

"Well, no. You see, so many of the Greeks are Bulgarophones. Some are secretly Bulgars, supporting the brigands out in the mountains. Let me warn you—don’t go outside of the town without a strong escort. Luka and his band of desperadoes come down into the plain sometimes.” I thanked him for his warning. 

The director of the school, who was a genuine Greek, imported, spoke German. I hadn’t come to discuss politics with him, but I was beginning to learn that the political hatreds of these people were so tense as to be almost incomprehensible to a foreigner. Here men rise to a day’s work with bitter hate uppermost in their hearts, pass the day scheming destruction to their fellowmen, and lie down at night with venomous thoughts to dream over. 

"Keep clear of the Bulgars,” were the director’s words of parting, "they are a treacherous people. They are a tribe of heretics.” 

"There is a cotton mill nearby,” said the Austrian as we left the school, "whose manager speaks English. Mr. Gregoraki has been in England. Let us visit him.” 

I pricked up my ears at the name. It was familiar. 

We climbed a narrow path cut out of the face of a steep bluff, on top of which was built what seemed a walled citadel. A heavily armed sentry in Albanian dress challenged us at the gates, then passed us by other equally well armed guards, into a yard before a big factory building. A middle aged, flabbily built man in European dress met us in the doorway of an office. The bluish bags of flesh under his eyes and about his jaws indicated, I thought, need of physical exercise. 

Gregoraki spoke a soft, fluent English, pleasantly persuasive. He showed me through his mill, explaining the machinery, as though absorbed in their workings, as though to regulate all those whirling spindles were his life work. It was I who casually broached politics. 

“This is a Greek enterprise,” he said, “therefore the Bulgars don't like it. They tried to blow us up once.” Not so much the factory, I thought, as Mr. Gregoraki himself, had been the object of that attack. I observed that out of the several hundred operatives not a score were adults; most of them were girls from eight to fourteen. Bulgars, too, as I could hear in passing them, but not Bulgars to be afraid of. 

We returned to a small, dingy office to drink coffee, while Gregoraki talked on, mostly about England. In one corner of the apartment stood a cot. Over it, on the wall, hung a modern repeating rifle. As I glanced further around at the unpretentious office furniture and the dusty files, it seemed hard to realize that here was more truly the center of administration for that turbulent province than in the office of the kaimakan. It was not easy to believe that this harmless looking, obese Greek with the appearance of a New York peanut vendor, was the chief and the organizer of as bloody a league of —brigands, I was about so say. But Gregoraki was no brigand. In the eyes of diplomatic Europe he was a man of law and order, as was his colleague, the bishop. 

Gregoraki was a man of executive ability. He guarded the temporal welfare of the Holy Church in that province. He reached out the shepherd’s crook when the sheep would stray from the flocks of the Patriarch. Through his “confidential agents” he studied the influence of peasant agitators, and, at the right moment, eliminated them from their fields of activity. For that purpose he supported organized bands of Christian soldiers whose duties it was, after the fashion of their ancestor of antiquity, to lop off the heads of the tallest stalks in the field of growing wheat. Gregoraki applied the flames that purge, he washed away the sins of the people with blood; their own blood, be it added. 

I was back at the inn before four, and there the Austrian left me, believing, I suppose, that I should be presently at work on the material I had just gathered. The boy was loitering about up the street.
After drinking a glass of tea I sauntered out again and followed the white fez of my guide through devious narrow streets, far behind him. Gradually we came to another end of town. Then, suddenly, he darted into an open doorway. I watched my chance; at a moment when the street seemed deserted, I slipped in. The door slammed behind me. I climbed a dark stairway and came abruptly into a lighted room. The floor along the walls was snugly fitted out with rugs and cushions. Three men, squatting on the rugs, rose and greeted me.

“Welcome, comrade.”

One was the doctor whose rudeness I had resented that day. He broke into a laugh as I recognized him.

“Eine unangenehme Begegnung in der Strasse kann doch sehr angenehm sein, wenn sie zu Hause ist”

He knew his German as well as a Heidelberg graduate can know it, but too much knowledge is sometimes compromising.

%CHAPTER II.

\chapter{JOINING THE CHETAS}

It was fully an hour before dark when I returned to the inn. The cafe was still crowded. A young police officer invited me to drink with him. He spoke enough German for us to get on with, but behind his casual questions I sensed the desire of the local authorities to find out what my plans for the morrow were. He proposed dominoes and we played two games, both of which I lost. I rose and casually asked when dinner would be ready. In half an hour, the innkeeper said. I looked at my watch, then observed:

“That leaves me time for a stroll to the railway station.”

I was uneasy lest the police officer should want to accompany me, but he satisfied himself with enjoining me to be back before dark, as then the hotel was closed and soldiers took possession of it for the night, that no one might enter or leave.

I left the hotel and walked leisurely down the street toward the open road across the vineyards to the station. The few people abroad were evidently hurrying homeward. Just beyond the last few houses was a bridge. Two men in rough peasant dress lounged against the hand railing. As I approached they started slowly along the road away from the town. I followed, making no effort to overtake them. Not another person was now in sight.

Half way out to the station the two men turned suddenly aside and dived into a vineyard. I quickened my pace and followed. Among the vines, which were of the American kind, reaching over my head, I caught up with them. Without talking we pushed on together through thick, soggy soil, for it had rained lately and was even now beginning to drizzle.

We had passed through about a mile of vineyards and orchards, leaping over hedges, crossing foot paths cautiously, when at last we came to open fields. By this time it was almost dark, but we halted a few minutes in a clump of bushes. One of the men passed me a Nagant revolver attached to a belt full of ammunition.

“Be careful now,” he whispered, “we cross the track here.”

We continued across the fields, slowly and cautiously. I could see the loom of the railway embankment before us; just at its base two of us squatted while the other crawled up to reconnoitre.

The railway lines in Turkey are as carefully patrolled by soldiers as the frontiers. The man above gave a low whistle and the two of us dashed up and across. About a quarter of a mile down the track a group of guards huddled around a small fire. But the night had now become so dark that I had difficulty in keeping close to the two figures before me.

Almost at once we struck rising ground and were getting up into the foot hills. For two hours we pushed on, pausing only to listen. The air became crisper, and splotches of snow behind a tree or a rock became increasingly frequent. Finally we halted again and sat just below the summit of a hill. One of my guides gave a call; an answering hoot resounded from above, then a man leading a horse appeared coming down toward us.

“Mount,” said one to me, “you aren't used to this yet.”

I was thankful for the mount, for my flimsy shoes and rain-soaked overcoat had been telling on my pace.

A few minutes' rest, and we continued, up a steep trail, one of the peasants leading the horse. In another hour we came to level ground. The barking of dogs announced the vicinity of a village, and presently we came in among a group of huts. The barking of dogs had grown furious about us, but I heard men’s voices calling in sharp, angry commands, and the noise gradually ceased. One by one about a dozen figures came out of the darkness and quietly greeted me with handshakes. Hard, calloused, swollen hands; I knew they were peasants.

Against a clearing sky I made out the black outlines of what appeared to be a walled enclosure about one of the houses. Five shadows emerged and as they approached I saw they were mounted men wrapped in huge, white, shepherds’ cloaks. From each right shoulder a black bar protruded; rifle barrels, so I had not to wait for their handshakes to know they were comitajis.

“Welcome, comrade,” a youthful voice greeted me, in pure, literary Bulgarian. “We’ve come down to meet you. The rest are further on.”

I bid my two guides goodbye, and we rode on. Our path was still dark, but ahead and above the white snowpeaks glowed in a pale, self-luminous, light. We traveled upward through snow and at times plunged into deep drifts where the wind had swept it into hollows.

The horizon to our right was lighting. A full moon rolled suddenly up into a clear sky, illuminating the wild, craggy, mountainous scenery, bringing out distant trees into vivid clearness, but still leaving the cloaked figures before and behind me in a vague blending of light and shadow. The leader’s boyish face seemed almost effeminate in the mellow radiance as he turned once to offer me a cigarette.

The climbing became slow and hard, in and out among rocks and boulders. I felt my horse’s shoulder muscles swelling and working against my thighs. Sometimes we mounted narrow trails winding around the corners of perpendicular cliffs, repassing by the same way half an hour later, only a few hundred feet higher up.

Thus we finally reached a piece of shelf land. We halted before the base of a cliff so steep that the snow clung only in the fissures. Dogs were barking seemingly in the air above. From up the cliff came a call which one of my companions answered. I heard the clattering of loose stones coming down the face of the bluff, then made out figures clambering down a narrow trail and caught the glint of weapons. A line of fully fifty men were descending. We approached again until I could distinguish a tall figure in the lead, a bearded man with a broad-brimmed, white felt hat. He called to me by name and I gripped hands with my old acquaintance from Sofia, Luka Ivanoff, the terrible Luka, against whom the kindly Greeks had warned me. The greater part of his rayon force was with him.

By skirting the base of the cliff and by a more gradual ascent we reached a large village. The cheta melted among the houses, leaving only Luka, four of his comrades and myself with the peasants. In a few minutes we were before a roaring fire in one of the houses. My clothes were near frozen stiff on me. Amid a great deal of talking, a steady fire of questions and answers, I crawled out of my rags and put on, first, heavy woolen underwear, a pair of grey trousers, heavy-knitted stockings, white leggings, cowhide moccasins, and finally, a grey, close fitting jacket; just such a uniform as Luka wore.
"Here the clothes make the man," laughed Luka’s young lieutenant. "Let a soldier see that colour, even through his field glass, and his Mauser begins popping." The peasants crowded into the room while a motherly old woman dosed me with a hot mixture, mostly cognac. The villagers were keenly excited, each insisting upon personal converse with the foreigner from distant America who had come to serve comitluk with the chetas. The old village priest maintained his right to delivering the official welcoming address, which was tolerated rather than listened to until we squatted down to a supper that would not have shamed the Imperial Hotel in Sofia; a whole roast lamb, chicken, fresh milk, eggs, preserved grapes, apples, walnuts, roasted chestnuts, and finally bottled beer from Salonica.

It was dawn when we rolled up to sleep. I was dozing off, when I felt somebody gently spreading a goat-hair cloak over me. It was Luka, and characteristic of his conviction that only a Bulgar could endure hardship. It took a week to convince him that I had slept in full dress before, and could slumber quite restfully without a nightcap.


%CHAPTER III. 

\chapter{WITH LUKA’S OUTLAWS.}

Friends, engaged in journalism, have since credited me with manipulating a clever sensation (to advertise my correspondence) in the manner of my joining Luka’s forces. But neither Luka nor I planned that event. Cables describing my positive death were published in American and European journals and never refuted. The Sofia papers all but reproduced photographs of the actual murderers, while the Greek papers had documentary proof that Luka had done the deed. Meanwhile we suspected nothing of this excitement, and were taken by surprise with the vigorous measures that were used in Voden.

On the evening of the next day villagers who had been sent to town, ostensibly to market, returned with reports of an unusual movement of troops. One brought a letter from the local committee stating the police had made a house to house search of suspected quarters of the town. Every one who had been seen in my company was in jail. Several government officials and gendarmerie officers had come up from Salonica, together with the American consul.

That evening we were on the move early, but an hour after our departure soldiers surrounded the village and searched every house, though half the inhabitants were Mohammedans. Had it been summer we could have avoided the military by camping out in the forests, but deep snow compelled roofed shelter. The cheta split into three bodies and moved to different parts of the rayon. We made long, forced night marches, shepherds following our trail with their flocks to obliterate our tracks. Not a day passed undisturbed by alarms of approaching troops. Once we actually encountered an advanced scouting patrol of a dozen cavalrymen, but as we at least equalled them in number they wheeled about and disappeared. They had seen us, though, and it meant we must get out of the locality before a whole battalion swept it through.

That night we descended into the valley about Voden, passing near enough to the town to see its houses, skirted the railroad track and spent next day in a village down in the plain. The following evening we were on the move again at sunset, all of us mounted. The rain poured driving torrents and the plain became partly submerged. At times we seemed to cross shoreless lakes, the horses struggling through water so deep that their riders had to fasten their cloaks and rifles and ammunition belts on top the saddles and swim beside, holding on to the necks of the beasts. Just before dawn we emerged on a low hillock and entered a village of a few dozen huts.

“Now we’re near the end,” Luka told me. “Tonight we shall be where Turkish asker never trouble.”

After breakfast and a short rest we continued afoot, leaving the horses to be returned to the village from which we had borrowed them. Before us the ground and some stunted trees blended into a vast, blue mist, a vague emptiness, as though we approached a precipice overlooking nothingness. The path was over soft, spongy marsh grass, ending abruptly at the edge of a swollen, brown stream along whose low bank four punt-like craft were moored.

“Is this the jumping off place?” I asked.

“Where the devil bids goodbye,” responded Teodor, Luka’s young lieutenant.

“Mark this region down on your map in red,” put in Luka, “for it’s ours by right of occupation.”

We embarked, four in a boat, and floated down the stream into the blue mist. There was a peculiar rising sensation, as though we shot up into space. Nothing about was visible, not even the turbid water on whose surface we drifted. At times black objects appeared suddenly and shot past overhead, disappearing so quickly that I failed at first to recognize them as limbs of trees. At such moments I realized that we moved swiftly.

After a while the mist thinned, then suddenly swept behind and away, leaving the region through which we passed visible on all sides. The boats drifted down a wide, flowing channel, through forests of half submerged, black, leafless trees, slimy creepers of a dirty green hanging from the bent and twisted limbs to the muddy water or the quivering mire below. Here and there, in more open spaces, appeared sparse patches of brown bulrushes among which sat monstrous, olive-green frogs with huge, yellow ringed eyes, as though wearing fantastic, gold-rimmed spectacles.

We branched off from the widening river and began zigzagging among the black tree trunks, through myriads of coughing, sputtering frogs. At times we stooped to pass under low, horizontal limbs. The speed was slackening; in each boat a man began poling. Stormy swarms of crows rose and gyrated furiously overhead, and once a screeching hawk darted out of a nest as we passed, wafting down a powerful stench of carrion. We floated into small lagoons from which open vistas of water appeared. Wild ducks and geese rose clattering before us. Several times I caught sight of motionless, meditative cranes, so perfectly reflected in the still pools that they duplicated themselves, standing on their inverted selves.

The trees were thinning when we glided into a narrow waterway winding in through tall cane grass, so tall, in fact, that the feathery tops met overhead in places, and we passed down arched avenues. Then, quite unexpectedly, to me, at least, we emerged into an open lake. On a low island, down the further end, were two rush huts. Smoke issued from one of them. There was a vigorous shout, and a score of men poured out of the larger hut; long haired, some bearded, all in the familiar grey trousers and white leggings. They grouped together at a small boat landing, then burst into a simultaneous shout. Our men in the boats answered with a similar cheer.

“Our permanent garrison,” explained Luka.

%CHAPTER IV. 

\chapter{THE SWAMP OF KARAFERIA.}

There was a time when all that vast swamp country in southern Macedonia was in the hands of the Turkish government. Negotiations were on with a foreign syndicate to drain it and exploit the rich alluvion for agricultural purposes. French scientists and engineers came, explored and surveyed and wrote glowing reports. But like all Turkish business, it went slow. Meanwhile asker built posts on the islands and garrisoned them. During the hot months they retired to the hill villages, for malarial fevers had decimated them the first summer.
It happened that a cheta was hard pressed by cavalry patrols on the plain, one summer, and the Bulgar fishermen lent the outlaws their boats to escape into the swamps. The place took their fancy and the idea suggested itself to them that they could stay there. They burned the military barracks, retaining only two posts for themselves. 

When the soldiers came back they were ignorant of what had happened until a volley of Manlicher bullets ripped through the high swamp cane and carried death to some and enlightenment to others. One young officer, more dense than his fellows, shouted: “We are asker! Kim sin sis?” The answer came out of the cane brake: “We are comitlara! Hurrah!” 

Many of the asker became confused and stepped out on the shining slime, meaning to take a short cut out of danger, and so sank into their graves alive. Without knowing much about tactics the survivors began realizing that approaching an invisible enemy in small, unwieldy punts along a channel three feet wide is giving him a chance to laugh at you, so they withdrew, in spite of the curses of the commander pasha who was fuming three miles in the rear. 

Next day an army corps gathered and bombarded the swamp with four-inch guns, until some foreign officers came up from Salonica and said they ought to be ashamed of themselves throwing four-inch shells at thirty men. Then they built special punts with metal shields and tried to send the whole army corps in at once. But the one approach to the comitaji camp being only wide enough for one punt at a time, it mattered little to the first that fifty came behind. So the government decided it did not need the swamp, and left it to the outlaws. 

It needed no expert knowledge to appreciate the strength of our position, considering the primitive tactics of the Turkish army. There we were bemoted by shallow but unwadable water, screened on all sides by a ten foot cane brake. From the center of the island rose a tall tree in whose branches a sentry kept perpetual watch. From there you saw, first, Olympus, and below, among the foothills, the white buildings of Karafferia, Negush and Voden. In the nearer distances shimmered the open water spaces through which an approaching enemy must pass. 

But these strategic points I appreciated later. The human interest rose first. Hitherto I had known only Luka and Teodor, for the villagers could only quarter us in small groups. The others had been vague, cloaked figures in the night, a line of silent shadows moving over dark mountain trails. 

Here they emerged into the life flush of reality, a crowd of beardless, laughing, chattering boys, picturesque to look at, in colored sashes, some long-haired, though by no means resembling the brigands presented to us by theatrical managers. There was too little personal adornment for that, too many good natured faces. They might, after all, have been a summer camp of militia volunteers on a real, rough outing. The grey uniform was predominant, though in individual cases torn, patched or worn. A distant, passing train, tooting and roaring over a trestle bridge helped to break up my effort at imagining the reality, that I was in an outlaw camp. 

At first they seemed rather shy. “They haven't got over the belief,” chuckled Teodor aside to me, “that you're some sort of an International Commission, with a Macedonian constitution in your handbag.” 

When evening came we all gathered in the big, oblong hut, where all the men slept. A big fire of log roots glowed in the center, the smoke filtering through the rush thatch. Against the rear wall a wide rug was spread over the thick rush flooring, and there Luka, Teodor and I established our permanent quarters. Never was a baronial hall more decorated with weaponry; above each man’s head, as he slept, hung his Manlicher carbine, the sword bayonet and his Nagant revolver. 

Though Luka was a man of military training, had even been an officer in the Russian army, I observed that his relations with the men were quite democratic. They gathered about in a thick semi-circle and needed little encouragement to ask questions, which Luka answered. I quickly saw they were not ignorant of world matters. Most of them had lived some time in Bulgaria, and a number were native born of the principality, as was Luka himself. Two of the boys impressed me especially; they were of the student type, and such they had been, in fact, one being a deserter from the Sofia military academy. 

To correct a false impression that I may have created just here, let me add that Luka was no creature of the Bulgarian political propaganda in Macedonia. In truth, he had once been implicated in a plot to dethrone Prince Ferdinand and to establish a Bulgarian republic. The discovery of the conspiracy resulted in a death sentence for Luka, but this the National Assembly commuted to a short term of imprisonment to censure the prince in some quarrel that happened to be on at that time between Ferdinand and the popular representatives. For a while Luka edited a radical journal; then the uprising of 1904 broke out in Macedonia, and he volunteered his military training. Since then he had remained in Macedonia as chief of the Vodensko rayon. It was on the occasion of one of his annual visits to Bulgaria, made for the purpose of procuring war materials which the Central Committee bought from Austrian merchants, that I had been introduced to Luka in Sofia by Damian Grueff, chief organizer of the revolutionary organization. 

I must have enjoyed a sound sleep that first night in the camp, for when I awoke next morning the others were already about. The sentry was shouting a challenge overhead. I rose, went outside and observed a boat approaching from the cane brake, towards our landing. In it were three men. The first to leap ashore was a slight, dark, middle-aged man in shabby, white, Albanian, skin-tight breeches and a black, loose sleeved shirt, armed not only with Manlicher and Nagant, but also with a silver hilted dagger dangling from his cartridge belt. Had I not known him personally, I could still have guessed his identity, for you see his portraits in all the taverns of Bulgaria, just in that brigand dress. I had met Apostol in Sofia the year before and he recognized me at once. 

If there was one of his rebellious giaour subjects on whom Abdul Hamid expended any personal hatred, it was the voyvoda Apostol. To the casual observed on the outside and to the common peasantry, both Christian and Turkish, he was the conspicuous figure of the revolution. Foreign gendarmerie officers and consular officials, ignorant of interior developments in the organization, opened secret negotiations with Apostol to get his views on various questions of reform, believing him to be the representative of the peasant masses. And I was yet to see the day when the Sultan was to send one of his own relatives to offer Apostol his own terms in withdrawing from the revolutionary forces. 

Apostol was Macedonia’s Robin Hood. For thirteen years he had followed the war trail. In the days before Damian Grueff organized the famous Central Committee, Apostol roamed the mountains, one of those picturesque brigands who have appeared among oppressed peoples during all the semi-barbaric periods of history, their exploits handed down in the folk songs of the peasants. Theirs was the single-ideaed creed of murder and destruction, the first instinct of primitive, illiterate men.
When the revolutionary organization appeared with a programme of co-operative effort, Apostol, unlike many of his colleagues in various parts of the country, offered his services, though its laws forbade a voyvoda to hold back one lira of booty. Moreover, he must bind himself to obey the commands of anaemic schoolmasters in the towns. 

Since then Apostol’s career had been a series of sensational affairs. Only nine months before he and thirty-eight men had been trapped in a village on the river Vardar and engaged half an army corps in a twelve hours’ fight. Artillery, cavalry and infantry, hurried up from Salonica by railway, had unsuccessfully attempted to dislodge him from his position. The band was finally destroyed; only two escaped by plunging into the river after dark. But Apostol was one of those two. 

Three hundred asker had fallen, but the government did not mind that, for they believed Apostol finally killed. A week later the vali pasha received a letter, bearing Apostol’s seal, announcing his recovery from a slight flesh wound. But this unwelcome intelligence was not published. Instead, an emissary was sent to Apostol’s wife in his native village offering him a fat pension abroad if he would only stay dead. Even then, when I met him in the swamp, it was not generally known in Bulgaria that he was alive, and Turkish officers were walking about the streets of Salonica wearing medals awarded them for participating in Apostol’s killing. 

Apostol and his cheta held the second of the two posts taken from the Turkish troops. He it was who had defended them against the attempted recapture. He came to invite Luka, Teodor and me to visit and dine with him. We went in two boats, winding down through the cane brake, emerging suddenly into a lake larger than ours. Apostol's post more resembled a farm; in the lake were tame ducks and geese while about his three huts wandered chickens, turkeys, guinea fowl, sheep and goats. There was a tame ram, the cheta’s mascot, an obnoxious beast that walked in and out of the huts at its pleasure, trampling and waking sleeping men, butting them when they resented his intrusion. Occasionally he was forcibly ejected, but I never observed any permanent results. They told me that he was a most excellent companion on the night marches. He walked at the head of the file, behind the voyvoda. Suddenly he would stop, give a wicked little snort and stamp a forefoot. This warning never failed; somebody would be ahead on the trail, friend or foe. In this way the ram had saved the cheta several nasty surprises. 

It struck me at once that Apostol's men averaged below Luka's in intelligence. Most of them were simple, illiterate peasant boys. Since the swamp had been taken it had become the haven for refugees from Turkish justice. There was one boy of fourteen who had stolen an old carbine from a Turkish field watcher whom he found asleep, and then fled —to the swamp. Another, an anaemic lad of seventeen, a tailor's apprentice, had stabbed a drunken soldier who was beating him. Off he ran—to the swamp. An old man had hailed the post one night, and was brought into camp. He had killed an Albanian land steward. A week later, his son, a mere child, badly wounded a gendarme with a stone. Then—off to the swamp. In this way Apostoi had acquired a large camp following of non-combatants —“Apostol’s orphan asylum,” as Luka called it. About twice a year there was a general clearing up; then a score or two were sent marching to Bulgaria. There the Central Committee housed them until they found work. 

Apostoi complained that he could never keep a secretary long. Luka caught my eye and smiled. Apostoi was not the kind of commanding officer a student would choose. 

Even in the organization there was a prejudice against Apostoi that I never quite sympathized with. The younger chiefs usually referred to him as “that old brigand.” Poor Apostoi, he found it hard to keep up with the evolutionary movements in the organization. It was hard having a pale-faced school teacher coming out from the town to criticise his tactics and inquire into his accounts. It was still harder having his military movements dictated to him by men who had never handled a more offensive weapon than a cane. But he did submit. To the end he remained loyal. 

A child of nature was Apostoi; his naivete was winning. When he related his past exploits it was more like a boy’s babble than conscious narrative. His evident pride in the affairs he had “precipitated,” to use his own pet expression, was so frankly obvious that it did not seem egotistical. When he felt your credulity was being tested, his eyes opened like round pennies, and he nodded his head emphatically; at such moments he was fascinating. 

For a week after our coming we received daily reports from Voden of the hubbub going on there and the movements of troops, all of which were read aloud about the fire and caused huge amusement. But, just as we anticipated, on the eleventh day the noise ceased like the turning of a hand; the officials returned to Salonica and the troops were called in from the villages. We then knew that my letter of explanation had reached New York and its contents published. 

The plans were to return to the hills then, but now came the news through confidential friends that the Greek Church was preparing for a campaign of purification among the nine villages on the plain. Simultaneously with this came information that a powerful band of “soldiers of Christ” had invaded the swamps down near Salonica, using as their base of supplies the genuine Greek villages thereabouts. 

Apostol sent out ten men to scout and they were gone three days. On the second day volley firing came rolling up from the southward, lasting about half an hour. When Apostol’s men returned next day, they reported meeting a similar scouting party of Greeks bound hitherward, but were fortunate in sighting them first They killed four at the first volley and came off with one punt and three Grat rifles booty. It was this incident that kept us waiting. 

For me, at least, the days passed quickly enough. Often I went hunting and fishing, either with Teodor or one of the chetniks. Luka would never join in these sports, considering them frivolous, besides holding certain Tolstoian scruples against the taking of animal life. Next to Teodor and Luka I became most intimate with Alexander, the ex-military cadet. He was a pleasant companion; well read and cynical enough to realize that even the Macedonian revolution was merely a historical incident. Like most intellectual young Bulgarians he had imbibed freely of socialistic doctrines, though with more intelligent discrimination than the average. I could trust his observations as being distorted by neither false enthusiasm nor racial pride nor prejudice. He held even that attitude of disparagement toward his own people, which is characteristic of the Bulgarians in general, a vivid contrast to the fanatical chauvinism of the Greeks. 

Tedor was my most constant companion; both of us had the habit of pacing up and down the space before the huts while the others smoked. He had not only an appreciative sense of humor, but the rarer gift of creating a humorous side to apparently serious subjects. It was this quality which enabled him to relate his own past exploits with no suspicion of egotism.
He dwelt most fondly on his school days. I heard a dozen versions of how they badgered the harmless old Musselman who taught them Turkish one hour each day. They slipped stones into the pockets of his big, loose jacket, and they wrote parodies on sentimental verse he composed. I could enjoy a good deal of this, but finally I would have to jog him on with his story.

The older students organized themselves into a branch of the revolutionary organization, under the supervision of the local agitators, who were, of course, on the college faculty. One year the Bulgarian Church, which is also the educational department for the Macedonian Bulgars, appointed a long, lean sycophant on the faculty who spied on his colleagues and reported their doings to the Exarch. Now the Exarch, while not a mortal enemy like the Greek Patriarch, was by no means a friend to the revolutionary cause.

To get rid of the spy, the professors ordered the students to demand his removal of the bishop. Teodor was chosen as the spokesman of the delegation that waited on the bishop with this demand. To put himself into the proper condition for an eloquent appeal he drank of the cheap wine of the country, whose effects he rather under-estimated, until they entered the warm apartment of the priest. His speech was eloquent, his comrades said afterwards, but hardly tuned to a bishop's ear.

A big row followed, the college was closed for the term and Teodor, with the rest of the delegation, was expelled without the diploma that was to have made a school teacher of him the next year. So a month later he joined a cheta in the hills.

He was a sickly boy in those days, so the voyvoda often sent him into town to recuperate and to make purchases of such stuffs as the cheta needed. One day he met three other chetniks also in town on business, and they went into a cafe for a chat. One of them, in pulling out his purse to pay, turned back the lapel of his coat.

A Greek who sat at the next table caught an instantaneous glimpse of a revolver butt. He had also heard them speaking Bulgarian. Five minutes later all four were suddenly seized in the street and carried off to a police station.

To each of the four torture was applied. They beat Teodor on the point of his chin with a mallet and put hot eggs into his armpits. They whipped the bare soles of his feet with fine wires. You do not feel the blows as they strike, he told me, but hot needles seem to prick and tear at your heart. Still they got nothing from Teodor. One of the others confessed, then went mad. It was a tremendous revelation to the Turks, for hitherto the organization had been a myth to them of which they would have taken little notice had not the Greek Patriarch, with his more accurate information, urged them on. That evening several hundred Bulgarians were arrested in Salonica; among them almost the entire Central Committee, Damian Grueff, the first organizer of the movement; Dr. Tatarcheff, a prominent physician patronized by the foreign colony and married to the Greek consul’s daughter, and a dozen others less famous but no less active. Even Europe was excited. The organization was all but crushed.

The trial was an international affair skillfully controlled by the Turks. The majority of the prisoners were condemned and sentenced to a hundred and one years’ imprisonment in the walled towns of Asia Minor. Teodor spent one year in the fortress of Akia, in Arabia. What was left of the organization sent them money, for the prisoners were expected to provide for their own subsistence, though otherwise they were not ill-treated.

One day the commandant of the prison came to them—they were four together—and said: “Give me twelve liras and I shall help you escape.” They wanted a day to think it over. “Six, then, and you are free,” bargained the bimbashi. No, they must first discuss it among themselves, and they were suspicious.

Teodor had become friendly with the Arab telegraph operator. He sought him out and asked, privately: “Is there any important news to-day?” The operator grinned slyly. “So you know?” he asked.

Teodor knew nothing, but he passed the Arab a medjedia and winked. “Good news makes one feel generous.” “You want the details,” laughed the operator, pocketing the coin. “Well, every political prisoner leaves to-morrow night for Smyrna. But keep quiet, it isn’t supposed to be known till the morning.”

European pressure had forced a general amnesty, and the bimbashi knew how to profit of the news, for with others he had been more successful. Teodor returned to his native village, rested a week and then made for the hills again. In a few months he was voyvoda of a cheta. He went on with his experience as chief. Looking at his face it all seemed so incongruous. A pair of spectacles would so well have suited his eyes. In about the same vein he had described a rumpus at a lecture, he told me how he, with seven of his men, cornered five “soldiers of the Church” in a stone school building; how his comrades drew their fire from the front while he crawled through a window in the rear of the basement, fixed a dynamite stick in the proper place, ignited the fuse and scampered out just in time to avoid going skywards in a burst of flame with the five Greeks.

In a rayon next to his was a cheta sent by a certain Bulgarian general in Sofia who said he knew best how Macedonia should be freed. Those were the days before Macedonians knew what a swine Prince Ferdinand of Bulgaria is. This particular cheta went into the tax collecting business. The persons of the chetniks were adorned with silver chains and gold rings and they jingled Turkish liras in their pockets. Teodor hunted up this cheta and tried to wipe it out, but through a tactical mistake he only partly succeeded.

Public sentiment in Bulgaria was hurt and the stout general wept before great audiences. The organization was in no position at the time to defy the Bulgarian government, so to placate the fat general and his royal master, Teodor was removed from his command. Then Luka was glad to get him as his second in command.

Of evenings we had gay times. There was a young fellow called Satyr, dwarfed in stature and old and wrinkled of face, who constantly acted the fool. He was especially good at mimicry, and one of his feats was the impersonation of the kaimakam directing the movements of his troops from his office in town. He carried on all his business through a window, received dispatches, answered them, gave orders to aide-de-camps, swore at imaginary officers, watched distant movements through his spyglasses so that you had a picture in your mind of extensive military manoeuvres being carried on outside.

His masterpiece, though, was his representation of a hodja delivering a sermon in the mosque. This was usually reserved for Sunday evenings when the hut was filled with visiting peasants and Apostol’s men were present. With a red sash Satyr made himself a tremendous turban and out of the cheese cloth for cleaning our pieces he fashioned a gown. Of a piece of wire he made a pair of huge spectacles.
Then he would take Luka’s copy of the organization’s by-laws, this to represent the Koran, from which he would drawl his half chanted texts. The hodja pauses a moment, slips behind an imaginary curtain, has a pull at a small bottle, after which he humbly begs the Koran’s pardon. In the sermon would be references to persons present and current events of the day. The sermon usually ended in a wild fit of supplication to Allah that these sinning giaours of comitajis might know his instant wrath and eternal punishment. The boys could stand any number of repetitions of Satyr’s hodja, for he never did it quite the same twice. 

Had these merry times continued much longer, I should have been getting the impression that comit-Iuk was indeed a jolly life. But one day an incident occurred which seemed especially intended to remind me that if the grim, ugly game these men played did seem to pause at times, it was none the less ghastly or real for all that. 

I have already referred to the nine villages on the plain against which we believed the Greeks were planning mischief. Of course the Greek bishop was exceedingly anxious to know what was happening in those villages, and who it was that was leading his children astray. The villagers were equally as determined that he should not know. 

One evening as we were at supper, a villager appeared and delivered a letter to Luka. He read it with knitted brows, then shoved it into his despatch bag and glanced about among his men. Of those who had already finished the meal he called five and bid them attend the courier on his return and take their orders from the village committee. 

“We shall probably have guests to-morrow,” Luka said to me as we rolled up in our cloaks to sleep. 

I had forgotten this incident by coffee next morning. Barely was breakfast over when sudden loud calls and exclamations brought me outside. Two boats were slowly emerging from the cane brake, manned by our five men. But what caught my eye was a kneeling figure in the bow of each boat; two men in peasant dress, their arms bound behind them. A sudden, unnatural silence came over the camp; I could hear the lapping of the water against the prows of the approaching boats until they bumped against the landing. Then the crowd quietly pressed back, to give the captives room to land, really, but the feeling was on me as if they were repelled with a sudden horror. 

The first captive to step ashore was a dark, fat, oily skinned man, not yet old, filthy and ill-clothed. Though he greeted us fluently in the dialect of the province, it was obvious that by race he was no Bulgarian. He was slimily repulsive, as though he had just crawled out of the swamp ooze. 

“Good day, Luka,” he said with an oily, self-assured grin, “so here you people live.” 

Nobody answered. 

The other was a gray, shrivelled, mouse-like creature. He was cleaner than his mate, though also in peasant dress. I saw him at that instant for the first time, yet I knew that his face was much altered, that its present expression was not habitual. Not a line suggested hope. 

The crowd parted to let the newcomers enter the hut. Luka took my arm and we followed them in. The captives seated themselves, one on either side of the fire; Luka remained standing a moment, thoughtfully. One by one the chetniks came in, as though casually, and seating themselves about the sides. Teodor and I seated ourselves on the rug. Luka seemed to me like a man trying hard to grasp the first words of a public speech. 

“What were you doing in our villages?” he demanded finally. 

The fat man took it upon himself to answer, as though he were humoring Luka in answering so foolish a question. 

“Why, to buy corn, of course.” 

“They write me from the village,” continued Luka, “that you are from the Greek monastery behind Negush—is that true?” 

The fat man blinked, nor did he answer quite so jovially as before. 

“Yes, I am a servant there. The abbot sent me to buy corn-” 

“Who is your friend ?” interrupted Luka. 

“He’s only a peasant I hired with his horse to accompany me.” 

“Let me see your hand, old man.” The old man held out one palm mechanically, and even I could see that it was as soft as a clerk’s. 

“I am too old to work much,” he explained. “My son supports me.” 

“That’s a lie; you have no son.” Every one turned. It was one of the oldest chetniks who had spoken from the doorway, an elderly man with a full beard sweeping over his breast. 

“What do you know about it?” demanded Luka, sharply. 

“What one neighbour knows about another,” replied the chetnik. 

He walked up to the old man and, pushing his beard back under his chin, held his face down close to the captive’s eyes. “Do you remember me, uncle?” he continued. “You didn’t hire out horses in those days. You were then a clerk in the office of the procuror. When compromising letters were found, it was you who rendered them into Turkish and Greek, for you are clever at learning languages. More than that I do not know against you, but such swine as this gypsy do not hire men like you when they go on journeys. He did not hire you, but you may have hired him. I know you, old man.” 

The captive made no further attempt to answer. The chetnik turned away from him, lit a cigarette and passed outside the hut. 

“Let me see the satchel you took from them,” said Luka, and Alexander passed it over to him. Meanwhile one of the chetniks had made coffee and now served a cup to each of the prisoners, as though they were merely visitors. Another had passed to each the paper and tobacco necessary to rolling a cigarette, but the old man’s fingers trembled too violently to roll his. One of the men finally rolled it for him and offered him a brand from the fire. The two mechanically took the coffee, but they spilled more than they drank. 

Teodor was examining one of the documents produced from the satchel. He could read Greek. 

“The words are mixed,” he was saying, “but they can have only one meaning. Corn is more plentiful nearer Negush than it is here. You don’t have to do a day’s travel in this part of the country to buy a sack of corn.” 

Luka said nothing, but suddenly he rose and went outside. Teodor and I followed presently and found him sitting meditatively on an overturned boat. One by one the men followed us, the two prisoners coming with them. 

Luka rose and began passing up and down. The fat man waved his hand as he addressed the silent chetniks; the old man stood and only twirled his thumbs. Luka turned abruptly and halted. 

“Let them go,” he commanded. “Anton, take them away to the lower landing. Yani, Stefan, Stoyan, you, too. Go, you two—don’t come near our villages again.” 

The four chetniks went inside for their pieces. Anton, the eldest, shoved out two boats. The others seized the two captives and began rebinding their arms. The fat man had lifted his goat hair cloak, but one of the chetniks threw it down again. For the first time the fat prisoner’s assurance left him. He began whimpering, his legs wobbled and his exposed, hairy chest quaked. 

“My cloak,” he blubbered, “give me my cloak. I am a poor man.” 

Luka stooped, lifted the cloak and threw it over the man’s shoulder.
At last the two were seated, each in the forepart of a boat, two helpless penguins. When their backs were thus turned I observed that Anton slipped our large wood axe in on the rushes in the bottom of one of the boats. The chetniks stepped in behind, two in each boat, and shoved off. The crowd ashore, Luka as well, stood watching until the six figures slipped into the cane brake and were gone. 

Teodor and I paced our old path in front of the big hut. The others lounged about as usual, though more subdued. It was not more than twenty minutes when the two boats returned, but no one addressed a word to the four chetniks, nor did they speak one word as they went inside to hang up their carbines. 

A noisy flock of crows went hurtling overhead and began circling about half a mile away. That was just a little more than my imagination could stand, so I went inside. Luka was lying on the rug, his face resting on his crossed arms, and so he remained till sundown.

%CHAPTER V. 

\chapter{THE BURNING OF NICI.}

Trouble came at last. We waited a long time, but we knew it would come, and it came. 

The peasants of the nine villages on the plain called the Slanitza, to whom I have already referred, are of the Slavic race. But the Greek bishop in Karafferia waved his hands hypnotically over them, and said, “You are Greeks.” They, in terror of the soldiers of the Church, obediently responded, “We are Greeks,” though even this they could not utter in the classic tongue, having to resort to a low barbarian jargon called Bulgarian. That happened some years ago, before the agitators appeared in Vodensko. 

It came to the ears of the Greek bishop that his flocks on the Slanitza had invited Luka’s chetas into their villages, that they gave him aid and comfort and food in the swamps, and guided him safely about the country. Then suddenly they ceased paying tithes to the Church, the Greek schools in the villages became empty of children and it was rumored that guns and ammunition were bought from contraband traders. This angered the bishop, so he called the captain of his soldiers to him and said: 

“Kosta, we must guard the chastity of our people. Write them a brotherly letter and warn them against the wiles of those brigand infidels. Be gentle—but firm.” 

Captain Kosta put his pen in his mouth and thought. Inspiration came, straight from Christ, whose figure on the cross he had imprinted on his letterheads. He wrote: 

“February 7, 1906. 

“To my brethren the villagers of Morenova, Lekovishtitza and Yanchesta: 

“We learn that you are being persecuted by the Bulgar brigands, who are trying to force you to renounce your Greek nationality and faith. We learn they are forcing you to protest to the Europeans against the soldiers of the faith, saying we persecute you. 

“I beg you to open your eyes; make no more such protests in the future. I shall make it painful for those that do. Such we shall kill; their wives and children shall not be spared, for we shall cut their limbs asunder. We kill all who do not join us. 

“I hope you understand and will obey, 

“Your brother in the Faith, 

“Constantine Akritas.” 

This inspired missive was sent, turned over to Luka by the villagers and later forwarded to the Balkan Committee in London, that Englishmen might enjoy a specimen of modern Greek humor. 

By reference to my diary I see that trouble began on March 10th. 

It was a clear, moonlit spring night; so brilliant was the light that every stalk of the surrounding cane brake was visible. 

We lay chatting before the fire when a sudden, low, ripping hum from outside set us listening. A pause, and again the ominous “gurr-gurr-r-r.” 

Every man made a break for his weaponry. 

“They’re fighting in Morenova!” shouted the sentry overhead. 

Six boats were shoved out and twenty men embarked. The frantic poling and paddling brought us to a landing in what was probably a short time. Across the plain the firing continued; the gun flashes were quite visible. Off we went on the run, through pools, over ditches and hedges. The house roofs were already visible ahead when the firing broke up into a few scattering shots, then silence. We rushed into the village shouting known passwords. Armed peasants rose from behind rubble hedges all about and came forward to greet us. 

An armed band of about fifty men had entered the village and gone to the house of the president of the local committee. Those that forced their way into the house were masked. The president was not at home, but they began dragging the children and the women folk out. By this time the alarm was spreading and the village militia turned out, numbering about twenty men, and they opened fire on the soldiers of the Church from behind the hedges. The fight had been going on about half an hour when the peasant militia arrived from a neighboring village, upon which the Greeks ran, helter-skelter, giving the peasants a chance to kill five of them. One we picked up then; a tall, handsome fellow with a full beard, on him a despatch bag full of writing material. Obviously he was Captain Kosta’s secretary. Of the peasants one was killed. 

We returned almost at once, fearing soldiers from Negush would be arriving. Next day we learned that soldiers did come, but in a way that suggested an understanding between Captain Kosta and the commander pasha. 

Troops had entered all the villages near Morenova, but through some mistake they arrived a little after the attack began. So the villages had sent out their militia chetas to the rescue of their neighbors. But on their return the soldiers were there to meet them. Most of them succeeded in throwing their guns behind bushes or hedges and so were not compromised, but five were caught and dragged to Voden. To the credit of the kaimakam be it said they were discharged next day. 

“Give it to those Greeks,” he said to one, aside. 

Luka brooded for days over this incident. This programme of burning and killing had been gone through the year before by the Greek bands. In Luka’s district alone near fifteen villages had suffered for supporting the revolutionary movement. In other parts of Macedonia it had been the same. What reader of European papers does not remember the massacre of Zagoritchne in 1905, where women and small babies were bayoneted in broad day by the hired soldiers of the Greek Church. With a restraint born of a misunderstanding of European civilization, the Central Committee had forbidden reprisals, beliving that so European sympathy could be gained, as though politics were moved by any such human weakness. Luka was more a man of the world than his colleagues, and he realized this mistake. 

Teodor, too, was moody. One evening, on our after supper walk, I drew him out. I had small doubt of fiery little Teodor's sentiments. 

“Luka,” he said to me, in a low voice, “is thinking deeply. I wish you'd talk to him. You have influence with him. It isn’t my place to broach the subject until he invites me.” 

“I am not enough of an expert to judge of the situation,” I replied. “Anyway, theoretically, I can’t approve of a reprisal on people not directly guilty. We can’t get at the andari themselves, you know.” 

Teodor looked at me for a moment. 

“You don’t understand me,” he replied. “I feel like you. To burn the houses of non-combatants, even if not a soul is killed, doesn’t taste well to me. It isn’t fighting—leave that to Greeks and bashi-bazouks.” 

I approached Luka, and he seemed relieved to have me broach the subject first. 

“Yes,” he admitted, bitterly, “I am going to do it.
Acritas and his cutthroats are sheltering in the swamp, just below a Greek village called Nici. They have no regular stronghold, but hide about in the bulrushes where the devil himself couldn’t find them. From Nici they get provisions, couriers and guides. If we burn Nici they will have to move. And we might be so fortunate as to meet on a fair field. If we can’t get at the reptiles themselves, at least we can burn their nest.”

“Next day,” I replied, “all the papers of Europe will publish exaggerated accounts of the affair, saying how many children and women were killed. The good opinion of Europe will be lost; they will say all are alike; Greeks, Turks and Bulgars.” 

I was about to grow eloquent, but Luka interrupted me with a laugh, the genial, hearty laugh which I had been so used to till a few days ago. “You are amusing,” he told me, “quite worthy of your countryman, Mark Twain, smashing old traditions by defending them with a straight face. A humor peculiarly American, I suppose. 

“But seriously, my dear fellow, you are not talking to Teodor now. You and I have knocked about the world; both of us have done journalism, and so been behind the scenes a few paces. How many people in Europe know whether we are revolutionists or only brigands out for loot. That isn’t what holds me back; I only hesitate before a distasteful act. Frankly, I would prefer dealing with the guilty ones. If you can show me how to get at the Greek bishop, I shan't burn any villages."

Still, I insisted, and I found no difficulty in persuading him, for he liked it no better than I. “But," he added, “if they make one more attack it’s our turn." We shook hands on that. 

There was little time left me for satisfaction over my own powers of persuasion. That night we were aroused from sleep by the cries of our sentry. Out we rushed. Above the cane brake to the northward the sky glowed crimson and a column of black smoke and sparks rose perpendicular in the calm night air. There was a heavy detonation—a dynamite bomb had been thrown. 

One of Apostol's boats came scurrying into our landing with a messenger asking ten men. The attack was on his territory. We should remain and guard against possible attacks nearer home, for the Greeks had crossed the swamp not far off, and this might be a ruse to draw our strength away from the islands. 

The flames, visible from the sentinel tree, soon died down. There had been little firing. Teodor had gone to join Apostol. Towards morning they appeared. They had seen the andari re-embarking and their boats making off south toward Nici They had not even paused to return Apostors volley. Two houses, standing a little outside the village, had been burned and one villager killed. 

“We shall have more such illuminations yet," remarked Luka. There was no further discussion of Europe’s opinion. 

We slept then, but were up again shortly after sunrise. The people in Karafferia had been organizing lately, and a week before they had sent us a present of a dozen bottles of beer. The empty bottles were now filled with petroleum. Every man carefully overhauled his carbine. 

At noon twenty-five of us embarked in eight boats, leaving behind only six men. Apostol and his men were waiting for us. In all we who were going numbered fifty-five. Teodor went ahead with five boats as patrol. The rest of the fleet followed a few hundred yards behind, one boat close behind the other. We traveled in a southwesterly direction, toward the open waters of Lake Enedgee, through a region different from what I had seen before. There were no trees, only swamp grass, sometimes so short that I could stand and survey the brown sea of waving grass tops for miles around. Though our progress was by poling, we went at a rapid rate. In places the grass thinned out and the waterways widened. A nervous kind of hilarity possessed us, manifest in a good deal of chaffing and humorous references to the Macedonian Navy and Admirals Luka and Apostol and a general inclination to give absurd commands couched in sea terms. 

In the middle of the afternoon we swung into a broad, deep flowing river, which we followed for a couple of hours, carried along by the current. An hour before sunset we entered the grass again and presently came to a small island on which stood the charred timbers of a hut. Here we disembarked to eat a supper of bread and cheese and to rest the paddlers. Our destination was now so near that nobody spoke above a whisper. Luka gave out the final instructions; the passwords were “Macedonia” and “Freedom”—who did not answer the second word to the first in the dark was an enemy. 

It was just dusk when we drifted slowly into an open space among low bulrushes. By standing we could just make out the rising land, and at the foot of a low hill, not half a mile away, we saw houses and figures walking among them. 

It was quite dark when we moved on again, our boat leading, for Luka, Apostol and I were together, and the lad who paddled us had been a fisher here a year before. Finally we bumped against the solid, just below what appeared to be a mole running out into open water. The other boats quickly ranged alongside, prows on to the land, and we all leaped out and gathered on a path on top of the mole. A full, red moon was just rising. 

A few hundred paces further on brought us to broad, level ground. The road continued parallel with the shore, but a smaller path forked up toward the village. 

Apostol and his men headed for the upper part of the village while Luka with about twenty men took the lower end, each man carrying a bottle of kerosene in his left hand. The rest, eight of us in number, advanced a few yards up the main road and there remained to guard against surprise from asker who garrisoned a village about half an hour distant. 

As we sat there in the dust of the highway, I could hear the steadily increasing uproar of dogs as they scented the approaching strangers. Somewhere out in the fields a shepherd whistled to quiet his alarmed flock. The roof tilings were quite visible in the increasing brilliance of the moonlight, but nowhere could I see the dark figures that I knew to be moving in on the village. That interval of waiting was the worst of it all. 

My heart gave a leap; a red ball of light crept up the wall of a house and expanded, glared, then burst into a streak of flame, spurting up from under the eaves. If there were human shouts, the howling of dogs drowned them. Then: “gurr-gurr-gurr-r-r!” three ripping volleys from the Manlichers. 

Another house at the lower end of the village began burning; the first was already blazing furiously. The roof dropped in with a dull crash, sending up a fiery cloud of sparks. Now we distincly caught the crackling of flames and the deep roar as of a furnace. Running figures silhouetted themselves against the lurid background. More houses were burning, a whole row of them. In one began a furor that sounded like the roll of musketry—one of Captain Kosta’s caches going off. From each separate blaze rose a column of black smoke and blazing brands that united overhead in a great, rolling, black cloud. 

Time cannot be reckoned during such events. An hour, perhaps two, had passed, and the whole village was the base of a roaring, twisting column of flame.

Meanwhile, we eight sat silent spectators. My eyes were as much up the road as toward the conflagration, for the terrible inactivity rendered me keenly sensitive to fear of asker. I saw a flash, then came a sharp, loud report. I remember exclaiming: “The idiots! Why don’t they come? Here are asker on top of us!” Then came the crackling roar of a volley; not Manlichers. “Not asker!” exclaimed Anton, who was beside me, “Andari.” We quickly sprang to our feet, facing up the road. Captain Kosta had come.

On the instant a second volley from the village, or a burst of scattered shooting, rather, for our men were much spread out. We eight made no reply, but advanced quickly some distance up our road to secure an advanced position before being discovered. There we dropped flat on our bellies. To our left the Manlichers continued popping, almost drowned in the roar of the Greek Grat guns. Judging by their fire the Greeks about equaled us in numbers. The firing became general, and we could hear the whining of bullets and their wicked spats against the walls of the houses.

The Greeks were not advancing, but seemed edging along the road toward us, intending, no doubt, to cut off retreat to the boats. Then we eight blazed forth, firing as rapidly as possible to give an impression of numbers. The whistle of bullets now came toward us, overhead, like the whirring flight of metallic birds. A few times we heard sharp plunks in the mud near the water's edge.

Suddenly the Greek fire thinned down to straggling shots, then ceased. Off in the distance, but not from the village, came several muffled reports. “Asker this time!" exclaimed one of the men. Apostol and his men swept in, and we ran for the boats. Luka and his boys followed. One last volley, then we embarked and shoved off.

Again we were in the cane brake, no one missing, no one hurt. We traveled all night, but ever behind us glowed a lurid sky, fading only with the morning light.

%CHAPTER VI.

\chapter{THROUGH THE MOUNTAINS OF VODINSKO}

What the consular reports said of the burning of Nici I have never learned, but the newspaper accounts throughout Europe were vivid enough in descriptive quality. Those French papers inspired from Greek sources, had it that five hundred men, led by Bulgarian army officers, had slaughtered men and women alike. We had it from the peasants that Kosta’s men swaggered about the streets of Negush boasting how they had driven us back to our boats in such disorder that we had barely time to carry off our dead.

Sifting it down: thirty houses, about half the village, were totally destroyed. Six dead were found on the field by the soldiers next morning. One was the Albanian steward of the bey, who, believing the village was being attacked by bashi-bazouk, rushed out shouting he was a Musselman. Some of Apostol’s men not taking this untimely announcement in good part, killed him out of ill temper. Another of the dead was a youth of sixteen struck by a stray bullet, while the third had been wilfully shot, Luka’s men having recognized him as a noted informer long ago fled from one of the Slanitza villages. These three we knew of ourselves; the others were undoubtedly Kosta’s men whom he had left behind in the dark.

In truth, not only the villagers, but their cattle, had been given ample time to get out, our men even assisting in driving horses and oxen out of the enclosures about the houses. Only one young woman, far gone in pregnancy, had remained behind with her husband. These two had remained with Luka during his stay in the village, and when he left he gave them a letter addressed to their people in general, in which the Greek peasants were informed that their homes had not been destroyed because of race hatred, but as a protective measure against Kosta’s raids on villages sympathetic to the revolutionary cause. If they would refrain from harbouring the mercenaries of the Church, no more Greek houses would be burned.

This warning must have had its effect, for thenceforth Kosta’s activities ceased. No more Bulgar villages were burned that year. Later on peasants going to and coming from market were waylaid on the highroads and killed, but always away from the villages. Nor were these assaults committed by andari, but by masked men who were obviously citizens of Negush, the stronghold of Greek terrorists.

Months afterwards, while the Olympic games were on, I read newspaper reports of a dinner given to distinguished foreigners then visiting in Greece at which one of the speakers was Captain Constantine Akritas of the regular Greek army, “just returned from a tour of investigation in Macedonia.”

For several weeks Turkish troops made a determined effort to blockade us in the swamp. All the watercourses and landings were ambuscaded at night, but except that we were brought down to a diet of bread and cheese after Apostol’s farm stock had been sacrificed to our needs, we experienced no real inconveniences.

Spring weather was melting the mountain snows and the freshets poured down into the plain and flooded the swamps. One day Apostol and over half his men left us for the hills. We too, began overhauling knapsacks and footgear, and when the first rush of waters subsided, fourteen of us, including Teodor and Alexander, embarked one evening and landed on the edge of the Slanitza just as dusk fell. Before us spread the broad, flat plain, and beyond circled the dark blue mountains.

It was fine, it was exhilarating, to be tramping the hard, cool earth again, Luka’s tall figure swinging ahead, the file following with swift, noiseless steps, trotting rather than walking, for the energy that was in our legs.

Once there was a warning call from the patrol ahead, and we squatted, with pieces ready. The patrol fell back on us with a badly frightened Turkish peasant who had been out after a straying horse. An uncomfortable feeling came over me that the poor fellow would be sacrificed to our safety, as there were soldiers in the villages we were passing.

“But Luka only lectured him on his imprudence at being abroad so late. “I suppose," he continued, “if I let you go, we'll have asker on our trail before an hour." “By God, no," the Turk swore in broken Bulgarian ; his teeth chattered. “You seem like an honest fellow," said Luka, “I believe we can trust your word. Go, find your horse, there are no more of us about. But take better care of yourself in the future: sometimes we challenge after we shoot."

I thought he would kiss Luka's hand, but one of the chetniks shoved him away good-naturedly, and we passed on. “Won't he talk?" I asked. “Not for a week," replied Luka. “If we hadn’t caught him, yes. He’s our friend forever, now. Turkish temperament doesn’t adapt itself to spying; you need giaours for that. When he thinks we're well away, he will talk. He will then be the hero of a personal encounter with fifty wild, long-haired comitajis, who finally overcame him by force of numbers and spared him out of admiration of his courage. Every time he tells the story our heights and the lengths of our beards will increase by an inch, and the eyes of his simple, peasant audience will bulge and their heads will wag as they exclaim: “Mashallah! Mashallah! Terrible people, those comitlara! What a pity they are not of the true faith!"

On we went toward the mountains, all that night.
Towards morning we struck rising ground and when day broke were crawling up a deep ravine to the shoulder of a lofty, wooded ridge. But on looking upward we still seemed only at the base of those huge, jagged piles of cliffs. We camped in a sparse pine forest. Looking downward gave a sense of our great height; the sixty kilometers across the plain we had marched in the night seemed humiliatingly compressed. Off under the morning sun glistened the waters of Lake Enedjee, on whose shore Nici had stood. And still beyond, more to the right, you could plainly see the gray walls and cupolas of Salonica against the deep blue of the Aegean. 

It was crisply cold up there. The deciduous trees down in the ravine were still bare, but through the thick carpet of pine needles on which we slept that forenoon crocuses and primroses already appeared. 

Early in the forenoon we resumed the trail. A few hour’s slow marching brought us out on a small plateau where a hut stood screened in scrub oak. Here we found five men; three charcoal burners and two swineherds whose beasts rooted in the thick mast among the trees. 

The peasants greeted us fearlessly and cordially. They were Vlachs, therefore friends, for the Roumanian Church had lately been created in Macedonia as a protest against the tyranny of the Greek ecclesiastics. For this act of secession the andari had retaliated severely on the Vlachs with burning and murder. The five peasants gloated over our details of the burning of Nici. 

“You ought to come to our village, Luka,” said one of the swineherds. “The spirit is there, but they like to see the cheta once in a while.” 

“You want another ‘affair’ in your town,” retorted Luka, with dry humor. “I thought Mesemer had enough of us last year.” 

“Clear out before dawn, that’s all,” suggested the peasant. They continued talking it over. 

I caught the name, and immediately there floated through my brain one of the prettiest and most popular of the folk songs. It may be a borrowed melody, of that I don’t know, but many of the younger citizens of Voden suffered imprisonment for innocently whistling it in the street. 

Mesemer is an outlying village from Voden, just beyond the barracks. Partisanship runs strong here, for, as it is in Voden, part of the population is Grecoman, Bulgars still loyal to the orthodox church, consequently loyal to the Sultan as well. But there were secessionists, and they had long ago organized a local committee. Occasionally the cheta came to for the latter left us soon after to prepare our reception in Mesemer. The four other peasants roasted us a young pig over a glowing bed of charcoal embers. 

It was dusk when we continued our march. All the trail was downward, through heavy forest. We passed over ground that was historic to Teodor, for last year he had invaded this territory and engaged Kosta’s predecessor. Captain Theodosius, and his band. While they were fighting, each force peppering the other from the shelter of rocks, asker came up from Voden and then began a three cornered, running battle. In one place a wounded Greek fell over a precipice and was caught in a clump of blackberry bushes. As we passed this spot now, Teodor, and those men who had been with him, became intensely excited on discovering a piece of weather stained cloth clinging to the thorny bushes below. This, they claimed, was part of the fellow’s fustani, the short, fluffy skirt which is part of the national Greek costume. 

“It looks like a ballet girl’s skirt,” Luka described it to me, “and it is fittingly suggestive of the men who wear it.” 

Darkness came, but soon the full moon rose. We came out on the brink of what seemed a deep amphitheatre. Far below glimmered the lights of numerous habitations. Two men were waiting for us just here. 

“We shan’t be caught in a trap this time,” said Luka; “only five of us shall go down. The rest must stay up here and keep a path open in case of unpleasant accidents.” 

So Luka, Teodor, Alexander, one other chetnik and I began the descent, using hands as well as feet, the two peasants guiding. The thing was to prevent small stones from rolling down, for the night was still and slight noises could be heard far. Finally we dropped into an orchard, glided down in among the trees, came into a back garden and slipped into a house. 

The room was crowded with men and some few women who greeted us cordially, but they were evidently in a tense state of uneasiness. Most of the men were elderly, rough, rugged fellows, each with about ten thicknesses of broad sash about his middle from the armpits down below the hips. All were with shaven jaws, for a beard in Macedonia, outside the towns, is considered little short of treason by the authorities, only because the comitajis are popularly believed to be all bearded. The exception is the village priest, whose beard and hair are never cut. 

They all listened attentively to Luka’s short, incisive talk. Afterwards they entered freely into discussion. The new laws of the organization were interpreted to them, for only six months before the first representative congress of the revolutionary masses had framed a constitution which granted universal suffrage to the peasants in the affairs of the committee. Hereafter, in theory, at least, no more uprisings could be declared without the consent of those who must fight in them and suffer most in case of failure. Hitherto local committees had been appointed by the rayon voyvoda; henceforth the people would chose him. There was now no time to elect a new local committee, but Luka instructed them in how it must be done and told them to send a list of the new members to the rayon committee in Voden. 

It was a little after midnight when we departed and rejoined our waiting comrades, the two peasants coming with us to the top of the bluff. 

Speaking of those two same peasants, I may here add what became known to us only a week afterwards through a letter from Voden. Though we had been less than four hours in Mesemer, and entered and departed as secretly as possible, yet it was known next day to the Greeks that we had been there. The two who had been our guides were found the next evening in the forest with their throats slit, and on one was pinned a letter bearing the familiar crucifixial seal of the military arm of the church. Said the letter: 

“Such is the fate of all those traitors who would betray their Church by serving the infidel brigand, Luka. Beware, you of the guilty hearts; when the brigand entered your houses last night, the eyes of the Holy Church was on him and the guilty ones.” 

There was no Sherlock Holmes there to investigate the murder, but even to one less skilled there would have been no difficulty to trace the crime home. Perhaps Gregoraki in Voden could tell. 

We spent two days, after leaving Mesemer, up among the crags, but on the third evening we crossed the valley and ascended into more comfortable territory. Here we were three days entertained in the largest village in Luka's rayon. A militia cheta of seventy men met us in the forest and escorted us in, with banners flying, one might say, for it was broad day and no asker near. The chief of militia was an old white bearded priest.
Like all the secessionist priests, his hands were horny from manual work; he was a carpenter and kept bees. I enjoyed immensely coming on him next day sawing a plank, his black, patched cassock girt up and his white locks escaping about his shoulders from a knot behind. 

Here we were safe from surprise, for on every eminence surrounding the town a woodcutter, ostensibly chopping wood, watched every possible approach. There were two Musselman families in the place, the men being blacksmiths, a trade seldom followed by Christians there, but they were as staunch as the other inhabitants. Said one of them as Luka and I came up: 

“Welcome Bie Luka; when you enter town, God strikes me blind and deaf.” 

“Which proves that you are under His protection,” replied Luka. “He wishes to preserve you for a long life.” 

We visited the school and I was made to feel much like a member of the school board on his rounds. The children sang us the Mesemer song and others of a similar nature. The teacher was a young girl in European dress and spoke college Bulgarian. She motioned to one of the older pupils, a girl of about thirteen. The child came out in front, declaimed a welcoming address of which I seemed to be the object, and then presented me with a bunch of wild flowers. Long after I learned from Alexander that it was Teodor who had planned this ceremony with the teacher. It seemed so characteristic of what Teodor was, but seemed not to be. 

Late that afternoon I stood in the upper window of a house overlooking the low rubble walls of an unfinished house on top of a small hillock. A crowd of children were congregated there, and I became interested in their play. 

About ten of the eldest boys were up among the walls, kicking up a tremendous din, shouting down to about thirty youngsters lying at the base of the incline. The boys of the larger group, some of them not more than four, were raking sods into their left arms with their right hands. Suddenly they made a simultaneous rush up the hill, shouting “Allah! Allah! Allah!” The boys above received them with a cloud of flying clods, yelling “Hurrah!” all the time. Then they all met in battle, which was hot while it lasted, but it was obvious at once that the attacking party must be defeated. Sure enough, presently they were all rolling over on their backs, the besieged running about prodding them with sticks they had in reserve. 

An abrupt resurrection of the dead followed, and they drew together in what I thought was a peace conference until I was enlightened by a piping voice in a wail of protest: 

“I don’t want to be asker this time.” 

It was a favorite game with Macedonian children and I have seen them play it in the vacant lots of Monastir city before crowds of passing Turks, asker among them, whom it seemed to amuse as much as it did me. In truth, whatever may happen during massacres, for I have never attended any, Turks are wonderfully tolerant of children. 

Again we continued our northward march, agitating in the villages, auditing accounts, holding elections and settling local disputes, appealed from the local committee, for it was part of the Committee’s new tactics to boycott Turkish courts. Most of the cases were boundary disputes and always sapped Luka’s patience. 

“Let it be known,” he said after his decisions, “that this holds good only until Macedonia has a constitution; then you may re-open your cases and wrangle them out in a civilized court. You Bulgars delight in such things, but you have got to devote your energies to more important work just now.” 

One night we entered a village on the edge of a lake, picturesquely situated at the base of a rocky eminence, on the top of which was builded a pretentious edifice. 

“Whose is that feudal hall I asked Luka, meaning to be facetious. 

“Whose, but the feudal lord’s?” replied Luka, I looked at him enquiringly. 

“You haven’t got used to your environment yet,” he continued. “In the twentieth century the peasant is exploited by commission agents and stock exchanges; here, in the fifteenth century, the land barons have not been displaced. We call them beys, and they only take half of what the peasant produces. That, you must admit, is comparatively reasonable. But wait; to-night you shall participate in a scene from one of Scott’s romances.” 

We had not been seated half an hour in the usual circle of garrulous peasants, when in stalked a most magnificent personage in a flaming red fez, the tassel from which hung down over his shoulders like a woman’s hair. He wore a gold laced vest, which left free his white, starched sleeves to the elbows, so wide and ruffled that they resembled wings. His baggy trousers disappeared at the knees in thick, white, felt leggings, and so begirt was he with ammunition belts, so hung with daggers and other murderous weapons, that our own equipments paled into insignificance. So huge and stiff was his mustache that birds might have perched on it. 

He bowed to Luka, touching, first his breast, then his lips and last his forehead. Having delivered this Oriental salutation he suddenly gripped Luka’s hand and said, “How are you?” in the commonest peasant dialect. Teodor and me he greeted in a similar manner. I understood then he was the bey’s steward, come to convey his master’s invitation to supper. 

An hour later Luka, Teodor and I scaled the rocks. The double doors of the mansion were open and the huge hallway inside lighted with lamps. A crowd of armed retainers were there to meet us, all in that picturesque Albanian costume, and evidently men picked for their husky statures. The bey himself stood in the lighted hallway, a princely young fellow of refined features, clothed less ostentatiously but more richly than his followers, armed with bejewelled and filigree work weapons. He welcomed each of us with a kiss. Then he led the way up a broad stairway until we came into an upper hallway, or ante-room, and here we deposited our guns in a corner. The bey entered another doorway, turned around on us and made a slight salaam, then led us into a large room, bare of any furniture, except that on the floor were spread decorative rugs, richly embroidered cushions and in the center stood a round table, raised from the ground not more than two feet. Again the bey touched his breast, lips and forehead to us, and we all squatted in the rugs. 

Later I found an opportunity to whisper in Luka’s ear, and in German: 

“Do you mean to say it’s safe to disarm in this man’s house?” 

“He’s an Albanian,” Luka replied, “and this is his house. We have been received as guests. Some day he may cut my throat, but just now he’s the creature of his race traditions. If asker came to trouble us now he would be the first to open fire on them.” 

A grave, serious young man the bey seemed to be. He spoke Bulgarian with a peculiar accent, for neither Luka nor I knew Turkish. The conversation was at first desultory; he told us he came for a week’s hunting and that his health suffered in town.

Later we discussed the Greeks and even the burning of Nici and questions of local administration touching persons I did not know. I felt from our host’s attitude, hardly expressed in words, that he knew me to be the American who had caused the stir in Voden, but he asked no questions that were personal. What I was doing among Bulgar comitajis was my business, not his. Nor did he diffuse any Oriental flattery; he neither commended nor deprecated our business. Once he referred to the Young Turk party in a manner that suggested sympathy with its principals.

The supper was presently served by two of the retainers who had laid aside their weapons. There was a whole roasted lamb, several baked chickens, broth, bread and excellent cream cheese. The steward joined us at table, so there were five of us. The meat was skillfully torn apart by the bey and the steward and served around; there were no knives or forks. Later wine was served, and the two Moslems drank their share, the Koran notwithstanding.

It was midnight when we departed, with a general exchange of kisses as before. One of the retainers had been sent ahead to announce our coming to the cheta, so we found them waiting for us in the open. Immediately we struck up into the mountains, the village being too near the railroad track to be safe in day time.

I was yet to meet other beys. Many of them saw it to their interest to be friendly with the chetas, for the new policy of boycott ruined many Turkish landowners. The Committee, on its side, found it advantageous to suspend the boycott in the case of friendly beys, for not only did the peasants suffer from it, having to till inferior soil, but with the presence of the landowner in the village asker and andari were not likely to disturb the people. It is only fair to add that there was something more than self-interest in the friendliness of some of these Musselman landlords; some were sincere partisans of the Young Turks, the reform party, desiring a constitution, and there were even many more radical with distinctly socialistic tendencies.

My last day with Luka came. We were encamped on a mountain that marked the northernmost boundary of his rayon. Here we met two men sent by the voyvoda of the next rayon to escort me into Lerinsko. Alexander was to accompany me, to be my constant companion on my wanderings. Though we had been only two months together, it was not pleasant to part from Luka and Teodor. The boys, too, took it sorely; Anton, whom I had not quite liked, he who executed the two spys in the swamp, blubbered like the naive child he was.

We fled down the trail, looking back only once to see the cheta at the edge of the forest above, waving their kerchiefs and hands to us. Many long months it was before I saw them again, or those of them that survived the risks of warfare.

%CHAPTER VII. 

\chapter{BECOMING TURKISH SUBJECTS.}

Our first ten days in Lerinsko I pass over cheerfully; they remain to me only as a memory of a bleak, stony region where we spent the Easter holidays dodging patrols on the hunt for comitajis who might be venturesome enough to join the festivities in the villages. The sub-cheta we were with consisted only of two outlaws and eight militiamen, peasant boys doing a few weeks’ temporary service. They were a dismal ten days; no child ever wanted its daddy so badly as I longed for Luka or Teodor and the pine-clad mountains of Vodensko. Teodor had, indeed, planned to accompany me on my journey, whereupon Luka had become keenly excited.

“No! No!” he cried, “take Sandy, he’s an intellectual. Leave me Teodor.” I thought that childish at the time. Now I understood. Luka’s first year in Vodensko must have been a desolate one. As these reflections came over me I caught Sandy’s clear cut features and dark gray, intelligent eyes, and I was reconciled. Now I realize it was better so. To have gone through my experiences with Teodor would have left much of the lesson untaught to me. He was possessed of that unconscious enthusiasm which overlooks the weak points in a situation, seeing only what is best; a temperament which is contagious, but deceptive. Sandy was not so. Perhaps even more than I suspect it was his influence which decided that my stay in Macedonia was not to be permanent.

One evening Sandy and I bid our simple peasant comrades goodbye and, accompanied by a courier, began a night’s march down into the Monastir lowlands. A few hours before morning we struck rolling country. Next, we were challenged by an outpost stationed to meet us and guide us into the village where Tanne, the Lerinsko voyvoda, was to join us next day.

We came abruptly in among rows of stone houses. Behind the garden walls I heard muffled growls. Not a dog had announced our arrival, though half an hour before we had heard them baying from the foot-hills.

“Asker must be in the village,” I whispered to Sandy.

“The railroad passes near below,” replied the courier, “and the barracks aren’t fifteen minutes away.”

Of people we saw not a sign; the houses seemed as lifeless as the ruins of antiquity. We turned a corner, and I caught the reflection of a faint light through an upper window. The guide called softly, and a door opened wide enough for us to slip through into a garden, thence into the house.

The family, a young peasant, his wife, two children and an old grandmother, were awake waiting for us. They had a broth boiling on the hearth for us, and the three of us ate. The guide stayed to ask the news from up country; had the tax collector been around lately, were the andari brewing mischief and how were the local garrisons disposed? He on his part was thoroughly posted on the disposition of troops within a day’s travel and their probable movements for the next twenty-four hours. This was a study the Macedonian peasants seemed to me to have developed into a science, even to gaining an accurate knowledge of the psychology of a Turkish commander’s brain. I have seldom known their forecasts to fail, or seldom been in dangers they failed to forecast, and often avoided dangers they warned us against. Nor shall I ever forget with what consummate skill they can ward off the climax to a dangerous situation wherein Turkish asker are concerned.

After eating, and still continuing our conversation, Sandy and I began slowly to unharness; first our ammunition belts, our bomb holders, then our leggings and our uniforms. All these articles were tightly rolled up in a bag and hidden in a hole under the brick flooring. The women and children having retired, we disrobed entirely, for our heavy, skin-tight, woolen underwear was no part of a peasant costume.

The young peasant gave us each a suit of coarse linen underclothes, almost as heavy as duck. Over this went baggy trousers, like slack balloons about the hips and uncomfortably tight below the knees; home knitted socks, a jacket of quilted cotton stuff, five yards of red sash about the waist, and ox hide moccasins, fastened by leather thongs winding up the leg. Last of all we were given each a dilapidated kalpak, resembling a fez and skull cap crossed. As a finishing touch I was shaved and Sandy’s yellow locks were shorn to conventional length. We looked at each other a moment, then burst out laughing, the peasants joining in out of sympathy, for surely they could not have appreciated how grotesque we two appeared to each other. Finally we rolled up in blankets to sleep, supposedly safe from the intrusions of inquisitive asker.
The hum of low conversation awakened me in the morning. Sitting up, I saw a man in European dress talking to our host. He turned, observed that I was awake and advanced with his left hand extended; the right sleeve was empty.

“Good morning, Bie Albert,” he greeted me, smiling.

“You are General Nogi,” I ventured. He laughed at my using the pseudonym with which he signed his official correspondence.

“I am he,” he admitted, “also the schoolmaster.” He was the secretary of the local committee, and as such had arranged our reception.

After the militia cheta he was refreshing company. While Sandy and I breakfasted on hot sheeps’ milk and bread, Nogi chatted with us, and when we had finished the three of us went out for a stroll. The village lay in a hollow between hills of vineyards except on the side toward the railroad track, so near that we could see two soldiers lounging beside a bridge they guarded. Most of the men were out hoeing the vines, but women and children were about and greeted us with smirks and knowing nods. We finally came on some old men loafing before the inn, and they dragged us in to drink plum brandy, a red-hot drink meant only for a peasant’s throat. The innkeeper’s son had lately returned from America and we two discussed the labor problem in the Pennsylvania coal regions. He spoke English, though of a sort that left us not a wide range of subjects of conversation to choose from. We did finally resort to Bulgarian, though not till the audience had been properly impressed with the young fellow’s accomplishments. As we sat chatting a woman came hurriedly in, crying:

“Asker are coming; the Monastir road swarms with them!”

Sandy and I instinctively jumped to our feet, but the schoolmaster and the old men took it quite calmly. The young immigrant went into a back room and returned with three hoes.

“It’s alright,” said the innkeeper, “but you had better walk out into the vineyards in case they stop here and get talking with us.”

“Your disguises are meant for the perspective,” the schoolmaster explained further, “nor are your tongues hidden in local costume.”

Each of us took a hoe on his shoulder and accompanied by the innkeeper’s son we started leisurely down the main street and turned up a lane toward the vineyards. The mounted vanguard of a large body of troops were already entering the village. So close were they that I could distinguish the white mustache of the old officer who rode at their head. Of course, we knew there was little danger, but I, for one, felt keenly that fear of black uniforms in great numbers common to all those who have carried arms illegally in Macedonia.

“You don’t enjoy such a sight when you hold only a hoe over your shoulder,” remarked Sandy.

Up in the vineyard, among scores of other workers, we hoed, watching from the corners of our eyes the cavalrymen as they watered their horses at the spring in the main street below.

An old fellow played on bagpipes, to the rhythm of which the peasants worked, some singing.

“Look out there, Yani,” a girl called from further up the hill, “don’t be hacking the roots.” The men about us broke out into laughter. I looked around and saw the girl blushing, mortified at her mistake. It seemed I had Yani’s trousers on, recognisable at some distance by a great, square patch on the seat thereof.

“Yani, or no Yani,” cried the old piper, who had ceased blowing to laugh, “you were right, Stoyanka, Teach him how vines are hoed; he’ll be an apt pupil if you are the teacher. Come, you aren’t always so shy.”

“Blow your pipes, dedo,” retorted the girl, pertly, “that’s pleasanter music.”

The bandying of retorts continued, some of the others joining in. The work in our neighborhood slackened and young men and girls came over to our group to enjoy the diversion. A young fellow in semi-Albanian costume came sauntering over the brow of the hill; I started, for he was armed; evidently the bey’s overseer, the becktchee.

“Resting again, you people!” he shouted.

“Don’t make such a noise,” retorted Stoyanka, “the bey isn’t near.”

The becktchee grinned and winked at me.

“Well,” he cried, putting aside his gun, “don’t stand cackling like hens. If you won’t work, let’s dance a horo. Puff up your bags, old dedo.”

The crowd responded with manifest good humor. There was a clear, level space where the becktchee stood, and about ten of the men and girls gathered there. The pipes uttered a preliminary bray, the becktchee grasped the hand of the peasant nearest to him and they began treading the slow, rythmic steps of the horo, the national dance of the Bulgars. The leader twirls in his free right hand a kerchief, swings two steps forward, one back, two forward, one back, twisting in and out and about, the rest following in a sinuous line, all their movements in harmony with the leader’s and the rhythm of the music. As I knew from Bulgaria, the peasants in the fields often pause for half an hour’s such recreation, so I thought this entertainment quite spontaneous.

I cast a side glance down into the village; the soldiers were strolling about the streets.

“When they find us jolly,” remarked a man beside me, “they suspect nothing. There will be no muster call.”

They tried to get Sandy and me into line, but we held back, preferring to witness. It was a pretty sight; the men, in somber brown, serving as a contrast to the white and red embroidered costumes of the girls, swinging back and forth with the harmony, if not the variety, of motion of a trained ballet, most of the girls quite as attractive as professional dancers. Stoyanka’s bronze tinted hair became undone and partly dropped to one shoulder; her brown cheeks flushed as the pace increased.

Sandy gripped my arm in sudden alarm.

“Look, asker are coming up here! Let’s make for the hills!”

“Keep quiet,” said one of the villagers, “do as we tell you, and nothing will happen.” He shoved us over to the dancers. A girl reached out and pulled Sandy in, while Stoyanka gripped my hand. Old dedo’s cheeks distended with a renewed effort to which pipes and dancers alike responded.

During the turnings I saw an officer and a dozen black-uniformed soldiers approaching along the path from the village. As they reached the circle of spectators their dark, sun-tanned faces spread in good-natured grins. The becktchee, now leading us quite a pace, cut a few gratuitous capers and shouted to them in Turkish. The officer clapped his hands and responded with something that might have been:

“Go it, you jolly beggars. I wouldn’t need much urging to join you.’’

As I caught the faces of the dancers to either side of me, it would have required a keener perception than mine to discover aught in them but the gayety of the moment. I felt it was more than fancy, though, that Stoyanka’s grip on my hand betrayed nervous tension.

“A rest!” shouted the becktchee, flinging himself loose and mopping his face with the kerchief. The line broke. Stoyanka threw her arm suddenly around my waist, and with the lingering motion of the dance swung me back in among the vines, where we subsided to a reclining position on the soil, her elbow resting oh my shoulder. Sandy’s partner had executed a similar move.

Such an amusing fellow the becktchee was! He kept the officer and his men and the peasants who grouped around them in a continuous roar of laughter. The jokes and rapid fire talk were lost on me, but the soldiers evidently thought these giaours the best of good fellows. Old dedo came in for his share of the banter, but responded with such readiness of tongue that the Turks fairly danced from glee.
The dance was called on again. The becktchee cut a grotesque pirouette with a yell that rolled up into the hills. But just then the clear, crisp call of a bugle swept over the vineyards. The officer and his men turned, paused a moment, shouted something back and began descending rapidly toward the village. Dedo played, not the horo, but one of the plaintive folk melodies, while we lay there watching the column of troops emerge from the village out on the broad highway and wind into the hills beyond. The last few companies were still in sight when Sandy and I started down the path. A vague desire to say something caused me to turn back, but Stoyanka and the becktchee had disappeared while the old dedo seemed too intent on his melody to notice us. All the merry dancers were once more the stolid, plodding peasants, the “gloomy, sullen Bulgars” that various writers have described to us.

%CHAPTER VIII.

\chapter{ON THE ROAD TO MARKET}

As Sandy and I came down into the village again we met the schoolmaster with a long, serious face. We asked him what the trouble was, and he announced to us the news of an event which will be recorded in Macedonian history as one of the most serious of many such tragedies.

“George Sougareff and his whole cheta were annihilated by asker last week,” He read us the details from a letter he had just received from Monastir City by courier.

One of my reasons for coming to Monastir had been to meet George Sougareff, chief revisor of the vilayet. The organization’s system of administrative subdivisions coincided almost exactly with those of the Turkish government. Each caza, or county was a rayon, governed by a rayon committee, represented in the field by the voyvoda and his armed escort, the cheta. A number of these cazas constituted the vilayet, or province, governed by a provincial committee composed of one delegate from each of the rayons. This superior committee was represented in the field by the revisor, who controlled the rayon voyvodas. Above this came only the famous Central Committee, composed of a delegate from each vilayet.

Of all the five vilayets Monastir was considered the most important, for there the aggressive tyranny of the Greek Church was most felt; there the most serious insurrections had occurred.

Besides being revisor of Monastir Vilayet, George Sougareff was also one of the ablest of the big leaders, and one of the first organizers with Damian Grueff. Above all, he was one of the few unaffected by the internal partisan broils introduced into the organization by Boris Sarafoff. His honesty was above question. I mention this fact, for it was an important factor in determining the events I have yet to narrate.

The calamity had been humiliatingly commonplace. It was the usual story; the cheta entered a village, was betrayed, surprised and surrounded by asker and engaged in a fight with an overwhelming force. Not one of the twenty-five escaped. The one peculiar feature was the mysterious disappearance of all traces of Sougareff himself. Only twenty-four bodies were recovered by the soldiers. Neither Sougareff’s remains nor his rifle were ever found, by either friend or foe. For months afterwards there were vague rumors that he had escaped and lay sorely wounded in some refuge. To-day there is little doubt to us who were in Monastir that year of what really became of the young chief. But let that develop as I go on.

The schoolmaster seemed unusually depressed, though it was true that he had known Sougareff intimately. He spoke of leaving Monastir for some other part of Macedonia, or even emigrating to America.

“One man’s death doesn’t condemn Macedonia,” suggested Sandy.

“But when one honest man after another is wiped off the stage and only the rogues remain, what hope is there?”

His vehemence startled me, but seeing no special foundation for it at the time, I finally put it down to his unsympathetic environment.

Later in the afternoon Tanne appeared; he and his cheta all mounted on swift ponies. They dashed and whirled about the downs and the bald hills like Tartar nomads, sometimes in full view of the soldiers, but too nimble to be successfully pursued.

Tanne was a remarkably handsome man; dark gray, scintillating eyes, clear cut features marred only by bad teeth; long, chestnut locks and a short golden beard. His dress included Albanian trousers and a jacket of European cut, a curious contrast to his lieutenant, a huge, hulking fellow dressed in an English Norfolk suit, wearing side whiskers, so that one might have thought him a British tourist out after big game.

Tanne’s greeting was a curious compound of the ignorant peasant, flattered to meet a European; the gracious lord of the district, and the polished gentleman of very high toned society. His attempt to converse in correct Bulgarian was as amusing as the speech of an Alabama country negro affecting white man’s diction. But withal, he had the reputation of being a desperate fighter. My acquaintance with him was short, for this was Saturday and Monday would be market day in the city.

Late that afternoon Sandy and I traveled two hours down into the plain in company with a peasant. It was dusk when we arrived in a small hamlet and were conducted to one of a group of low, mean-looking huts, some of them built over the ruins of larger houses. Ruins there were on all sides; it was as if a fire had swept the place. An unusual quiet prevailed; even the dogs were silent.

“Somebody must be dead,” I ventured.

“The lingering shadows of the uprising,” returned Sandy. “If the chetas could come here the people would regain self-confidence. Here they’re at the mercy of every passing vagabond with a gun.”

Inside our hut was illuminated by a burning wick in a saucer of oil, not strong enough to reveal the corners of the single room, but enabling us to observe low rafters, shiny black from the smoke of the fire usually burning in the middle of the earth floor, but now only aglow.

It was quite late when our host and his family came in from the fields: Itcho, the local president, a tall, gaunt, leathery-faced man, by no means so mean as his dwelling. His brother lived with him; their wives began feeding the fire and cooking, assisted by the widow of a brother who had lost his life in the insurrection. Then there was the mother of the two brothers, so old that she could only sit doddering by the fire. There was a grim cheerfulness about the two men that restored my confidence. Itcho’s brother had been in America, even in my native San Francisco. The two held an important post in the provincial organization; they controlled the postal courier system between the city and the southern rayons. The year before they had conducted Damian Grueff into the city disguised as a priest. Grueff had been commander-in-chief of the insurgent forces in the field.

It was late when we supped on boiled eggs and bread and cheese. The old woman by the fire revived.
"Two—two of them have come," she croaked, ominously. "Two boys with smooth brows."

"All right, mother," replied Itcho, good-naturedly, "they’re our boys."

"Beware, you smooth-faced boys. They have keen fangs. There are wolves about, they have long teeth. They drink young blood."

"She lost her head when the Turks burned us out," Itcho explained between two large bites at his bread. "Don’t worry, mother, the wolves are getting tame these days."

"I saw them," she continued muttering, "they were here many days—in packs. They tore women and children to pieces. They killed my son Itcho up in the hills."

And so she kept on until she nodded and they laid her down to sleep on a pile of rags. Itcho had been with the chetas during the insurrection, and on his return after the amnesty, his mother had never recognized him.

We rose at dawn, and after a cup of Turkish coffee, assisted Itcho and his brother to load four horses with market produce. A few alterations were made in our dress; about our kalpaks we wound woolen turbans, the ends of which were bound under our chins. A careful rehearsal of identities followed; Sandy was Mitso of Boreshnitza, son of Nicola; I was Ivancho of the same village, son of Stefan. Further than that no Turk would inquire. Then we mounted and rode out on the main highway.

My spirits rose as we came out among the open fields, for down here on the plain the trees and meadows were already green and the orchards were pink and white with blossoms. From every rise in the road we saw the white minarets of the city against a background of hazy green and blue mountain.

We began meeting people; first a group of Musselman peasants going to Lerin, greeting us with a good nature inspired of the splendid weather. More such followed at intervals, some mounted on donkeys. Further on came the railroad crossing where a knot of asker hailed us with:

"Good luck to your market bargainings, comrades!" To which Itcho and his brother responded with:

"Long lives to you, brothers!"

More peasants passed, some cheerful others stolidly silent, but at least nodding their heads. Several times veiled women mounted on donkeys passed, always attended by young boys on foot, and never in the company of men.

In the shade of a grove of trees, by a spring, rested a dozen fierce looking Albanians, bristling with revolver butts and knife handles about their sashes. Horse thieves were they, but still gentlemen, in this land of anarchy.

Again, we met an aged hodja, loose-robed and turbaned with white, accompanied by a small group of retainers, all mounted on asses. They, too, had a friendly greeting for us.

"May your gains be large!" called the old white-beard. Truly, it was hard to tell that this was a land of hatred and discord. All simple, kindly folk; even a dozen drunken asker about the inn of a village through which we passed, in great fear on my part, had nothing but good-natured jests for us.

We came abruptly over a rise and all Monastir lay before us; acres of wine-colored tiling, from which rose green trees and the white domes and slender pinnacles of the mosques. Out in the suburbs stretched lines of tall poplars and regular neat hedges dividing patches of truck gardens. On either side of us, along the road, were cafe gardens, crowded with men and women in semi-European costumes; military officers, slim young boys in cadet uniforms, and elderly Musselmans in ponderous turbans seated crosslegged on raised platforms. Unarmed soldiers strolled about, in chattering groups, some making us the subjects of remarks obviously humorous.

A column of marching girls in sedate grey uniform dress, passed, followed by two women and an Albanian gavass. There was something strikingly familiar in the dress of those two women; as I passed I caught three distinct words, “Is it far?” I recognized the good familiar twang I had not heard for some years. It was the American missionary school out walking. But I was not then so situated that I could give way to my keen desire to introduce myself to my two countrywomen. Nor did I ever have that pleasure.

Finally we came to the guardhouse, or tollhouse, but the official lounging there did not trouble himself about us, and we passed down into the crowded, narrow streets. We dismounted, and led our horses through the jostling throngs, slowly pressing our way down the main street, over a bridge, down a side alley and so came into the courtyard of a hahn. Here a hostler took our horses from us and we retired to a room on an upper floor.

From the lining of a saddle we took out our papers and the innkeeper went off himself to deliver my letter to the proper person. He returned within fifteen minutes.

"It's all right,” he whispered, “come with me.”

Sandy and I bade Itcho and his brother a regretful goodbye. Then we were out in the street again, following our guide some distance behind, the two of us clasping hands, as young peasants are wont to do in town.

We entered a church. Before the altar stood a dozen peasants in silent worship. We joined them, our heads bowed, occasionally crossing ourselves as the priests droned through their chants. An old priest came slowly down the aisle and as he passed me his elbow brushed mine. We turned and followed out into the courtyard, then swiftly slipped into a doorway and up a dark stairway. It was a relief to have the play acting over with for a while.

%CHAPTER IX. 

\chapter{ON THE INTRIGUES OF A PRINCE.}

Here the narrative pauses, inartistically, perhaps, but necessarily, to treat of certain dry, historical facts, without which this might be merely a tale of inconsequential adventure. Nor not so dry, either, to one who relishes a flavour of court intrigue and human passion and greed clothed in royal purple.

When the revolutionary organization in Macedonia had by its numerous members and branches become a potential factor in the internal affairs of European Turkey, its leaders decided that it should be represented abroad by a “committee of representatives." Their business it would be to present to Europe the true aims of revolutionary Macedonia, dispelling the popular belief that the chetas were brigand bands and the leaders only robbers out for loot. For this purpose they would publish a weekly organ.

In the principality of Bulgaria were 50,000 Macedonian immigrants, most of them political refugees, who had sought refuge across the frontier among their own blood kinsmen, for, racially, Macedonia and Bulgaria are one. Many of these immigrants had prospered; some were wealthy and some had risen to high posts in government office. It was, therefore, only natural that the committee of representatives should establish itself in Bulgaria, for another of its functions would be the collection of contributions for the revolutionary cause from the Macedonians who had prospered in Bulgaria.

The appointment to the head of this committee required careful selection. One man must devote all his time to the business; the rest could be volunteer workers engaged also with their own affairs. But that one man must be one who could treat with ministers of cabinet, newspaper editors and even pass a pleasant word with foreign consular officials.
He must be diplomatic, energetic and something of a journalist. He must also be acceptable to Bulgarian government officials, for Bulgaria was the principal channel through which munitions of war, bought in Austria, could be smuggled across the frontier into Macedonia. Thus the choice fell to a dashing young officer in the Bulgarian army, Boris Sarafoff, Macedonian born, but fairly well educated, of brilliant wit and of magnetic personality. Even Sarafoff’s enemies will not deny that at the time he seemed the best fitted candidate for the office.

On the throne of Bulgaria sat a prince, also an ex-army officer, ambitious, of dashing personality, and a diplomat. To him such a one as Sarafoff was a man of sympathetic nature. In so small a city as Sofia it was only natural they should come together and discover they had common interests. I have said Prince Ferdinand was ambitious. Like all princes, his eyes roved across his frontier. He saw there a people divided from his own subjects only by arbitrary political boundaries. These two adventurers came together, each seeking help from the other. But, unfortunately for Macedonia, the adventurer on the throne was the craftier of the two.

At once Sarafoff seemed to prove the Committee’s wisdom of choice. As a press agent he made known all over Europe, and even to America, the name of the “Macedonian Committee.” To this day editors are under the belief that Sarafoff was the chief of revolutionary Macedonia, for modesty was not one of his qualities. But it was in more substantial ways that he gave most satisfaction; as a collector of contributions from the Macedonian immigrants in Bulgaria.

The slim resources of the revolutionary treasury swelled and Manlicher rifles poured into Macedonia. The youthful leaders in the field were too enthusiastic to inquire deeply into how Sarafoff procured the means, nor were they experienced enough in statescraft to note the significance of the fact that on the stocks of the Manlicher rifles were branded the Bulgarian government’s coat-of-arms. If any realized that these were government rifles, too numerous to be stolen, they did not reflect that states do not make presents from sentimental motives. Men engaged in active battle are not likely to speculate over the moral nature of the means whereby they are supplied with the materials of war.

Sarafoff collected vast sums of money. Those who had formerly given one franc for love of country now gave two or more from other motives. Sarafoff and his agents were curt and determined; they lost no time in persuasive argument. Give, or die, and enough paid that penalty to impress the others. Of course, there were protests, but of what use is protest when the police is against you? It will be seen by this that the understanding between the prince and Sarafoff was bringing material results. It may be naturally inferred that what Prince Ferdinand expected out of this compact lay in the future. Meanwhile Sarafoff’s success gained him greater power; he became the sole representative of revolutionary Macedonia to the world beyond Turkish frontiers.

But the young revolutionist finally committed an indiscretion. He sent one of his strong armed deputies beyond the territory of his friend Prince Ferdinand. A Rumanian editor in Bucharest sought the popular vein by publishing disagreeable facts regarding Sarafoff’s methods. His paper found its way into Macedonia and was read by the leaders. Sarafoff had him assassinated.

The Rumanian police held different views on such matters from the Bulgarian police. They wanted the murderer punished, and to enforce this opinion, a Rumanian army gathered along the banks of the Danube. Then Sarafoff was arrested in Sofia and indicted for murder. Later, when popular sentiment in Rumania had cooled, he was tried and acquitted, but his royal partner thought it best for the firm’s interests that Sarafoff make a journey abroad. So, for a time, Sarafoff haunted the cafes and boulevards of Paris.

Meanwhile his successor was appointed, but not by the Macedonian Committee. A staid old Bulgarian general by the name of Tsoncheff, another sorry name in Macedonian history, was inserted in Sarafoff’s place, but he is not directly concerned with this story.

After Sarafoff’s downfall followed an inquiry into his methods by the Macedonian leaders in the field. He was severely condemned. They attempted to reorganize the Sofia committee, but the Prince’s man, General Tsoncheff, refused to surrender his place. He had the machinery of the collection agency in his hands, as well as the official organ of the revolutionists, and their attempts to oust him were futile. For through the official organ, and through all the Bulgarian papers, he made the Macedonians in Bulgaria believe that he represented revolutionary Macedonia. Moreover, he had the police behind him.

Unfortunately, just at this time occurred the famous betrayal in Salonica, referred to in a previous chapter. The important leaders were imprisoned in Asia Minor, and, for a while, Tsoncheff had the field of action almost to himself. At this moment occurred an important incident, to be treated of in the last chapter in this book.

Later European diplomats made a moral demonstration by forcing the Sultan to grant a general amnesty. The leaders were liberated, and finding a big sum of money on hand (how obtained, stated in the last chapter), they declared an uprising in Monastir, hoping thereby to precipitate European intervention.

At that opportune moment Sarafoff reappeared and humbly offered his services. He fought brilliantly under Grueff’s command, and for this his past sins were forgiven.

It was then that Sarafoff made use of those same qualities which had procured him his appointment in Sofia. Here his magnetic personality gained him the blind admiration and the loyal support of those youthful chiefs whose minds were of that type which follows only personal leadership, not yet broad enough to grasp an abstract idea and make that their guide to action.

The insurrection had of course disorganized the whole revolutionary organization. All the rayon forces had concentrated in Monastir, and of the voyvodas many were killed or crippled.

After the insurrection followed the re-organization; new voyvodas were appointed to the various rayons. In this work Sarafoff's undoubted ability as a military organizer gave him much influence with the other chiefs. When the committee realized that half the voyvodas in the field were of those whom Sarafoff had hypnotized, it was too late to mend matters. At least, it was the determined effort to mend the mistake that cleaved the organization into two opposing factions.

Here definition should be accurate. When an organization begins to number two million members, including whole cities and provinces en masse, it ceases to be a club, or a committee. By now it must be evident that the '‘Macedonian-Adrianopolitan Interior Revolutionary Organization” had outgrown its name, that it had become, in fact, a provisional system of government established by the Macedonian peasantry to replace Turkish anarchy. Though imperfect in details by the very force of the obstacles opposing it, it was still a well articulated republic in form, swelling to burst through the artificial surface of an obsolete system. What was the Central Committee but the essence of a popular assembly, representing a federation of districts, or states ? How could the chetas be better defined than as the constabulary of this underground republic.
In the necessarily secret nature of this national organization lay its dangers, and it will be observed that the powers of administration were vested in groups, rather than in individuals, and that the control of the chetas, the armed force of the system, was as diffused as was consistent with efficiency. So great was the fear of one man control that the tendency was ever towards decentralization and a weakening of the power of the Central Committee.

Here was a chance for an adventurer of Sarafoff’s type. The greater the discentralization, the weaker was the power opposing him; he could not be summarily disposed of. Wherever government is diffused there is the demagogue’s chance to gain power through an appeal to popular sentiments and emotions. To hold the loyal friendship of the voyvodas was to control them and their chetas, and to control the chetas was to gather the power of the entire underground republic into his own hands. Against him he had the brainiest and the most clear-sighted individuals, but behind him were the ignorant masses, and all those whom money could buy, for he was as adept in bribery as a Tammany politician. This campaign of corruption Prince Ferdinand financed. I know of none of my friends in the organization, of any influence, who have not been approached by Sarafoff or his agents at some time or other.

Seen at that time, his chances of success were tempting. With the power of the organization in his hands in Macedonia, he could count on the outside support of Prince Ferdinand, commanding one of the best organized armies in Europe. So he might hold his position as dictator until the time when, in the natural process of evolution, aided by human effort, the husk of Turkish rule would fall off, leaving him in care of Macedonia's destiny. For then no one suspected the virility of the Young Turk movement.

After all, this much is due Sarafoff. His means were unscrupulous, but his end may not have been entirely selfish. Quite possibly his visions included a re-established Bulgar Empire, ruled over by a German prince, hateful to all Bulgars, but still a Bulgar Czar. No one doubts that he saw himself looming up definitely behind the imperial throne. To realize the bitterness of the opposition against him, it must be understood that the Bulgarian temperament is by nature democratic, to which imperialism is hateful; a temperament which takes more naturally to socialism. Most of Sarafoffs opponents were indeed socialists, and recognized in him only the creature of Prince Ferdinand. But their power was on the increase, so much so, that in the last congress they had brought Sarafoff to his knees. He came with tearful protestations of his good faith to the principle of Macedonia for Macedonians, and so, by a slim majority, retained his official position and his influence as one of the two chief revisors for another year. The other revisor, one in whom the opposition trusted, was paired off with him as a sort of check. It was Sarafoffs last chance: eventually he broke faith again, but he paid the penalty, as did his partner, whom he corrupted. But that lies in the future. Revisor he was at the time this narrative is resumed.

Chapter X. Monastir City.

%CHAPTER X. 

\chapter{MONASTIR CITY.}

Sandy and I found ourselves in a small cell-like room furnished with only a cot, two stools, a table under a grated window and a shelf meagerly supplied with some shabby books of religious titles. The priest who had led us in stood beside the table; an old, venerable man, his white locks straggling over his shoulders, his equally white beard sweeping over the breast of his black, shiny cassock. A humorous smile played about his huge mouth.

“What a costume for an American citizen!” he remarked as he reached out his hand in greeting.

“And you, father, are-” I could have guessed, but he answered:

“A prominent personage in history—Mirabeau.”

I knew him then as the president of the provincial committee, one of GruefFs first recruits. He was one of the few of the veteran organizers not yet discovered and outlawed. As we talked another person entered the room, a slim young dandy in well-fitting European costume, waxed moustache, an ivory headed cane and a high crowned fez set jauntily over one ear.

“Hello, my good fellows,” he greeted us carelessly, then looked searchingly about the room. The priest chuckled.

“Damn it,” cried the gentleman, suddenly, in clear, crisp English, “you’re not the American, are you?” And he grasped my hand cordially.

The Lerinsko schoolmaster had been loquacious, so I recognized the “Eagle,” delegate from Lerinsko on the provincial committee and its secretary.

The “Eagle” was off presently to prepare our reception in safe quarters, whistling a French melody as he passed down the stairs. After half an hour’s general conversation with the priest we two were out in the street again, wandering leisurely through the Sunday crowds, but following the lead of a young woman in a peculiar green skirt. Passing over the stream rushing down the main street, through the business quarter of the city, we came finally to the Yeni Mahli, a quarter in the suburbs built up of houses rather pretentious and modern for that country. At a favorable moment we slipped into a gateway, through a pretty garden, thence into a large house.

We passed up a stairway of polished wood and were presently reclining among pillows in the corner of a well, almost luxuriously, furnished apartment, surrounded by a group of young girls, a stripling of sixteen and a matronly woman in the background. An adorable family group, wanting only the father who came in presently, an elderly, corpulent Bulgar in European costume, smiling benignantly. His presence encouraged the girls, and then began such a treatment of hero worship as was at once embarrassing and still deliciously endurable. It was the practical mother who suggested a bath, so downstairs we went and washed in a great tub of water; washing off, it seemed to me, not only the thick coating of Lerinsko dust, but the glamour of comitluk as well, for you cannot feel picturesque in clean clothes of plain, conventional cut, such as were given us to put on.

When we returned upstairs a new contingent of girls had arrived from next door; dark blond, hazel-eyed, Slavic girls, prettily, but not picturesquely, dressed, so that it seemed as if we participated in a most commonplace family gathering. It was like coming home from the wars to be lionized, whereas, in fact, hardly to be realized in such environment, we were as much in the field as we ever had been.

Then came dinner, still further suggestive of that atmosphere of the home life so idealized by the western races, such a genial, prosaic Sunday dinner, that sitting cross-legged about a low table suddenly struck me as an incongruous joke. But the damask cover, the napkins, the knives and forks and the separate cover for each of us were all pleasantly conventional.

During the afternoon we two sat on cushions in a corner receiving youthful visitors, boys and girls, the juvenile members of the family seated beside us, like priests of the temple interpreting the words of the gods and regulating the worship. Under this genial influence Sandy thawed, even to relating a story of fighting, something I had never known him to do before. For the first time since leaving Voden I felt free of the nerve tension habitual to all who wander in Macedonia along illegal paths.
Evening was drawing on and the hilarity on the increase, when we heard a muffled knocking below, followed by the unbolting of the gates. A pause, then the room door opened and the Eagle entered, smiling and jaunty. 

“Hello, old chap!” he cried, flinging his fez aside, “you seem to be right in it.” 

I felt, rather than saw, that the crowd in the room froze stiff. When I looked, some had withdrawn, others were edging out through the doorway. The girls beside us drooped their eyes from apparent shyness, and presently they rose and left the room. The host came to offer the Eagle a glass of wine, but his face was peasant-like in its stolidity. 

My irritation at this shower bath wore off soon; after all, it seemed natural that one of the provincial committee should inspire awe, especially among young people. Then, the Eagle was brimful of news, for he had just returned from spending the Easter holidays in Bulgaria. So I forgot all else in the animated conversation that we entered into then, though later I observed casually that Sandy had fallen asleep. It was a little egotistical of me, but months had passed since I had heard English spoken; years since I had enjoyed that New England twang which the faculty of Robert College in Constantinople seem to impart even to their students. 

No one could deny that the Eagle was a fascinating fellow; especially among women, I imagined. It was natural that I should find him entertaining, and forgive him those mannerisms that at first impressed me so unfavorably, for, quite aside from his English, he had been years abroad, even in England, though he had finished his studies in France. He was born and bred in Bulgaria, but had been appointed teacher of mathematics in the Monastir gymnasium. Being a Bulgarian subject and on intimate terms with the Bulgarian diplomatic agent and other foreign representatives, his position was exceptionally secure, for the police must have absolute proof of his guilt to disturb him. But on that first evening we spoke little of business, for I had almost three months’ ignorance of current events to be satisfied. 

Supper was not so dismal as I feared it would be, for though the girls waited on us instead of eating with us, yet the Eagle drew out their smiles and infused some cheerfulness into Tashko, our host. Of course, the conversation was entirely in Bulgarian then, and Sandy and the Eagle discovered they had acquaintances in common in Bulgaria. 

Sandy and I began feeling the effects of this unusual day early, so we soon retired. On account of the rigorous curfew laws the Eagle shared a front room with us that night. It mattered not that the mattresses were on the polished floor; it was delicious to shove our glowing bodies in between cool, white sheets and allow our consciousness to fade and float off into a dreamless sleep. 

%CHAPTER XI. 

\chapter{MEETING OFFICIALS OF THE UNDERGROUND REPUBLIC.}

The process of awakening to that first morning in Monastir was a series of bewildering illusions; first, I was back in the swamp, then in Bulgaria, and finally in jail. The last sensation woke me with a start. Sandy sat before me, himself just awakening, his face fat and heavy with sleep. 

“He’s gone,” I remarked, observing that the Eagle’s bed was empty. 

“Yes,” replied Sandy. “Eagles fly early.” Then he added: “I’ve heard of an ass personating a lion, but never an eagle.” 

“That’s treason,” I protested, “besides, he is jolly company.” 

“I will do him more justice,” said Sandy. “I meant to say that I’d never heard of a hawk in eagle’s feathers.” 

“You don’t seem to like him.” 

“I don’t” 

Just then came a knock at the door, and the mother called us out to coffee. We found a bundle of clothes to replace those we had put on temporarily the day before; two well cut suits and a pair of shoes for each of us; all new. It was strange to see Sandy dressed so; with only the addition of a derby hat he might have passed down Broadway and attracted no notice, except as a fine, well built lad. He, too, seemed struck by my sudden respectability. Then we each put on a brilliant crimson fez and strutted about before a mirror, studying angles, and arranging our ties. 

Pinned to one of the bundles was a note from the Eagle bidding me not to be tempted by the streets yet, and adding that he would dine with us again in the evening. To Sandy the day must have been dull, for the girls were away to school and appeared only late in the afternoon, though even then they seemed shy of us. Tashko joined us at lunch with all his original good humor. He was a master builder and busy at that time of the year, so he left us again early in the afternoon. I had plenty to occupy my time, however, for with the clothes the Eagle had forwarded me my correspondence from Bulgaria, which I looked over and began answering; slow work, for much of it was in cipher. Even the Austrian post was not secure. 

Late in the afternoon the Eagle came with a bundle of foreign newspapers. They contained lurid descriptions of the destruction of my native city, San Francisco, and rather upset me for the evening. 

“We’ll dine in another house,” said the Eagle; “the treasurer of the committee will meet us there.” 

So we left our pleasant friends, Tashko and his family, passing out into his garden, through a doorway in the back wall, into a second garden and so into another house on another street. A very old, but still sprightly, woman greeted us as hostess, but she remained decidedly in the background that evening. 

“I am leaving for Salonica by the morning train,” said the Eagle, as we settled down among the pillows. “As I may be away a week or more, we’ve decided to get you into safer quarters. This part of the city is always under police suspicion. You’ll be over toward the hills, in the Bel Shishma quarter. It’s the same house Grueff made his headquarters.” 

Not knowing one end of the city from the other, I could make no criticism of this arrangement, though Yeni Mahli seemed an agreeable location. I cheered up under the Eagle’s conversational spell, but Sandy remained thoughtful and brooding, as though his town had been quaked instead of mine. 

I felt that the Eagle sounded me on my partisan tendencies, but I had the sense to pretend neutrality, though I could not hide the fact that my general pass was signed by the committee of representatives in Sofia, now all strong anti-Sarafoffists, for by this time General Tsoncheff had been eliminated from Macedonian affairs. We were still fencing on this point when the treasurer arrived. 

He was a young, boorish sort of fellow with a face that would have passed unnoticed in a group of half educated peasants; he might have been an inferior village teacher. He, too, had a pseudonym, which I have forgotten, for there was afterwards enough reason to remember him by his real name: Peter Legusheff. We had supper at once, but the conversation continued between the Eagle and me, for Legusheff seemed of a moody temperament. 

During the meal I noticed once a youth in gymnasia students' uniform pass through the room and pause in a doorway leading into an inner apartment. I observed that he eyed me keenly several times; then passed out by the front again.

“Let me caution you,” said Legusheff, beginning the longest discourse he held that evening, “to keep your identity secret. Tell no one you are the American. Because, if the police know you’re in town they’ll begin such a search that all our houses will be turned inside out. Have nothing to do with the students; they’re good fellows, but they’re gossips, especially the girls.”

This seemed reasonable advice; it was their business to caution us against all the dangers of that locality. Legusheff left us soon after supper. “I wouldn’t go if I were you,” remarked the Eagle, as Legusheff lit his lantern. “I don’t want to sleep in this house,” he muttered under his breath; “besides, it’s only a dozen steps.”

It was the Eagle who needed to retire early this night. When we awoke in the morning he had again flown. The old woman served us coffee. She was quite talkative now, but her dialect was so broad and her teeth so bad that even Sandy hardly understood her, except that she referred frequently to somebody named Michael.

At noon the young student appeared and we all ate together. He proved to be the old woman's son. In a shy, but still persistent manner, he tried to question me. “You speak Bulgarian with a foreign accent," he remarked. “Yes, I’ve been long abroad." He knew I lied.

When he was gone, Sandy began, “What made you snub the boy?" “Didn’t you hear what Legusheff said last night?" He reflected a moment, then replied: “It’s curious. Yesterday all those girls might know you. To-day students, usually the most active members of a local organization, mustn't know."

“I notice they've remedied the mistake," I observed. “It isn't twenty steps to Tashko’s house, and still the girls haven't been near us." “And the garden gate’s barred on this side," rejoined Sandy. “This is the only garden isolated by a board screen along the top of the wall. A Turkish family must have lived here once."

I had been reading the foreign papers. When I finished I called the baba and asked her to burn them, for all were more or less illegal. “I never burn printed books or papers,” she said, “because Michael always said, ‘mother, our people will only be free when they know what’s in the books,’ so I never burn them; such things are holy.” She went to a cupboard and exposed a shelf loaded with paper bound books and pamphlets. I took a few down and examined them; “Force and Matter,” by Buchner, and “Carl Marx,” interpreted by an obscure Bulgarian writer. There were pamphlets on or by Kautsky, Vandervelde, Mill, Bernstein, and a romance by Andraeff in the original Russian. I opened a fly leaf and read: Michael H—ff.

“Are you Michael H—ff’s mother?” I asked. “Yes, I am,” she said proudly. Sandy approached, interested. The name was a famous one in Macedonia; that of one of the brainiest of the leaders, one of the avowed socialists arrayed against Sarafoff.

“But if all these books are discovered during some search?” I protested. “What can they do to an old woman?” replied the mother. “And what Turk would understand those titles?” put in Sandy. “Unless he’s a Young Turk, and then he’s a friend.”

I was anxious now to meet baba’s younger son again, if for no other reason than to apologize for my rudeness. But before he appeared came a man in a kimonalike costume who was to escort us to our new quarters.

We followed him by devious ways to that quarter of the city up against Mt. Pellister, out near the market gardens. Here we entered a small house, unlovely as a village hut, and found ourselves again among peasants; the man was a horse dealer, the woman illiterate, the children dirty and the rooms bare of such furniture as makes a house habitable.

“What did Grueff do here for six months?” I growled as Sandy and I stretched out on a straw mat for the night. “Market gardening,” suggested Sandy.

%CHAPTER XII. 

\chapter{A LONESOME WALK.}

From the verandas of the houses high up in the Bel Shishma quarter, Monastir city in perspective presents a decidedly picturesque aspect, for you see only the roofs and the minarets. Tourist writers love to dwell on these features, though a few have descended into the narrow, dirty, dog-pestered streets and mixed with the motley crowd of Jews, Turks, Greeks and Bulgar peasants. Considering its population of sixty or seventy thousands, the city spreads over a small area; small, compared with a more recent city like Sofia. A modern board of health would be as destructive here as an invading army.

But to a prisoner even a picturesque view grows tiresome, and that we were prisoners Sandy and I discovered on our second day in the Bel Shishma Mahli. Poor Sandy suffered most, for I had enough work to keep me busy the first week.

First, we were cautioned against presenting ourselves near the open windows. Then the man in the kimona brought the news that Yeni Mahli had been searched by the police. Next came the report of Father Mirabeau’s arrest.

Our host, Itso, came home late and was gone early, but he was a sociable companion while he was with us. I asked him about Grueff, but he had only lived in the house since last winter, so he knew nothing of its antecedents. During the day we were alone with his wife, who was quite as sociable as the door jamb and no more communicative.

We lived well, though. Each day the messenger brought us bottled beer and sardines and other delicacies foreign to a peasant diet, sent by Legusheff, as a short note enclosed in one of the packages indicated, reiterating the warning to keep close to our hiding. Had anybody except the messenger come to visit us, our situation would have been more endurable, but even after we heard that the Eagle had returned from Salonica, there was no break in the monotonous routine of our isolation.

About a week of this confinement had passed when Itso came home one evening in unusually good humor. Sandy and I had quarreled over some trifling matter, and Itso finally noticed our sullenness, and asked the cause thereof. Sandy voiced our complaints in a few, terse words.

“You need a walk in the open air,” replied Itso. ‘I'll take you to the public baths tomorrow morning.”

It was agreed that only one of us should go at one time. So I was up early next morning; Itso opened the gate, looked out to see that the street was clear, and we swung down the sidewalk side by side, toward the thick of the city. We came finally to a small square on which faced a large, brick building, domed on top and exuding vapors through numerous loop holes.

The attendants inside were Turkish. On one side I observed a dressing-room full of iron bedsteads; against the walls hung black, soldiers’ uniforms. But we turned to another side and entered a room with only four beds.

“Here’s where the military come,” explained Itso, when we were alone and undressing. “It’s safer; nobody would look for you here, and few Christians come to these baths. Turks don’t ask you personal questions.”

Out in the tank room it would have been hard to study individuals, for everybody was naked, save for a cloth about the loins, and the hot, steamy atmosphere was too enervating to inspire curiosity in one’s fellows. The masseurs were unusually taciturn.

We returned to the dressing-room, swathed in huge gowns and towels, and, reclining on the beds, drank coffee. On a bed opposite me lay an old, white-haired man, also swathed in towels.
"Very hot," he observed to me, "they ought to open some more of the vents." His correct speech led me to believe him a Bulgar, so I replied: "But it makes you sweat well." He replied again, and a languid conversation seemed imminent. But Itso inserted a remark and carried the stranger’s interest away from me. Soon he rose and dressed, and my nerves almost failed me when I saw him quietly slipping on the uniform of a police official. Finally he put on his fez and departed, touching his forehead to us. Itso must have observed my horror, for he laughed.

"Who was he?" I whispered.

"The sub-chief of police."

When I came home and Itso was gone I related my experience to Sandy, adding a remark I had heard the police chief make to Itso; that he was going to his country place that day. We agreed at once that there could be no very hot hunt on if the higher police officials went off on holidays.

Shortly afterwards Sandy reached for his fez behind a picture frame.

"I’ll go down into the town and try to find out something," he observed.

"I’d rather you wouldn’t," I returned. He paused, then demanded:

"Why not?"

"There may be danger after all," I said. "Arrest wouldn’t mean so much to me as it would to you, so I prefer to go myself. Then, I want to know more about the Eagle and Legusheff before we separate, even if it is only for a few hours. They might tell me you had been arrested; I wouldn’t know whether you had been or not."

Sandy regarded me thoughtfully.

"You think it might be so bad as that?" he remarked slowly.

Then followed a long pause. Suddenly I put on my fez, and before Sandy could protest, as I saw he meant to do, I went out into the kitchen.

"The gate," I said to the woman. "Have a look." She understood, and went out into the yard. Opening the gate, she looked out, up and down the street, then drew in and nodded her head. I passed out.

There were women and children about, but all evidently Bulgars. I followed the way I had gone to the baths with Itso, walking briskly, as though on business.

Of every turning and corner I made a mental note, besides carefully sensing the general direction by such tall buildings and minarets as rose above the area of roofs. So I came finally to a large market place, crowded with peasants’ carts and horses and vendors, the sides of the square lined with low, open shops kept mostly by Jews. Here I wandered about half an hour or more interested in the life and movement.

I passed on again and came ultimately to the main street, which was familiar to me. My object was to find the church in which we had met Father Mirabeau. But at first I was unsuccessful.

I went into a peasants' restaurant and drank some hot sheeps' milk. A boy served me, and being a giaour, began a series of question touching on my personal history. But he finally directed me to the church.

I was approaching the distant blue cupolas of the church building, along a narrow street, when the gate to a private garden opened and a girl came out, almost knocking into me. Both of us paused. She greeted me smilingly, though obviously puzzled, for it was Tashko’s eldest daughter.

"Don’t stand out here," she said, and pulled me by the sleeve in through the gateway, bringing the gate to after us. "I thought you were back in the hills."

"No," I replied, "but we are keeping very quiet. I suppose you know the obisque in Yeni Mahli was for us. Did the gendarmes make you much trouble?"

Her eyes opened.

"Obisque?" she repeated, "what obisque?"

"Didn’t the police turn Yeni Mahli upside down a day or two after we left you?"

"No."

I pretended indifference, and only said, carelessly:

"Oh, well, perhaps the old woman who told us had a dream. Have there been any arrests lately?"

"Yes, a priest. But there was no proof against him. He has only been sent to a village. But doesn’t the professor tell you these things?"

"I haven’t seen him lately. You don’t seem to like him."

My abruptness startled her, but she recovered quickly.

"No," she said, decidedly, "we don’t."

Destiny itself seems to hang on just such insignificant utterances. Even then I felt that a mere shade of encouragement would have induced her to confide in me. Following events might have taken a different course, then. But I had already acquired the instinctive caution which went with the rest of the year's course in revolutionary deportment that Macedonia taught me. So the opportunity passed. We shook hands and parted.

I hurried home at once, beginning to realize how angry I was with the Eagle and those who had thought it necessary to stuff us with such paltry lies. The messenger was there, and we kept him till I had written a letter to the Eagle, all the more caustic for being in English.

Early the next morning came an answer, in the Eagle's characteristic style:

"My most dear fellow:

"Of a certainty you do me the great injustice. Can you imagine the disgrace to us should evil occurrence betide you under our care? It is that I at once observe in you the fearless temperament which requires exaggeration that you should heed the danger. Let you at once return to Yeni Mahli, but observe that I protest against the risk. I beg that you comport yourself there with the secretiveness necessary to your safe sojourn in this our city, so honeycombed with treachery. Have a care, old fellow, of the youths who talk excessively."

That evening we returned to Yeni Mahli, too pleased to remember vividly our previous resentment.

%CHAPTER XIII. 

\chapter{BACK IN YENI MAHLI.}

The week that followed our return to Yeni Mahli restored us to a more complacent state of mind. Our new quarters were in a pleasant well furnished room whose two open windows overlooked the thick foliage of a house orchard. Our host was no less a personage than the president of the municipal committee, an important official in a city like Monastir. Quite apart from this distinction he was of a genial disposition, stout and rosy-faced, of Pickwickian aspect, with the rather misleading pseudonym of Moses. By profession he was a prosperous shopkeeper, by creed a Presbyterian, a convert to the American missionaries. But in conversation I found him, like all Bulgars aspiring to that state of intellectuality manifested by a coat of European cut, an unbelieving sinner. The shadows of great battleships, in which our missionaries in Turkey are enveloped, are often a convenient refuge for those who tamper with the Sultan’s laws.

There was another member of the local organization who came often to dine with us; Georgie, a lithe, lean, fair-haired man with one blue eye. I admired him, even when I afterwards learned his shady record, for his fierce, barbaric virility, blended with an impetuous good humour almost as genial as that of Moses. It was a generally observed discretion that nobody should recognize an illegal when he walked the streets, but during my walks, when Georgie passed me at his bustling, swinging gait, his kimona swishing about his lean shanks, he never failed to wink that solitary eye of his. His temperament fitted him to his office; he was chief armourer, a politer name for chief terrorist, for he it was who directly superintended the campaign against the armed bullies of the bishop, a fierce feud that claimed a victim at least once a week on either side while I was in Monastir. Incidentally, it was Georgie who disposed of any chance spies that might be discovered in the organization itself.
Both Sandy and I took frequent walks out into the city and even into the adjacent fields beyond the suburbs. My usual stroll was to buy cigars, to be had in only one shop in the city, for the burghers smoked only cigarettes. This shop was kept by an aged Turk, who spoke Bulgarian imperfectly, so I ran no danger of detection from him. We even established a jovial sort of intimacy based on pleasant inquires concerning each other’s health and prosperity.

One day Sandy had an experience, the humour of which rather overshadowed its danger. He had seated himself in a cafe and ordered a Turkish coffee. Two police officials came in and took possession of an adjacent table. He felt presently that they scrutinized him with unusual attention. With a slight movement he disposed himself so that he could make a free break for the door. Then he caught the words of their conversation, for being a native of Varna, he knew Turkish fairly well.

“I haven’t seen that young fellow before,” observed one of them, “I think I may say that my memory for faces is good.”

“Quite true,” replied the other, “he is a recent arrival. I’ve had my eyes on him for some days. He’s a village schoolmaster in for medical treatment; you see how pale he is.” Thus, by their desire to impress each other with their perspicacity, Sandy escaped an investigation that either one of them alone might have followed up.

The Eagle came to visit us several times, but only for an hour or two during the afternoons. I was too contented with my present situation, and he too clever, to allow any survival of my former ill-feeling. The gymnasium had really been searched for the committee’s archives, he told me, and this I subsequently learned to be true. In a burst of confidence he went on to tell me of his troubles with the socialist faction in the local organization, who desired to influence the coming election in favor of their own policies—hopelessly idealistic and impracticable. Dreamers, he called them. This was the first inkling I had of the local strife.

Work I had in plenty, too, writing numerous articles on the situation in general, and preparing a circular statement of the troubles with the Greeks, which the Central Committee wanted circulated among foreign newspapers and societies. It was as a sort of press agent that the Central Committee had issued me my commission, which entitled me to the rights of inspection of a revisor, except that I had the unusual privilege of publishing such information as I gathered which would betray no individual of the organization. They gave me a free hand, which was the safest policy, for the truth could in no way harm the organization as a whole. And it may be well to add here that my salary was equal to that of every other member of the organization, which was exactly nothing.

The Eagle undertook to assist me, arranging meetings with foreign gendarmerie officers and with the British consul. The dialogue he had with the latter was rather amusing. After the Eagle had spent about fifteen minutes explaining who I was and my desire to meet the consul with the object of exchanging information which might be mutually valuable, that gentleman finally asked:

“But how did he come here? By train? On passport? Surely, I shall be glad to meet any American traveling for information. Has he called on the kaimakam?”

“You misunderstand,” persisted the Eagle. “He has been in the mountains with the comitajis. He joined them in Vodensko; you may perhaps have heard of his disappearance.”

“Oh, I say, you don’t mean the man who was killed by the brigands? He’s joined them, eh? But that alters the situation; he’s outlawed by the government.”

“Let me put it in this way,” tried the Eagle. “He’s a newspaper correspondent on our side of the lines.”

“On your side of the lines? Really, that’s hardly to be taken seriously. There aren’t any two sides to the line; the British Government haven’t recognized any armed force hostile to the Turkish Empire.”

“But he doesn’t want to meet you officially; merely as one private gentleman with another. He doesn’t come as a revolutionary representative.”

“Oh, but that doesn’t matter. I am His Majesty’s diplomatic agent here, and this gentleman is an outlaw.” The Eagle now became alarmed that this punctilious diplomat would compromise him, but he finally extracted a promise that the interview should remain confidential. This muddle-headed gentleman was an exception; with several other foreign representatives I had no such difficulties. One even wrote me a signed letter.

One afternoon as I sat in the room writing I heard voices in the garden. Looking out of the window I saw Sandy conversing with a youth in European dress. Later when Sandy came in I asked who the stranger was.

“A chetnik; he’s in the next house. He’s the only survivor of Sougareff’s cheta. Just before the disaster he was left in a village on account of sickness. Now he’s in town to do a killing. As he’s a stranger here, and the Greeks don’t know him, they want him to try the Greek bishop. Perhaps you’d better not say anything about it to the Eagle.”

All this seemed so plausible that I thought very little more about it at the time. The Church terrorists had indeed been quite active; twice within the last week they had shot a member of the local organization in the open market place. The committee had not yet retaliated, but I knew from what Georgie had let drop that they planned a killing higher up, one that would be impressive.

Some days afterwards Sandy and I changed quarters to a house across the street. Here we were even more comfortably situated, for there was a second floor in the building where we could walk about without fear of approaching the windows. The family was large; the father was a tailor and one of the sons was a schoolmaster.

On Sunday we had moved. It was Tuesday evening, I believe, just at dusk, when the family had gathered together and we were about to sit down to dinner. Moses and Georgie had dropped in for a chat, and the gates were barred for the night as they went. And then, two revolver shots were fired outside in the street, followed by a loud cry.

All of us rushed upstairs and made for the balcony outside the front windows. Below, in the street somebody shouted three times, “Devree! Devree! Devree!” and presently the night patrol came running in response, one of the gendarmes with a lantern.

“The Greeks have got Georgie, or Moses!” exclaimed the schoolmaster beside us.

We saw the gendarmes lift a prostrate figure further up the street and carry it around a corner. A woman came racing back.

“What’s happened?” cried the schoolmaster. The woman called back some inarticulate answer, but we could distinguish only the one word “doctor.” We stepped back from the balcony into the room where the women of the household were collected.

“Do Greek terrorists carry Nagant revolvers?” asked Sandy.

“No, I don’t think they do,” replied one of the men. “Was it a Nagant?”

Again came running footsteps outside; the schoolmaster leaned out and called:

“Who was shot?”

“Peter Legusheff,” came the answer, audible to all of us.

It was next morning before we got more detailed accounts of the shooting. Legusheff had been passing down our street on his way home when a stranger emerged from an alley and fired at him. One of the bullets had pierced his neck, and he was now in a critical condition, though not beyond hope of recovery.
After this outrage, we expected a quick reprisal on the Greeks; Georgie said as much, in fact. But there were no immediate results. Sandy tried once to find the chetnik, Stoyan, but he had evidently moved to other quarters. 

In spite of our comparative freedom, we really met few people removed from our own immediate environment. I had hoped to see Michael H—ff’s brother again, and once I did meet him on the street, but he passed me with no sign of recognition. Still, that was a usual precaution. Tashko’s girls I saw also, several times, but they were equally discreet. One day, toward the end of our stay in Monastir, we had revealed to us one of the causes of our isolation from the young people with whom we would have enjoyed freer intercourse. 

Late in the afternoons, after my work for the day was finished, I used to lie out on the balcony and watch the passersby in the street below, without danger of being seen myself. My constant companion was Spiro, a ten year old boy in the household. He would point out individuals to me and tell me what he knew of them, which was considerable. 

Several times there passed two girls who attracted my attention because of their well dressed appearance; they might, indeed, have harmonized with a more cultured environment than Monastir. One, about eighteen, was exceedingly pretty, tall and slender, rather noticeable in a country where everybody is of peasant origin. Once she suddenly looked up, our eyes met, and her face flushed as though she were caught in some culpable act. 

“Who's she?” I asked Spiro. 

“Who? The girl who does her hair up like the Europeans?”

“Yes; with the grey dress.”

“That's Helen. She's T—ff's sister.” He named a noted leader in the insurrection, now exiled in Bulgaria. Sandy was in the room and craned his neck to see the object of my inquiry. The girl was a student in the gymnasium and boarded with friends further up the street. 

It was some days after this incident that I was seated on our rear balcony overlooking the garden. Below was one of the messengers who came to deliver letters to me, one of Georgie’s men. He had remained a few minutes to drink a glass of slivi and exchange a bit of gossip with the women. Sandy was in the back of the garden reading a book under a tree. 

There was a knock at the gate and one of the women opened. Then I heard a woman's voice asking for me. The courier had stepped behind an angle of the building. He came forward and demanded:

“What do you want to see him for?”

“Is it your business?” came the voice, angrily, from the gateway under me. 

“My business? Yes,” replied the courier. “I am one who guards his secrecy. Get your permission—you know where.”

I started hastily for the stairway and descended. As I came down into the lower room, I saw Helen T—ff through an open window, standing in the gateway, her face flushed with anger. 

“You'll get new masters soon, you lickspittle,” she flung at him, and passed out. 

I made for the doorway, meaning to have her called back. As I came out on the stoop Sandy sprang by and had the courier by the dress about his throat. 

“You swine!” he cried, shaking him. “What right have you to turn our visitors away?”

I came down with a bound and threw myself in between them, hurling them apart. The courier reeled up against the house, his hand clutching down in the folds of his sash. But he did not draw the knife. 

“This is cowardly, comrade,” he cried, “I do my duty, as I am told, and you strike me, as though I were a Greek. I can’t strike back, it’s cowardly, I say.”

“Why did you turn her away?” demanded Sandy, furiously. She wasn’t a spy, or a Grecoman. Can’t she be trusted, T—ff’s sister?”

I felt almost as angry as Sandy, but kept my temper. 

“That’s all right, comrade,” I put in, “you aren’t to blame. Go on about your business. Perhaps it’s better not to see her. Who told her we were here, anyway?”

Diplomacy is not always the best means to an end; my speech did lay the ruffled feathers of the courier, but it happened that the women overheard me, and my words were repeated afterwards by one of the girls in the family, who happened to be intimate with Helen. For the present, the courier departed with all indications of having taken the incident in good part. 

That evening the Eagle came, at my solicitation, and with him we planned our departure for the western rayons. Try as I would, I could not subdue the resentment roused within me by the incident of the afternoon. My manner toward the Eagle was certainly less cordial than usual, while Sandy's deportment was almost insulting. To his inquiries regarding my route after leaving Monastir I replied with unconcealed evasion. So we parted next morning in mutual ill humour. 

We were to leave the following Thursday. The committee would notify the nearest cheta to meet us in a village five hours distant in Liselo, to me a name never to be forgotten. 

Thursday came, but still we were delayed; then, it was to be Sunday. It was finally delayed till the next Thursday. 

The night before our actual departure the schoolmaster came home flushed with exciting intelligence. 

“There’s been another killing,” he told us. 

“Stoyan, one of Sougareff’s old chetniks, who was in town-”

“Killed the bishop?” I interrupted, excitedly. 

“No, was found dead out in the fields with a knife between his shoulder blades.”

“So the Greeks got him,” I said. 

“Always the Greeks,” commented Sandy. “What a murderous crew they must be.”

%CHAPTER XIV. 

\chapter{MASSACRE OF A CHETA.}

It was the middle of May already, with spring weather that was like a northern summer. Though we had enjoyed the spell in town, our blood throbbed for the open again and the comradeship of lusty fellows, camping out in the cool, mountain forests. 

Georgie and Moses came on Thursday afternoon, and the four of us started as if for an evening stroll. Arm in arm, we walked out into a lane winding through the open fields. Here we quickened our pace and so came, in less than an hour, to a cluster of huts, a dairy, where the city people often came during the summer months for a lunch of pure pot cheese and a glass of fresh milk. Here Moses and Georgie left us, with sincere regret and kindly feeling, I have often thought. 

The two of us quickly changed into peasant costumes, and just before dusk we set out with two peasants for a village five miles further out in the plain. 

Here was the chief source of the city’s milk supply. The plain was covered with grazing herds of sheep, and the pastures came in to the walls of the houses; hardly a garden was visible. 

Next day we had nothing to do but wait. It happened that the dairymen were on strike for higher prices, so they had large quantities of fresh milk on hand. In consequence, Sandy and I ate nothing all day, but drank only milk. It is the custom in that country to boil milk before drinking it, adding sugar, which makes it digestible. But Sandy had learned from me the habit of drinking it cold and raw, and so we guzzled two gallons that day, I believe. 

Towards evening Sandy was taken ill with severe pains in the bowels. I, too, began to suffer likewise. By dark Sandy became feverish. We were in no condition to travel, so, much against our will, we postponed our departure, suffering excruciating pains. It was perhaps a just punishment for our gluttony, but, as will presently develop, the moral was attended by an anti-climax.
Next morning we were better, and we slept during the forenoon. After a rational dinner of eggs, bread and cheese we were entirely recovered, and to lose as little time as need be we set out in the middle of the afternoon with two shepherds, traveling on foot toward the village in the foothills where we expected to find the voyvoda Petsov and his cheta. About half a mile away ran the highway, but we traveled parallel with it, toward the upper end of the Monastir plain, where the hills converged.

In an hour we came to a solitary hut, a temporary shelter for the shepherds of cold nights. Two were there now, their flocks nearby. We sat down to rest and to enjoy a few minutes’ chat. As we talked one of the shepherds jumped to his feet, then kept turning his head around, like a hound catching a scent.

“What’s the matter, uncle?” demanded Sandy.

“I am hearing queer noises all afternoon,” replied the old man. “An hour ago I thought I heard volley firing toward Liselo; now, I am sure I hear cavalry hoofs. Listen; stand behind the hut, you two.”

Sure enough; from the low, bushy trees lining the highway came a clear bugle call. Then there issued into a clear space of the road four mounted men. The distance was short enough to recognize them as Turkish cavalry scouts.

Scarcely a hundred yards behind them swept a solid column of cavalry, at a quick trot, till a full troop was in view. A few minutes more, and a second troop followed, then a third; fully three hundred asker. This was no patrol. Sandy and I looked at each other; his face was pale.

As soon as the troops had passed we two and our guides hurried on. There was a slight rise of ground before us. As we neared the top I heard a sound that quickened my pulse by a beat or two. Something like the short, sharp rip of tough canvas.

We reached the top. Before us lay a wide amphitheater, a semi-circle of hills beyond, against the blue background of the further balkans. In a small hollow we saw the red tiles of a village contrasting with the vivid green of orchards and vineyards.

The three troops of cavalry had just forked over into a smaller road that led up toward the village.

The four of us turned at a run for the hills to our right. In half an hour we struck a sheep trail, making for a village on a ridge about a mile above, overlooking the valley, and almost opposite Liselo. Meanwhile the firing intensified; Manlicher and Mauser.

We ran into a depression; Liselo and most of the valley were hidden from us. Finally we came up to the huts, just below the ridge, sheltered from the north winds. Above, on the summit, stood fully a hundred men, women and children. We toiled up among them, panting heavily.

There was little noise, only low, short exclamations from the women. Some of the men leaned on Grat rifles, some on hoes. A few of the women huddled babies to them; one girl sat with her back to the valley, her hands to her jaws, sobbing. But the rest of all those stolid faces were turned in the one direction—toward Liselo, not two miles away.

It required no field glasses to see what was doing. Above Liselo, among the sloping vineyards, lay a wide semi-circle of black figures, prone to the earth. Sometimes one rose and ran, then dropped again. Below the village, lined up behind a rubble hedge, stood the riderless horses of the three troops.

In the besieged village not a moving object was visible. To one side of the thick of the houses stood a more pretentious building, the bey's konak. It needed no expert to surmise that Petsov and his men were fighting behind its walls.

How long we watched this motionless scene I do not know. Perhaps not long, for obviously the arrival of re-enforcements would be shortly followed by decisive action.

There seemed to come a lull in the firing. A movement was going on along the black, semi-circular line. Then, with a sudden spurt, it broke up into two large, scattered groups which threw themselves down the slope, toward the konak. A sharp, vicious volley split the air, and another—all Manlichers.

About us the peasants began talking, low, but excited.

“A charge! God help them now !”

The running mass of figures swept in toward the konak. The firing ceased. A pause, then a ball of white smoke shot upwards from the garden of the konak, followed by a heavy detonation, like crackling sheets of tin hurtling through the air above us. The asker dropped flat, simultaneously.

“There went Petsov’s bomb!” exclaimed someone.

Again the soldiers leaped up and forward and began pouring over the stone walls about the garden of the konak. I distinctly saw several fall and roll over.

Another puff of dense smoke—another boom, then scattered firing. A low roar of voices, as of shouts, rolled across the valley.

The sun was setting, and objects blended into the descending shadow. Above the konak rose a blue wisp of haze, then through the dusk shone a brilliant flare. As darkness settled the konak burst into blaze, and the firing died down to a few scattered shots. Some of the women about us lay prone, sobbing with their faces in their arms. As the dark, moonless night fell we descended with the peasants to their huts. If any of Petsov’s fourteen men had escaped, there would be a hot patrol of the countryside next day; we could not linger here.

But such fears need not have worried us. Petsov’s cheta was as cleanly wiped out as Sougareff’s.

%CHAPTER XV. 

\chapter{AMONG THE MASSES.}

Even in turbulent Macedonia peace and quiet are a normal condition and violence the exception, though the exception is frequent enough there to render the nerves susceptible to an atmosphere overcharged with the electricity of human passions. For near a month after the massacre of Petsov’s cheta Sandy and I wandered among the Monastir villages; a period that was uneventful save for the little incidents that taught me more of the lives and customs of the people. Even our precautions against calamity became a routine, unattended by any excitement. We might, indeed, have been on a walking tour sanctioned by all the formalities of authority.

Our route was westward, but long and circuitous, for there were Turkish streaks in the population to be avoided. Down in the lowlands we traveled by night, but even that was not an inconvenience, for the days were hot and dusty. Once up in the hill country our marches were more frequently by day.

Though we came among them in peasant dress the people knew as comitajis, and as such we were received and cared for as were the itinerant monks among the peasants of feudal Europe. Often conditions made it necessary that we halt for several days in one village, and then we spent the idle hours, sometimes with the women and children in the village, while the men worked in the fields, or, more often, in the forest where the scrub oaks screened us from the observation of chance strangers. Here the women brought us food and drink, and helped the time pass with an hour or two of gossip.

I observed with some satisfaction that the villagers always trusted to our correct attitude toward their women folk. You might come among these people as a stranger with a leather bound passport, signed and countersigned by a dozen consuls and pashas, and you would see very little of their women. But the slip of paper, stamped with the seal of the Central Committee, which passed us along among them, seemed also a testimonial of our moral characters. Even the young girls were hardly shy, and would often come to us out in the forest, unattended, for a chat or a jolly exchange of pleasantries.
I was especially struck with the beauty of the children; dark, golden blond, hazel eyes and fair skinned, though delicately tinted from straw cream to nut brown where their faces and bare arms and legs were exposed to the sun. They came to us quite fearlessly, shyly, at first, though later we had them climbing over us and their arms about our necks. 

This confidence was pleasing, not as a direct, personal tribute, for it was more than that, but for what it signified; that of the many voyvodas and chetniks who had passed through here before us, most of them as armed bands, possessed of the arbitrary power of the organization, none had ever abused the naive trust of these simple people. 

This was one of the strong impressions that remained with me after my Macedonian experiences, especially after I returned to a conventional environment, where I could contemplate them in perspective. Many of these armed revolutionaries, who, I knew, had led loose lives in Bulgaria, would roam about Macedonia for years, as clean lived as the celibate monk, without his fanaticism to uphold them. This was one of the laws of the Committee I saw absolutely fulfilled, even by such men as Apostol and Tanne of Lerinsko. Curiously enough, even the old time brigands, who knew no laws but their own, were not only careful observers of women’s chastity themselves, but were ever ready to avenge such wrongs. As for the peasants themselves, they, like most peoples in a primitive state, were keenly intolerant of offences against the laws of chastity, making little discrimination, if any, between men and women. 

Early of evenings, when the men came in from the fields, the local committee would call a general meeting. Sandy and I were then propped up on stools, one on each side of the fireplace, facing the crowded room, the elders in a semi-circle on the floor before us. Then we talked. It was a duty demanded of us, for we were to them the priests of the revolution, and, as such, we must hold services. 

Most unattractive beings, at a first meeting: big, stolid, with rugged, leathery faces, neither bearded nor shaved, so horny handed that their palms were cold to the touch. At greeting they reached out mild handshakes, sweeping their gnarled hands back to the broad, red sash after each greeting as though depositing it there for future contemplation. Had Greek mythology developed a god of Pathetic Endurance, his likeness must have been that of a Macedonian peasant. You see his hunching shoulders supporting them all; a degenerate ruling race, a corrupt clergy, a breed of vampire landlords and a revolutionary organization, which, though self-imposed, still seemed at times to be holding back its rewards for future generations. Even to a Greek priest it should seem unsportsmanlike to strike down such a figure. 

Yet they took this business of revolution seriously, quite as seriously as they took their religion. Indeed, it was a clear manifestation of a strong religious spirit in itself, diverted from the usual channel of church service, for they are a people reluctant to accept symbols as a substitute for fulfillment. I could feel the rough, untrained brain force of those audiences sweeping through me, repelling all that was irrelevant to the one interest, the coming liberation. It seemed to me it was the one power of abstract thought they had, something quite different from mere chauvinism, for freedom was to them more than a vindication of Bulgar superiority; it was a perfect system of human affairs, applicable to Bulgars, Greeks and Turks alike, a sort of paradise to which all the world should aspire. 

The Russian revolution supplied us with inexhaustible material for our talks. They already realized that what Russia had done for Bulgaria could not be expected again unless for a gain to which they would be the losers. They knew more than the moujik himself knows; that the Russian Government and the Russian people are divided in interests. Occasionally a young fellow would rise during the discussions and expound on the doctrine that was even then penetrating the masses of the cities—socialism. His ideas on the subject, gathered from some village schoolmaster, might not be quite definite, but they took in all mankind. 

That not only the Bulgars, radical by temperament, were the converts of this new doctrine was impressed on me by an incident that was reported to us when we came up among the mountains of the Demir Hizzar caza. 

The schoolmaster of the village, where we spent several days, was an unimaginative young fellow, local secretary, but decidedly not a socialist. He told the story merely to illustrate the peculiar turns the Turkish mind could take. 

The teachers in all the Demir Hizzarsko villages had decided to strike for higher pay. To arrive at a common plan of action they agreed to meet at a central point in the rayon. They chose an old, half ruined monastery where they could talk all the evening and sleep during the night. 

They had already convened and were discussing their business, when a peasant appeared and warned them that asker approached. Their business was quite legitimate, so they waited, and when a heavy knock sounded at the door they admitted an uzbash and a squad of soldiers. The Turkish officer eyed them suspiciously. 

“You are revolutionists,” he said. 

“Not we,” replied one of the teachers. “We’ve met to decide on a demand for higher salaries,” and he showed a letter they had begun, addressed to the Bulgarian bishop in Monastir. The captain motioned his men out. 

“I am pleased,” he said, “for I want no trouble with the comitajis.” He seated himself and showed, by his inquiries, an active interest in their condition, their salaries, the cost of living and many similar problems. 

“I like men who desire to better their condition,” he continued. “You schoolmasters should organize throughout the country. Why doesn’t the committee help you ?” 

They were shy of so delicate a question. 

“Brothers,” he said, putting one hand on his breast, “I want none of you to commit himself. I know you are all revolutionists, whether you admit it to me or not. But though my present duty makes me your enemy, my heart is with you. Say nothing to compromise individuals, but let us discuss the question broadly. I desire to learn your views; you shall know mine.” 

His assurances carried so much conviction that they entered into a discussion that became freer as the evening went on. Finally one of the teachers remarked: 

“You are evidently a Young Turk.” 

“I am that, and more,” replied the officer. “My creed is broader than Turkey; I am a socialist.” 

“How is it then,” demanded the teacher, “that you can serve such a tyrant as the Sultan?” 

“Because,” returned the officer, “I am not an individualist. I believe in mass action. The army, as a whole, must be with us, and if we progressives leave it, how are those who remain to be converted ? While the majority are not yet of our minds, the army obeys the Sultan, and we, too, must obey conscientiously, until we can convince the majority that the Sultan should be no longer obeyed. Any other tactics would only split us into hopeless partisanship.” 

He left them, near midnight, enjoining them to keep the nature of their discussion from the peasants, who, believing him to be a liberal, might expect him to close his eyes to illegal acts and so bring trouble on themselves.
"And tell your voyvoda that I would like to meet him: Not in my official capacity," he added, laughingly, "but as friends. I am willing to come alone to any rendez-vous he may appoint. Such an interview might do us both good."

Shortly before we had been making toward the caza of Ressen, just above Lake Prespa, expecting to get into touch with the Ressensko cheta, when news of another calamity reached us. On this occasion there was less cause for depression among the peasantry, for only nine of the sixteen members of the cheta were killed and near a hundred soldiers had fallen, dead, or severely wounded. The seven survivors blew a passage through the Turkish lines after dark with their bombs. We barely missed the voyvoda, an ex-monk, known as the “Deacon," going to the city for medical treatment, disguised as a priest.

It was then we turned north toward Demir Hizzar and entered into communication with Tashko, the rayon voyvoda. From the village schoolmasters we learned that he was a young Vlach whose hobby was chemistry, that his cheta never exceeded half a dozen men and that he had been a student in the gymnasium. One day we received a letter from him appointing a meeting place in the mountains near a village called Babino.

We struck the trail in earnest that night and next morning reached a cluster of shepherds’ huts up a steep, narrow gorge. Here we slept all day, and on awakening near evening had word that Tashko would arrive that night. It was cold there, and we crowded around a fire in a smoky hovel with the shepherds. Outposts were stationed for a mile out in three directions, for it was known that a patrol of asker was about, undoubtedly the young socialist officer’s company.

Toward midnight a distant challenge woke us from a doze, and fifteen minutes later six heavily cloaked armed men entered and greeted us with the usual kiss of brotherhood. All were mere boys, and the eldest was Tashko, a slim, tired looking, narrow chested youth with a scraggy tuft of whisker on either side of his pointed chin. He sank wearily down beside me.

"So you’ve been in Monastir?” he said.

“Yes, near a month ago,” I replied.

“And how are the people there?”

“Pretty well. I suppose they would have sent you their greetings if they had known that I was to meet you.”

Tashko eyed me critically for a moment.

“To hell with their greetings,” he retorted, finally. “A dose of strychnine is what I would send them in return for their greetings.”

“They seemed a shady lot,” I admitted, “but I can’t definitely say where they are crooked.”

“Crooked!” he snapped, angrily raising himself to his elbow. “They killed Sougareff.”

“How do you make that out?”

“Sougareff was an honest man,” continued Tashko, “and he was making ready to put an end to their roguery. They persuaded the only honest man among them, a silly old priest, whom Sougareff trusted, to get him down in the plain for an important conference. I doubt if the old man knows to this day that there was treachery on the committee.”

This was a stunning revelation, but only half of what was yet to come.

“And what about Petsov?” put in Sandy.

“Petsov was a rough, uneducated man,” continued Tashko, reclining again, for evidently he was much fatigued, “but he could read through their double game. But he never suspected they were ready to betray him. They invited me down in the plain, too, but I flatly refused. I am one anti-Sarafoffist they can’t so easily get rid of.”

Sandy and I exchanged one awful glance. Here was matter to dream over.

%CHAPTER XVI. 

\chapter{A VOYVODA WITH A HOBBY.}

On our second day with Tashko I learned the reason of his reputation as a chemist. We descended into Babino and there, in the cellar of one of the houses, he revealed to me his laboratory, or rather his studio. For just then he was interested in chemistry only in so much as applied to photography. Out of the wood of a kerosene box he had fashioned a camera, a sort of a large pill box telescoping into its cover. As a lens he used the glasses of an old binocular. He showed me some of the productions of this curious device, in negatives and in finished prints. One represented the cheta, the central figures of which were quite clear, but near the edges they broadened out over the adjacent landscape, as though reflected in convex mirrors.

Now, on coming to Macedonia I had brought with me a small, plate kodak, of about 4x9 size, quite a perfect specimen of its make. In the swamps of Karafferia Luka and I had experimented with it. First, it fell into the water, or rather, I had fallen into the water with it in my knapsack, and Luka sent it to Salonica to be repaired. Then, a batch of undeveloped negatives, sent to the Voden committee for development, had been opened by that body prematurely and ruined. Luka had wanted to invade the town on a punitive expedition. The second batch had been ruined because the courier had placed the package of plates inside his jacket and then perspired. The third batch I had carried with me to Monastir and given to the Eagle to have them developed, and had not heard of them since. After that I had carried the camera with me only because it was intrinsically too valuable to be thrown away.

The nearest approach to a joyful smile I ever saw on Tashko’s face appeared when I produced my kodak from my knapsack. It almost repeated itself when he discovered that he had two dozen Lumier plates fitting it.

That week together with Tashko has always seemed to me a distinct break in my revolutionary experiences Even Sandy became interested in photography, and the chetniks had long since become secondary kodak fiends.

Tashko had lovingly scrutinized my camera, taking it apart and putting it together again. We spent all day in the village, expending a dozen plates in snapping each other, alone, in groups and as a whole cheta. Then we posed a group of children for a time exposure, using both cameras.

When the first shades of twilight deepened we dived into the cellar and began developing. First we tried one of the plates from my camera, and it proved a complete success. Next came the group of girls taken with the pill box. Anxiously we watched the negative as the high lights slowly appeared; the dark wall in the background, the black stripes in the skirts, the red stockings, when:

“Asker! Asker! Run for your lives! They are surrounding the village!”

Tashko lowered the tray containing the developing fluid carefully, then uttered a fervent, profound curse. In the house above sounded quick commotion.

“Drop it into the bath and cover it over!” I exclaimed, “we can get it later.”

He did so, and we rushed out into the village street. The dogs were barking furiously down in the lower end of the village. The schoolmaster secretary and the members of the local committee gyrated about us in the darkness.

“Where are they?” demanded Tashko, in a loud whisper.

“Down by the brook,” replied a peasant. “They are rousting out old Natcho, the moftar.”

We heard a distant pounding and a murmur of excited voices.

“A squad just passed up the other side of the brook, toward the upper end of the village,” exclaimed a young villager, running into our midst.

“Nick,” said Tashko, to his second in command, “lead the way up through the churchyard. Pass by the trail over the upper ridge; they can’t have many men posted there. Loose your bomb pouches, the rest of you. I’ll follow presently.
Taking Nick's lead, we struck slantingly up the mountain side, swiftly, though cautiously, ascending above the village. The noise down among the houses was increasing and lights flickered back and forth, from house to house. 

Nick pushed on, past the churchyard, till we came to a sheep path passing over the ridge. Here we flattened down, our pieces before us, watching the black sky line. 

“Let’s wait for Tashko here,” whispered Nick, fumbling with his bomb pouch. 

We waited, perhaps ten minutes. Then came a slight rustle of brushwood behind us. 

“Go on,” came Tashko's whisper, “we'll rush them if they're there.” 

Up again, running swiftly, our bodies bent. I fancied I heard a curious squeaking, as of a tin pail dangling. We reached a clear space of level ground, the summit. Across this we crept, almost on our hands and knees, pausing as we came to the opposite descent. 

“Let’s rest,” whispered Tashko. We flattened down again. A few minutes passed. 

“Look,” whispered Tashko, in my ear. My eyes followed his pointed hand. A hundred yards away, faintly silhouetted against the blue mist in the valley, stood a dozen, silent figures, huddling together, as though cold; woeful looking shadows. 

“Asker,” I murmured, “we could easily settle with those few. 

Tashko gave a low chuckle. 

“Leave them, they might be socialists.” Even through the darkness I fancied I saw that sardonic, half derisive smile of his I had shown some enthusiasm about arranging a meeting with the young Turkish officer. 

We glided down through a forest of oak saplings, came out again on a trail, then struck vigorously upward. Thus we pushed onward, hour after hour, though resting frequently. Grey dawn came, then bright daylight. We were far into a heavy forest with a thick underbrush. Tashko had been bringing up the rear; I turned and saw him coming. In one hand he carried his Manlicher, in the other a tin bucket. As he came nearer I saw it was covered with a piece of thick cloth which was wet. 

“I wasn’t going to leave it to those clumsy fellows to finger and spoil,” he said. My unrestrained laughter hurt him a little. 

“You haven’t the love of the art in you,” he exclaimed, angrily. “Should I let a lot of dirty asker spoil my work ?” 

We pitched our camp where we were, and the little tin pail was carefully hidden in a tree hollow. About noon a peasant arrived from Babino and reported the asker gone. It had been the young socialist, and he had been very decent, not molesting the villagers more than was necessary. Toward evening the schoolmaster came with a boy leading a donkey loaded down with the pill box and all the photographic supplies. Immediately we prepared to resume our developing that evening, constructing a sort of tent with our cloaks to keep out the starlight. Fortunately the water had been very cold, and the gelatine of the rescued negative had not melted. It came out a tolerable success; a trifle under developed; to this day I keep a print of it, three little girls in a group, a bit vague about the feet. 

For three days we remained in that one spot, devoting almost every minute of our waking hours to photography, a fresh supply of plates arriving from Monastir by a peddler, whom Tashko had commissioned a week before, just as the old supply gave out. I must admit that I, too, was thoroughly possessed of the craze at the time, for the glowing enthusiasm of the young Vlach was contagious. 

On the day after our meeting I had forwarded, through Tashko, a letter to Ochrida rayon, announcing my arrival and my desire to meet the cheta. Now, about a week after, came the answer—Petrush, with his cheta, was waiting for us in a village some two hours distant. 

After dark we set out, and after a vigorous march along an upward trail we suddenly brought up into a howling, barking pack of dogs, then found ourselves among houses. Somebody challenged, a door opened to admit us and closed after us again. It was dark, till a second door opened, and we entered a room brilliantly illuminated by an open fire. 

Against the wall opposite us rose eight, slim, youthful figures. 

It took me ten minutes after the handshaking to recover myself. Here were eight, slender boys, nattily dressed in what appeared to me cadets' uniforms and students' visored caps. Not a mustache among the whole lot, their faces were as smooth as a girl's. And by no means the eldest was Petrush, tall and straight, but with the features of a page in a mediaeval court. 

Tashko grinned. Later, when we were alone in the other room, arranging our paraphernalia, he remarked : 

“A corporal’s squad from a military academy, eh?" 

%CHAPTER XVII. 

\chapter{PETRUSH "THE JUST.”}

Our first conference with Petrush lasted all night. He was a talkative youth, rather boastful. Though we had much serious business to discuss, I knew, before we rolled up in our cloaks at dawn, that Petrush had put five hundred asker at their wits ends by an elusive series of flights some months before, that he had refused to meet an Italian gendarmerie officer a week since, that he had passed the fifth grade of the gymnasium, and that his father, who was a fisherman on Lake Ochrida, used a silk net in his fishing 

The two combined chetas gave us some force now, so we slept in the village, but were up by noon. The Ochridsko boys seemed unusually intelligent; all had been in Bulgaria, and one, whose pate was as bare as his face, and whom the others called the “hodja” because of his talkativeness, had even traveled as far as Russia. All of them spoke clear cut Bulgarian. 

It was our last chance, so Tashko and I did some photography, taking each cheta, the two combined and each of us individually. We divided the plates between us for future development, promising to send each other prints. 

Early in the afternoon the villagers brought ten horses, one for each of us who was to travel. Tashko made me a present of a complete outfit of the necessary chemicals for developing, some trays, a piece of red glass out of which to make a dark room lantern and several dozen Lumier plates. Then Petrush, his cheta, Sandy and I mounted and set off up a steep trail toward the snow peaks, waving the Demir Hizzarsko cheta a last goodbye as we entered a fringe of young oak forest. 

Our mounts were tough, mountain ponies, and they carried us up along narrow, winding trails, up the bouldered beds of dried mountain torrents, among bare crags, over crumbling sandstone, their shoulders working like steel winches against our thighs. Below, first on one side, then on the other, lay all Demir Hizzar through a blue mist, growing flatter as village after village appeared, only to dwindle to mere red specks. Level with our gaze sailed hawks and eagles in lazy, sweeping circuits. The air grew keen and crisp, our words and shouts seemed to grow distincter; up here you could let yourself out. Whoever would could see us now from below, asker or spies; not even a Mauser bullet could reach us here, and we felt it. Up those same heights Philip and Alexander of Macedonia together had clambered with their armies to meet the barbarian Ilyrians in the defiles above, but that as a bythought; contemplation of the past troubled none of us, for as we ascended we became a group of shouting, laughing, chattering boys, conscious only of our present exuberation.
It was late June, and though the sun dropped beyond the white peaks above, twilight came slowly and deepened reluctantly. The clouds overhead glowed in the horizontal sunlight, diffusing through the atmosphere a purple mist and tinged the peaks a warm red. One more burst of effort, and the ponies broke into a trot over a level space in between the snow inclines, crunching through the thin crusts in the hollows. We were in a defile at the summit. Here we dismounted and built a fire of dried wood which the boys had picked up in the timbered regions below. “Hodja” began chanting a Turkish love song while he prepared for the spit a young goat which the villagers had given us. Those few of us not actively engaged in preparing supper crowded around the blazing fire, our cloaks opened wide behind our backs that the heat might be kept within the circle. Then the whole crowd broke into a militant revolutionary song; shouted it, whooped it. I seemed to have known Petrush and his boys since olden days. It was a schoolboys’ outing just then, nothing more.

Two peasants, who had accompanied us to take the ponies back, feasted with us, then reluctantly departed on their return. It must have been late, though still not dark, when we resumed the march on foot, filing downwards, Petrush and I leading and conversing, all of us more subdued. A sudden turn brought us to a full view of the lowlands, in deep purple, save a long streak of glowing red on the southwestern horizon, the famous Lake Ochrida, some day to be a center of attraction to European tourists. Far to westward gleamed a rugged, white line: the mountains of Albania, ancient Ilyria.

We seemed to plunge abruptly downward into night, into a black, damp mist, deeper and deeper. I knew when we had passed the timber line only by feeling wet boughs slapping my face. Finally we struck a hut on a brief level, and into this rough shelter the ten of us crowded to pass the night, leaving two shepherds to watch over us as well as their flock.

Next morning was clear; just below us was a village into which we precipitated ourselves down a steep incline. Here, too, the insurrection had been lively, for the heaps of crumbled walls equalled the habitable houses in number and the trees in the gardens were bare and black.

At breakfast this morning I observed a bundle of old and greasy papers choking Petrush’s despatch bag. Among them was a well worn copy of the constitution and by-laws.

“Why do you carry around the rayon's archives?" I asked.

“They aren’t archives." His voice sounded resentful. “They’re cases. I carry them with me because I have to study them."

“Cases?" I exclaimed. “What cases?"

My ignorance seemed to hurt him. He explained, laboriously and patiently:

"Articles 10 to 15, of Section IV. of the constitution say: ‘The voyvoda shall settle disputes, etc/ You know we're boycotting the Turkish courts.”

"Do you have them presented in writing?” I asked in astonishment

"Why not? I can’t judge them hot off the fire. Here, as an instance, is a land case. The priest presents one side, the schoolmaster the other. Twenty-five years ago this man, Ivancho, leased a field to Stoyan. Ten years ago Ivancho wanted his land back, but Stoyan refused to give it back unless he was paid for a barn he built on it and an orchard he planted. They went to court and the litigation continued till the boycott was declared. Lately they put it up before me.”

"Some day,” I said, "a legal court will reopen these cases.” He agreed, but if ever I read a man’s thoughts aright, I did then. Petrush saw the cases re-opened, years hence, but he also saw himself, then a passed student of law, re-trying them.

I had not long to wait to witness one of Petrush’s courts in session. The local committee ushered us into the school house, which was empty because it was Saturday. At first we had an informal talk with the peasants, but presently Petrush, ignoring chairs and tables, which he ordered cleared away, threw his cloak down in a corner, seated himself on it and invited Sandy and me down on either side of him. The local committee and the chetniks scattered along the sides of the room. The mass of the villagers took up the center and back. The court was open.

There were plaintiffs and defendants in plenty. They formed groups on either side of us. The rest of the villagers seemed to take the proceedings as in the nature of a cock fight.

The first case was a communal one and was presented by the local committee. Asker had come into the village some time previously and assessed the community six liras for the maintenance of police posts along the road to the town of Ochrida, as a protection against brigands for those going to market.

“Last time they came, six months ago,” said the moftar, “they flogged us for not paying. It was twelve liras then. Shall we be flogged again for only six liras ?”

“If it were only six paras,” replied Petrush, “I would flog you for paying. So, if you get flogged in either case, save your six liras. If they flog you again, send a delegation to Monastir and roar out your complaints to the foreign consuls until they take notice.

“This police post business,” continued Petrush, aside to me, “is an old game worked out. The people gave the money for years while the country was full of Albanian brigands, but there are still no police posts. The money goes into the kaimakam’s pocket.”

The next was a charge against the local bey who owned the only flour mill in the community.

“He has raised the price of grinding by a third,” was the complaint.

“Don’t go to his mill,” spoke Petrush.

“There is no other.”

“Build one; you have water.”

“None of us has the money.”

“Then make it a communal enterprise. Labor costs you nothing. I authorize you to draw on your treasurer for the cost of materials, which shall be returned to the treasury from the first profits. After that you can run the mill at cost, or at a small profit for the benefit of the treasury.”

After the general cases, People versus Government, defendant not present, individual cases appeared.

A woman pressed a long standing suit for divorce. I literally translate the ground of the complaint:

“Incompatibility of temperament. The old man is all right, but we can’t agree on any one point.”

I nearly cried out in astonishment. This seemed getting near home.

“If you both agree on it after this long period of consideration,” said Petrush, “I decide in favour of your separation. Pop,” he continued, addressing the village priest, “ see that this case is presented in such a way to the bishop in Ochrida that the church also stamps its approval on my decision. ”

Next came a criminal case. One had gone to a distant village and stolen a horse. The offender was sentenced to twenty blows from a cane. Petrush told him:

“Next week you will take out a legal passport and go to Bulgaria. You will leave your wife and children here. I will give you a letter to the committee of representatives in Sofia, which will secure you employment. You shall support your family from there. If you return before the end of a year you’ll regret it. If you commit crimes in Bulgaria I shall send your wife and children to their relatives and burn your house.”

The three weeks we spent in Ochrida were much like the progress of a circuit court. Next to judicial cases, Petrush enjoyed elections, holding them whenever the constitution permitted. Each election was accompanied by a lecture on civil government.
He carried it so far that he even caused a rayon election to be held wherein he was a candidate for re-election as voyvoda. He won, as he wrote me later, but there was a chance of defeat, though to him the enjoyment was probably worth the chance. 

As we began skirting the lowlands the summer heat made sleeping in the villages uncomfortable, so our life became entirely an outdoor one. Some of our visits, in fact, amounted only to meeting the villagers out in the forest, where courts and elections could be held just as well as in the stuffy houses. 

They were three weeks of pleasant outing. The country was not so rugged, the forests were older, the trees were larger, with less underbrush, than I had seen elsewhere. We camped in open glades in the tall grass, cooking our own food, being usually a kid or a lamb which was spitted on a green pole made of a sapling, turned on two crotches, one on either side of the open fire. The young fellows from the villages would join us in wrestling or jumping matches and "putting” boulders or other athletic feats. We could shout and sing as loudly as we liked, for asker do not climb about much when the mountain torrents are dry. We could satisfy our thirst at a trickle of water among the stones, hardly sufficient for a company of fifty or a hundred soldiers. 

In Hodja we had a merry comrade; he was as jolly a rogue as Satyr in the swamps. He knew countless comic songs, most of them parodies on popular, sentimental melodies. Then he would impersonate the revolutionary orators in Bulgaria, those who contributed only oratory to the cause. If a priest was present, especially one a little above the average intelligence of his kind, Hodja would deliver a typical socialist harangue; in those priest-ridden countries socialist orators become most virulent when referring to the church. The conclusion was usually rolled out in tremendous frenzy: “Never shall there be peace on earth until the last king has been hanged with the waist cord of the last priest.” 

They are a peculiar people, those Bulgars, with that sense of humour capable of extracting laughter from subjects one would think most sacred to them. Irreligious, in the commonly accepted sense, with hardly any pride of race, apparently incapable of fierce emotions, you look in vain for those qualities that seem essential to the fanaticism of martyrdom. One hardly associates a Jean d'Arc or a Bishop Latimer with the tribe of humorists. 

Yet, of those same Bulgar youths, who seemed to me so devoid of reverence of what we call the higher ideals, unbelievers in the solace of a spiritual hereafter, many have since closed their lives no less dramatically than the martyrs of early Christianity. Of those who died by the hand of the enemy I say nothing. But it is a matter of record that no whole cheta has ever capitulated to an enemy, though such terms are invariably offered, and the individual members of chetas who have been taken alive could probably be enumerated on the fingers of your hands. It is difficult to state such a fact without seeming to strive after emotional effect, but I might publish an interminable list of instances in which the survivors of a besieged cheta, finding escape cut off and nothing left but surrender, have deliberately chosen the alternative of ending their own lives with the last cartridge. What I should have done had this alternative been presented to me, I do not know, but it was this sentiment among my revolutionary comrades which gave the cry of “asker!” a peculiarly ugly sound to my ear. 

I have already referred to the fight between asker and the Ressensko cheta, wherein seven men escaped, though sorely wounded. The chief, known as the Deacon, had gone into Monastir for medical treatment, but the others had found shelter in the mountains where they were attended by a sub-chief from another rayon who had once been a medical student. 

Ressen lay next to Ochrida, and directly in my line of travel. We had tried to communicate with the Deacon’s lieutenant, Krusty Trikov, but so secret was the location of the comitaji hospital that our letter traveled much before it reached its destination. Finally an answer came from Krusty announcing the recovery of himself and his comrades and appointing a rendez-vous near the boundary of the two rayons. 

We happened to be in a village called Plakia at the time, much engrossed in another matter. A rumor had spread some days before that a notorious Turkish brigand, Islam Tchaoush, had invaded the district from his headquarters in the Albanian mountains and was hovering about on the watch for a profitable capture. These visits had been repeated with religious regularity each summer, for nineteen years, and rarely had Islam failed to carry off some prosperous peasant for ransom, in spite of the furious efforts of the Ochridsko committee to terminate his career. In the previous summer Petrush had actually come up with the wily old outlaw and exchanged a few shots with him, but on account of some misunderstanding of signals, the village militia had failed to cut off Islam’s retreat 

For several days we had been on Islam’s trail, and had finally tracked him down to the vicinity of Plakia. Couriers had been sent with detailed instructions to all the villages, and now every trail was watched by day and ambushed by night. The whole arrangement was well planned, as was proved by the brilliant results afterwards achieved. 

In Plakia we learned that Islam and his seven men had taken refuge in a nearby Turkish village which was garrisoned by asker. There he stayed, while we maintained a feverish watch from an adjoining mountain, hoping he would eventually emerge. To expedite the issue, we had written a letter to the Italian Gendarmerie officer stationed in the town of Ochrida. This message I had composed in Italian, a little incoherently, perhaps, for my study of that language, years before, had been desultory. To an ordinarily intelligent person it would have been evident that no Bulgar, nor even a continental European, had written that note. American caligraphy is quite distinct from that taught in Europe, excepting England, perhaps. At any rate, it must have been obvious to those into whose hands the letter fell that a foreigner had written it. And, of course, there was only one foreigner among the comitajis. 

It was then that Krusty’s letter came from Ressen. As our communication to the gendarmerie officer seemed to be bringing no results, and the village chetas were pretty well covering Islam’s avenues of escape, beside which the rendez-vous was almost as near the Turkish village as Plakia, we decided to join Krusty that night. 

We started before dusk, a piece of carelessness that is usually committed after a long period of unmolested security. The Ressensko boundary was across a bare valley from us, and as we filed over the green fields we were undoubtedly observed by Islam’s lookouts, stationed in the forest above the Turkish village.

After a march of four hours we came into rugged country again. Petrush was telling me what he knew of Krusty. A year previously he had come from Bulgaria as a chetnik under a voyvoda sent by the Central Committe, a schoolmaster of delicate health. In six months the voyvoda had found the life too strenuous for his frail physique. Whereupon he had returned to Bulgaria, and Krusty, fulfilling the pre-arranged plans of Boris Sarafoff, of whom he was a devoted admirer, took command of the cheta. But for some reason which Petrush could not explain, Krusty quarreled with the provincial committee in Monastir, and the Deacon was sent to relieve him of his command. There had been some trouble between the Deacon and Krusty as a result of which Krusty had once started for Bulgaria, but immediately a protest on the part of the village committees had obliged the Deacon to retain Krusty. 

A full moon was rising as we approached the rendez-vous, the base of a wooded hill. A shrill whistle greeted one of our men who went ahead to reconnoitre, then dark figures began pouring out of the lower fringe of the forest. A broad, bearded, gorilla-like man darted forward and began gripping hands, one after another, as though stabbing enemies in a spirited affray. His dark colored, skin tight, Albanian costume, minus the fluffy sleeves, gave him the appearance of a young, vigorous Henry the Eighth (by Holbein), come to life, an illusion which was heightened by the quaint dialect in which the low, rumbling greetings were uttered. A powerful fellow, I realized, as he gripped my hand. He retained his hold, transferring it to my arm, slung his carbine strap over the other shoulder and dragged Petrush and me toward the trees, as if we were captives. 

We were hustled through a fringe of scrub oak and underbrush into an open glade on the brow of the hill. Krusty rushed us forward and threw us down on a bed of boughs and dried grass spread under a big tree, then flung himself down between us. The others, eighteen in all, we were, disposed themselves likewise under the trees; it was good to be one of so big a force again. 

I could observe Krusty’s face by the moonlight as he spoke, deep and guttural, as though through his chest. It was broad, fringed with fiery red whiskers and beard and a mop of chestnut hair. No wonder, I thought, that Sarafoff is a power in the organization, with such followers. 

“I’ve heard a good deal about you in Monastir, Krusty,” I said.

“Whom did you meet there?” demanded Krusty.

“Your Ressensko delegate, for one, Legusheff.”

“That damned wolf,” growled Krusty. “Never mind—if you are a friend of his. I’ll treat you fairly, but don’t stay in my rayon any longer than you need. I don’t believe you are, otherwise he would have warned you against coming here.”

Here was another puzzle to solve; why Krusty, a Sarafoffist, should be at outs with Legusheff. Like many other simple problems its solution was hidden by its very simplicity.

%CHAPTER XVIII. 

\chapter{KRUSTY OF RESSEN.}

It was Krusty’s deep voice that awoke me in the morning.

"'Take his gun away from him,' says the Deacon, but not a chetnik raised his hand. Then he turns to the militia voyvoda and says: 'If you don’t disarm him I’ll flog you.’ ‘Flog me then,’ says the villager, ‘I’ll only take Krusty’s gun if he gives it to me.’ Then I shouts out to the Deacon, 'why don’t you take it from me yourself?’ He didn’t answer. ‘I’ll go to Bulgaria,’ says I, ‘a Patagonian brigand like you shan’t be my leader. But I’ll carry my gun with me to the frontier.’ So off I goes, with four of the boys, but when we got to Krushevo, a delegation came from Ressen and told me if I went they’d have no cheta at all in Ressensko rayon.”

"What started the row?” asked Petrush.

"It was the committee in Monastir. They appointed a rayon committee in Ressen, when it should have been elected. But the rogues they appointed! All shopkeepers. That was all right, but soon the villagers came complaining to me they were forced to buy their stores from these committeemen, at double prices. I looked into it; they called it a boycott against the enemies of the organization. They wanted to ruin other shopkeepers who didn’t want Legusheff as provincial delegate, and wanted an election held. Damn it, those other chaps were right. I dissolved the committee in Ressen. They complained to Monastir, and Legusheff issued a decree of some sort against me. I sent some husky young villagers into Ressen, and they dissolved the committee. Then the Deacon came, and I gave in to him, because I saw he was an educated man, and I am only an illiterate fellow with no headpiece for tactics and the kind of educating of the masses that the new constitution wants. I am only good for fighting, and that isn’t what’s needed now. But the Deacon turned out a terror. All his tactics were against our own people, against the schoolmasters. And then he kept diving into the village treasuries and sending the people’s money into Monastir. Trouble would have come to a head soon if the fight hadn’t come off. It was more than his wounds that took the Deacon into town. I haven’t seen him since the fight, he broke through one way, the rest of us another.”

Just then came the rustling of leaves and brushwood, and a perspiring peasant appeared. He knelt beside Krusty and began whispering. Krusty’s voice could not be so easily deadened. I heard his reply:

“Two battalions! Where are they now?”

“One took position on top of the opposite hill; the other has broken into small detachments which are patrolling all the cow paths through the forest. Those on the hill lie low, so we can’t see them.”

I sat up, keenly awake.

Krusty turned to Petrush. “You people must have thought you were knocking about the city park in Sofia last night,” he grumbled.

One of our outposts crept softly in and reported a detachment of asker approaching along the trail we had taken the night before. There was a general but quiet commotion among the men, the snapping of breech blocks and the rattle of cartridge clips. The peasant hurried down through the brushwood. A tense silence followed, all of us on one knee with our pieces cocked; then came the shuffling of many feet and the click of hob-nailed boots against stones, passing just below us. I could barely make out, through the foliage and above the underbrush, the moving fezes of asker. They passed on, and the footsteps receded. The peasant came hurrying back.

“They are only following the trail,” he said.

All day we lay there, never speaking above a whisper, startled by every movement of the brushwood, fearing it might be an outpost coming in to warn us of another approach. Usually it was one of the loyal peasants come to report the movement of the patrols or to bring us food, always carrying axes over their shoulders, as though out only to cut firewood.

The approach of night brought some relief from the nerve tension, but we remained where we were, for the peasants had observed that all the trails were ambuscaded, so we preferred to continue the present situation rather than revealing ourselves by the almost certain skirmish that would have attended an attempt to cut our way through.

When morning came again there was a rather anxious wait for news from the villagers. One came at last and told us that the asker were still about, but the main body had moved northward into Ochrida, toward Plakia. At noon we heard the last detachments were gone. The reaction manifested itself in a steady chatter among the chetniks; Hodja would have given us a comic song in Turkish had not Krusty restrained him.

It was still daylight when we broke camp and dropped down to the running brook beside the trail. We filed leisurely down the path toward Krushevo. Somebody remarked: “Look at that fellow tearing down the road from Swineshto.” A man mounted on a pony was riding furiously down a narrow path along the face of the opposite mountain. We watched him in silence as he approached, not troubling to conceal ourselves, for obviously he was a peasant. As he reached a point directly opposite, he saw us and suddenly pulled up. Then he plunged his horse down through the brushwood and rode toward us.

“Islam's escaped!" he shouted, even before he reached us. “He broke away late in the afternoon, two hours ago, came over to Swineshto, took old Save and his son and five horses and made for the hills. I was coming to Krushevo to let you know.”

All that was not lawyer in Petrush boiled up in a crimson glow into his delicately shaped features. His lips twitched as he snapped out his commands: “Krusty, you'll help me; if you watch the southern pass, we'll circle around above Plakia. Come on, boys, tighten your belts This will be the swiftest and the longest run we've had yet. Good-bye, old fellow, and you too, Sandy. No use your coming with us, it might be a week before you could be back here again.” We watched the eight young fellows running up the mountain trail toward Swineshto, and then disappear in the gloaming.

The next day we were camped on a mountain side overlooking the southern trails toward Debor in Albania. The night before Krusty had covered them with the village militia chetas. Islam could not travel so swiftly, for he would be hampered by his captives. About noon we caught a low sound of volley firing from Ochridsko way.

“Petrush has got him,” remarked a chetnik. Krusty was stretched on the ground, listening. “Islam doesn't carry Mausers,” he replied presently.

The firing continued till nightfall. Meanwhile, believing that all asker must be up where the fighting was going on, and more than expecting that Islam would take advantage of the diversion by passing our way, we broke camp and descended into an open valley below us. Sandy, with several of Krusty’s chetniks crossed a brook and continued down the opposite side from us, the stream between us.

Suddenly, we, on our side, saw two figures emerge from a strip of thick timber towards which Sandy and the chetniks were advancing. The first glance revealed them as asker, and obviously the scouts of an approaching body of soldiers.

“Look out for the wolves ahead of you!” roared Krusty, but not before the chetniks had almost collided with the two scouts. Both parties came to an abrupt halt, then the two Turks whirled about back into the timber.

The chetniks quickly joined us, and then the whole body of us, Krusty leading, broke into the brushwood, heading south toward a low rise, making a circuit to the village to which we were going. We came hurriedly in among the houses and the packs of barking dogs. The villagers gave us bread and cheese with which we were filling our knapsacks when a boy came panting in, crying out that a large body of asker approached and were debouching into the fields on both sides of the village, meaning evidently, to trap us there. We spread out into firing formation and ran across the plowed clearings for some sparse timber. It was growing dark, so we could not have been seen at any great distance; at any rate we were not fired upon, though the barking of the dogs indicated to us the near presence of moving bodies of men. As we drew into the timber we fell into file formation again, but continued hurriedly onward.

That night’s march was a hot one, mostly across plowed fields, for we crossed the valley to the mountains opposite. Before morning we were camped and asleep in thick brush, near a village, but that night we made another hard march, bringing us up into the range of hills that separate Lake Ochrida from Ressen.

On the second day we learned that the soldiers were still hot on our trail. Cavalry scoured the country and infantry patrols infested the mountain paths. In several places where they appeared the officers tried to bribe the children with coins and soft words.

“Have you heard of the American around here?” or “Has the foreign comitaji been in your village?” From this it was obvious that the presence of a foreigner who did journalism occasionally was annoying the authorities.

One night we came out on the shore of Lake Prespa, a sheet of water second only to Lake Ochrida in size and beauty. In a secluded spot on the beach we found six fishermen waiting for us and embarked with them in two large boats.

I do not know the exact size of Lake Prespa, but I remember that it was only on clear days that you could see from shore to shore. We were two nights traveling from the north shore to an inlet half way down the western shore. In the brilliant moonlight we rowed along the base of a line of high, rugged cliffs, perforated with small grottoes. In one of these I saw the altar of a chapel built centuries before, probably while the Romans still persecuted the followers of Christ. Even now the fishermen still worship there and set burning tapers before the ancient ikon suspended from the rock wall.

The country hereabout was thinly populated and wild, almost impassable above the lake. On the second day we camped openly on a wide, pebbly beach between two tall cliffs. The shelving, white sand dipped into the rippling water at our feet, so clear that you could see the bottom continuing its downward incline for fifty feet before it finally dived into the indigo depths beyond. Fish, in schools and alone, small and large, sailed by, as distinct to our sight as though they scurried through a blue mist. We bathed freely, going in several times during the day. Then we fished, but it was the fishermen who pulled out the fish large enough to bake over the open fire.

There was a village further up the inlet which Krusty and I visited several times. The president of the local committee was an Albanian youth who lounged about with us as though he were only a prosperous villager with plenty of time on his hands. Krusty mentioned to me casually that he was the bey of that district, that he owned all the land about the village. It was he who had arranged our passage by the boats. Krusty called him “Beggy” in an easy familiar way which indicated their long intimacy, but it took me some time to become used to an Albanian bey as an official in the organization. I soon realized that the bey was also in communication with his countrymen beyond the mountains, and that among them there was increasing revolutionary activity. Had there been time Beggy would have arranged a short trip for me with the object of meeting a revolutionary Albanian cheta which operated over toward Kortcho.
After a week of picnicking at the edge of the lake, we began slowly returning northward, visiting the fisher villages along the lake shore in regular order. By this time the agitation of the authorities over the reappearance of myself with the supposedly annihilated cheta had abated. Our visits in the villages, after the disaster of May, had a beneficial influence, even aside from the enthusiasm produced by the thrilling accounts of that affair related by the chetniks to the groups of fascinated peasants.

At our first village meeting Krusty almost frightened me with his violence of speech. The locality was especially impoverished, having been destroyed during the insurrection and only partly rebuilt since. The contents of the local treasury, which had been augmented by a tax levied on peasants from other parts who came there to cut timber, had been loaned out to indigent members of the community, which Krusty discovered after demanding a rendering of accounts. Nor could the treasurer collect the debts, though all debtors were present. I thought Krusty would flay them on the spot. He stationed a chetnik at the door with fixed bayonet, then began a thunderous denunciation. But no money was produced.

“Do you know what I shall do to you?” he yelled. The assembly was mute. “You Patagonian brigands!” (His favorite invective; he confused Patagonia with Calabria, whose brigands he had heard of.) 'I'll tell you what I shall do! I shall appoint three old women as local committee, and you’ll have to obey them. And they’ll know how to take care of your money.” They really seemed impressed by this threat, for about a quarter of a lira was repaid, though twenty were still owing. They evidently knew Krusty, for his wrath subsided. “Poor devils,” he whispered to me, aside, “what are you going to do about it? Just bellow a little, to prevent them from making it a habit. Yarn’,” to the treasurer, “give me a light from the coals, will you? Look out for me, next time; I’ll appoint your wife treasurer, that’ll upset your household.”

In much the same way he administered justice in the local disputes. At the end of a few days, I, as well as the villagers, had grown used to Krusty’s explosive mannerisms. One of the chetniks told me that Krusty had never had an offender flogged. He told me also of one who had carried information to the kaimakam which had resulted in a rather narrow escape from asker. When the offender was dragged up for judgment, it was found that he had six small children. In no other rayon would he have escaped death. But Krusty sentenced him to continual confinement to the village; he could not go out into the forest to cut wood unless accompanied by one of the village militia. That was the only case of espionage that had ever occurred in the rayon.

Coming up into upper Ressen again, we kept to the higher ranges. It was here that a letter from Petrush finally reached me. Sandy read it aloud to the assembled chetniks and villagers.

“We followed Islam's trail all night after leaving you,” continued the letter, “and next morning almost came up with him. We would have brought him to a stand if we hadn’t run into a company of asker who forced us into a defensive position. The kaimakam was there in person, and we broke his ankle with a bullet, which made him so angry that he went into Plakia and hung a villager. Two asker were killed in the fight, but at dusk we withdrew without any loss or injury.

“Meanwhile the village chetas had cut off Islam’s retreat, and one of them pursued him to the outskirts of a Turkish village. He entered there and demanded a relay of horses for himself, his men and the captives. When they refused he began beating their moftar, whereupon they shot him. They released the captives who returned to Swineshto where we met them and heard how Islam's long career was ended.”

That was all that Sandy read aloud, but there was more for me personally. Petrush seemed ostensibly contented that Islam had finally been disposed of, but if I read his heart aright between the lines, his joy was mixed with one keen regret. It was that he had failed to capture Islam alive and had him properly tried before a regular session of his court.

%CHAPTER XIX. 

\chapter{YENI MALI AGAIN.}

A range of mountains, whose highest peak is Pellister, separates the Monastir Valley from Lake Prespa. Along a road leading through a gap in this range the Lake Prespa fishermen bring their fish to the Monastir market. It was a hot day late in August; three perspiring men, apparently fishermen, were driving two laden asses along this dusty highway toward the city. I say apparently, for one of the three tramping figures was the author of this narrative, but so well disguised, I assert confidently, that more penetrating eyes than those of the sentry before the block house on the summit would have failed to distinguish me from my two genuine companions.

It was still early in the afternoon when we passed the excise officials at the gates of the city. Thence we continued on down the hot, narrow streets and so came to the center of the city, driving our beasts slowly and laboriously down the main thoroughfare to the neighborhood of the meat markets. Here we entered a hahn, gave the two animals into the care of the stable boy and carried our fish around to the fish stalls in an adjoining alley. Before they had been profitably disposed of, I had taken up three good sized perch on a string and slipped out of the crowd, back into the streets.

I did not feel so confident of myself as I may have looked. For the first time I was alone, dependent on my own wit and resources in dealing with the uncertainties of my situation. For reasons that will develop, Sandy had remained behind with Krusty and his cheta.

My arrival was unofficial. First, Krusty had some time before severed his relations with the provincial committee on the ground that until Ressen could send a properly elected delegate to Monastir, the rayon remained unrepresented. Instead of at once authorizing an election, the provincial committee declared Krusty in mutiny against the laws of the organization and under sentence of death. But as yet no one had ventured to execute the sentence.

Furthermore, from information we had gathered, there now remained some doubt as to the present existence of a provincial committee. Three of its members, including the Eagle, had fled beyond Turkish territory, and Legusheff was supposed to be in prison. There still remained one of whom I knew, a Jewish money lender, but to him I did not propose to trust myself.

Following the memorized directions given me by the village schoolmaster, on whose information we had acted, I crossed the busy portion of the city and so came to a family house in the suburbs. It was the home of the schoolmaster’s parents, who received me in accordance with instructions from their son and had ready for me a suit of conventional cut to replace my peasant costume. Of local revolutionary politics the two old people knew little, so their house served me only as a temporary refuge. I did learn, though, that while many of my old acquaintances had fled, and some were in prison, the municipal committee remained intact, with Moses still in office. Georgie I saw passing through the street from my window one day, so at least one active instrument of the provincial committee remained. I finally decided to make my presence known to Moses, and wrote him a note.
He came that same afternoon, greeting me as effusively as ever, panting and perspiring profusely from his walk. I decided then, and have ever since believed, that whatever treachery there may have been in Petsov’s death, he had no part in it. 

That evening I shifted my lodgings to Yeni Mahli. It felt good to catch glimpses of familiar faces as I passed down the winding streets, some lighting up with pleasant recognition. There was Spiro engaged in an altercation with another urchin in front of his houses. When he saw me he bolted into the gateway, leaving the gate ajar. The assembled family greeted me as an old friend and dragged me down to the supper table at once, where I found an extra place awaiting me, for Moses had forewarned them of my coming. 

For several days I was busy writing and opening communications with Bulgaria, which had been completely ruptured since my last visit. But quite casually I made efforts to inform myself of the local situation. Georgie came once to dine with me, and of his sincerity, at least towards myself, I was also convinced. He had also been arrested, but had been released without even an examination. The Eagle and Legusheff, he told me, had been betrayed by the socialist faction in the city. 

Of Helen T—ff, the young woman who had once made an effort to see me, I found it difficult to learn anything definite. Spiro had the impression that she had gone to Bulgaria to enter the university in Sofia. But Spiro’s brother, the schoolmaster, declared that she was spending the holidays at the home of a friend in a village beyond the suburbs. I could not persist in the matter, for she belonged to the hated “socialist faction/’ and the name still antagonized many who were by no means dishonest. But chance favored me. 

I was busy writing in the room on the upper floor, one day, when an old baba came in and warned us that there was to be a general obisque of Yeni Mahli. The alarm was spread in the household immediately. I quickly gathered together my papers, stuffed them into my inside pockets, and just as a violent pounding began at the gate I leaped over the back wall into the next garden. In all the yards and gardens women and children were running to and fro in nervous excitement. Spiro had somehow preceded me through a hole in the wall, and I followed him through a gateway, passing by a group of alarmed women, and so through garden after garden. Once we were about to emerge into a street, but a woman pushed us back in time to slam the gate against a squad of gendarmes. We bolted in another direction, Spiro trotting ahead with desperate energy. In the last yard down the block we halted in a crowd of women. 

“This way! This way!” somebody called, and I darted through a gateway into a blind alley. A girl grasped me by the sleeve, and we ran up the alleyway, leaving Spiro behind. We crossed a street and came into a continuation of the alley in the next block of houses and so reached an open field beyond. 

“Follow that lane,” said the girl, “and go on till you come out into the fields. Return this way when it’s near dark; I shall be here watching for you.” 

I continued down the lane, and as I came out into the open country I slackened my pace to a leisurely stroll. It was early afternoon and a beautiful, clear summer day. As I got out under the trees and into the tall grass my excitement cooled, and I began enjoying the walk. I came to a brook and turned to follow its bank out into the green plain. So I came finally to the intersecting point of two rubble hedges, and here, under a huge buckeye tree, I laid down. It was deliciously restful to be deep in the grass, listening to the leaves rustling overhead. I think I was dozing when a low murmur of voices began approaching, then the sounds of light footsteps. 

‘Til leave you here,” a woman's voice was saying, “but you may expect me this evening.” 

“Yes, come early; we’ll sleep on the balcony.” 

The answer was also in a woman's voice, and sounded at once familiar to me. I raised on one elbow. One girl was approaching me, slowly and backwards. The other was backing in the other direction. They exchanged final good-byes, then turned. The one nearest gave a visible start as she saw me, for I was now standing and leaning against the hedge. Then she continued onward, looking before her. Once more she gave me a sideward glance, then stopped short. It was obvious that she only then recognized me. 

“What are you doing here?” she exclaimed. “Don't you know that the field patrols pass along this road ? They might ask you questions you can't answer.” 

“The police are too busy in Yeni Mahli just now to come here,” I answered. “How did you know me?” 

“I've seen you often enough,” she said. “I was in Sofia last summer, and there the cafes reach into the middle of the street. I saw you drinking beer with Grueff once, and my brother told me who you were.” 

There was an awkward pause; now that I had found her, I was embarrassed. Finally I said: 

“Somebody told me you had gone to Sofia.” 

“And how do you know who I am?” she demanded quickly. 

“I saw you in -ff’s gateway—from a window,” I blurted out. She flushed, and pressed her lips together. Instantly I realized my clumsy beginning. 

“It wasn’t a social visit,” she retorted stiffly. “Were you afraid of that fellow?” 

“You wouldn’t say so if you had seen the treatment he got after you were gone.” 

“ ‘Perhaps it’s better we didn’t see her/ ” she quoted me, with vicious mockery. “ ‘Who told her we were here, anyway ?’ ” 

It was my turn to flush. 

“Those words,” I replied bitterly, “were intended to shield you, and not to be repeated by idle gossips.” Another pause followed. 

“Come away from the road,” she said finally. “You can’t be left alone here; why didn’t they send somebody with you? Let us walk slowly towards that mill over there.” 

We started sauntering leisurely across the fields, following a faint footpath. She was the first to break the embarrassing silence: 

“You mustn’t be angry with Marika; she didn’t mean it as gossip. It was only when I wanted her to help me in another effort to see you that she repeated your words to discourage me. She was our spy on you, but not to do you harm; only to protect us against mistakes.” 

“But why did you come ?” 

“To arrange a meeting between you and one who desired to talk with you.” 

“Who ?” 

“He’s gone to Bulgaria now; it doesn’t matter.” 

I knew she spoke an untruth. 

“You have been with the chetas?” she added, by way of diversion. 

“Yes.” 

“With whom?” 

“We started to join Petsov, but fortunately we were delayed in a village.” It was a random shot, but it struck home.
“What!” she exclaimed, “they arranged you should join Petsov?” 

“Yes; your visit may have been responsible for it. I was extremely disagreeable to the Eagle that evening, and he may have thought I knew more than I did.” 

“I remember,” she said, meditatively. “Marika told me that you were sour to each other that night. But—why are you back here?” 

“I have some private business in Sofia to arrange, and I couldn’t do it from the rayons. I would have come before, but I distrusted Legusheff.” 

“And you trusted the voyvodas?” 

“Why not? all except the Deacon are honest men.” 

“Were you in Ressen?” 

“Yes—with Krusty.” 

“Krusty! But he was sent by Sarafoff.” 

“Perhaps, but his honesty is stronger than his partisanship. He was condemned to death by Legusheff because he demanded an election according to the new constitution.” 

“And the others are non-partisans?” 

“Absolutely; all those I met—Tashko and Petrush as well.” 

“What a pity,” she said mournfully, “we never knew. God! If only I had seen you!” I almost thought tears would start from her eyes. 

“Is it too late now?” I asked. 

“Too late? Yes, almost. The organization here is wrecked, our best men are murdered or refugees abroad; we shall never recover.” 

We had reached a grove of trees and now seated ourselves in their shade in the luxuriant grass. 

“Comrade,” she said, suddenly, “I feel intuitively that I can trust you. At least, I risk only myself. What I meant to tell you last May I shall tell you now, and even if you publish it broadcast some day, it can be no shame to Macedonia; we had honest men. 

“Everything has happened within the last eighteen months, since the uprising. I was with my brother during the fighting, helping in the field hospital, but when our forces were broken up into flying chetas, I came down into the city, and here I have been ever since. I am not telling you anything from hearsay. 

“The city was full of women and children from the burned villages, but all the best men, who had organized the committees here, were either refugees in the mountains or exiles abroad. The disorganization was complete; it meant beginning all over again. 

“A few months later Grueff came into town and began to reorganize. George Sougareff and Boris Sarafoff were doing the same in the rayons, and all three were supposed to be working together, under Grueff's supervision. But there was no material to rebuild with. The new provincial committee was composed of simple, uneducated tradesmen and workingmen, with no executive ability at all. The only intelligent one of the group was Father Mirabeau, but he was seventy-five years old. When Grueff went he left Father Mirabeau president of the provincial committee. 

“Then, through Sarafoff's influence, Peter Legusheff and Black Peter of Debor were appointed on the committee. Black Peter was an innkeeper and a notorious exploiter of the peasants. These two soon controlled the situation. There was nobody to oppose them, for the general amnesty had not yet brought back our men, and we women were not even recognized by the new organization. Among the other members of the provincial committee there were honest men, but they were weak and incompetent, nor did they ever understand the situation. One by one Legusheff and Black George began to oust them out, and next Israel the Jew and his brother Gabriel joined. They were both money lenders, and in return for their services and the donations they were able to collect from their wealthy co-religionists they were given a monopoly in fleecing the poor peasants whose houses had been burned during the insurrection. 

“Then the rayon, or municipal, committee was organized, and Moses was made its head. As far as we know, Moses is pecuniarily honest, but his tremendous ambitions would well harmonize with the schemes of the Sarafoffists. At any rate, Legusheff had him completely under his influence. 

“Under pretext of fighting the spies of the Greek Church, they next organized a group of terrorists and made Georgie their chief. Georgie was an intense admirer of Sarafoff, and as Sarfof's lieutenant Legusheff dominated Georgie. With the terrorists behind them the committee became an autocratic power. 

“A year ago, when the gymnasium opened, the Eagle appeared, appointed by the Bulgarian Church, which is subservient, of course, to Sarafoff’s master. You can imagine that when a young Bulgar with a finished European education leaves a promising career in the Bulgarian diplomatic service to become a sub-professor in a second rate gymnasium, there are political reasons behind it. The provincial committee opened up to receive the Eagle as delegate from Lerinsko, though neither he knew Lerinsko nor Lerinsko knew him. Here you have the controlling members of our provincial committee, the group which governs Monastir, the most important province of revolutionary Macedonia. 

“Their next work was to gain full control of the rayons. We know that Sougareff was their chief obstacle, but which of the voyvodas Sougareff appointed and which were the creatures of Sarafoff, you know better than we did, for at that time the exiles were only beginning to return. We know that Petsov was sent into the city by Sougareff to confer with the committee, and that they quarreled. We have no documentary proof, but we are convinced that Sougareff was persuaded to come down into the plain on the pretext of a conference and that Legusheff had him betrayed, knowing that he would never be taken alive. We are also convinced that Sougareff escaped from the massacres, though badly wounded, that he came into the city for medical treatment, never suspecting treachery, and that he was here murdered. Neither his body nor his gun were ever found. 

“By this time some of the exiles had returned. One of them was a young teacher, Atanas, a socialist, and naturally against Sarafoff and his imperial schemes. We students in the gymnasium were properly ripe to follow Atanase’s influence, and we organized in a body to fight the committee. Nearly all the teachers and some of the honest tradesmen and workingmen joined us. When the new constitution was proclaimed we had ground to stand on. 

“At this time Father Mirabeau, whom we believed an honest man, either began realizing the truth or thought the others were going too far. Undoubtedly they were finding him an obstacle, for he was mysteriously betrayed, and exiled to the villages. 

“At first we openly demanded an election. The committee replied that the peasants were not enough recovered from the disasters of the insurrection to choose proper leaders. We demanded that copies of the constitution be distributed, but the committee said they had none yet, though we knew they had burned a bundle sent them by the Central Committee, for we found fragments of the pamphlets among the ashes of the stove in the basement of the gymnasium 

“Finally a municipal election was promised us. It was held in the basement of the gymnasium, but when all the votes had been deposited, they refused to let our watchers assist in the counting. One of our watchers, Vladimir, a young student, protested, and when they tried to eject him from the room he struck Moses and precipitated a fight. In the scuffle the ballot box was upset, and Vladi grasped a handful of ballots and ran out. When we examined them later we found most of them mutiliated. Next day the provincial committee declared the old municipal committee re-elected, but of course we refused to accept such a fraud.
Some days afterwards the police raided the gymnasium. After searching the building, they examined the stove and there they found the ashes of burned papers. You know how the ink remains visible on the ash of good writing paper? The police gathered together enough of the ashes to recognize a letter addressed to the Central Committee. You know that the constitution says that nobody, unless a publicly avowed enemy of the organization, can be condemned to death by any lesser body than the Central Committee.

“Some days after the search Atanas and Vladi were summoned by the police and asked to tell what they knew of the revolutionary organization in the city. Of course, they denied any knowledge at all. “‘You are foolish to hide them,’” said the chief of police, ‘they have condemned you to death.’ And he showed them the ash of the letter to the Central Committee, demanding a death judgment against certain persons. Only two names were legible; those of Atanas and Vladi.

“It was shortly after this that the Eagle went to Salonica, where three members of the Central Committee happened to be at the time. There he tried to persuade them to sentence our comrades to death, but, if what we learned later is true, the Central Committee declined unless the accused could be fairly tried.

“Meanwhile Legusheff was engaged in an intrigue all his own. A prosperous young peasant, who had been prominent in the uprising, was hiding in the city. He had a pretty wife who lived here legally as a refugee from the burned villages. One day Legusheff came into her house and attempted to embrace her, but she cried out and neighbors came in. Before the woman could see her husband, Legusheff rushed off to his hiding place and told him to make his escape to the hills, for the police were after him. He gave the man a letter to the Deacon in Ressen, and the young fellow left the city without seeing his wife. But out on the road he was caught in a rain storm and the letter in his jacket became wet. In taking it out, meaning to put it into his cap, he found the gum of the envelope had loosened, so he took the letter out and read it. It commanded the voyvoda to “kill this man when he reaches you, he is dangerous to us.”

“The peasant returned to town and found shelter in the house of a friend without the knowledge of the committee. When he secretly visited his wife he learned of Legusheff’s attempt on her, and saw through it all. He also chanced to meet an old comrade in arms, a former secretary of Sougareff, who was in town for his health. To this chetnik, Stoyan, the peasant, confided his whole story, and from Stoyan we learned it. He brought us the letter to the Deacon, and we kept it as future evidence when we should denounce Legusheff to the Central Committee.

“At this time you were in town; had it not been for Tashko’s girls we should never have known. We lost trace of you for some weeks, but one day Marika told us you had come to her house. The committee did not suspect her connection with us. I had seen an article of yours in a Bulgarian paper which convinced me you were a socialist, beside which I knew you were intimate with Grueff. From this I believed you could be no friend of the Sarafoff brood and their schemes for Bulgarian czar making. I told my comrades, and we decided that if you were going to pass through the rayons, coming in intimate touch with the voyvodas, it was desirable that you should know the local situation, for with a thorough understanding between a few of the honest voyvodas and ourselves, we should soon have settled accounts with the committee. We dared send no man of our group, for the letter Legusheff had written to the Deacon led us to believe that many of the voyvodas were with him, especially as we had always thought that the Deacon was not a Sarafoffist. There you have the reason of my visit.

“Meanwhile the peasant whose wife had been insulted remained here hiding. One day Stoyan lent him his Nagant revolver, and that evening he shot Legusheff in Yeni Mali, but unfortunately, not fatally. Stoyan’s complicity was suspected. One day Georgie, we suppose, invited Stoyan to accompany him on a stroll through these fields and stabbed him to death.

“One of our comrades was acquainted with an Albanian clerk in the vali’s office. He met him one day and the Albanian said: “‘You did me a friendly service once. I wish to return it. Look out—there are two traitors in your organization.’ “‘Who are they?’ “‘One is a Jew—the other is the son of a baker.’ Legusheff is the son of a baker. Our comrade asked further proof. “‘I have copies of secret official reports in my pocket,’ continued the Albanian. ‘I will not show them all to you, but you shall see a list of names of those who are in danger.’ And he showed him a list of names; all these of us who had organized against the provincial committee. “‘I know still more,’ continued the Albanian, and he repeated in turn the name and office of each member of the provincial committee.

“Our comrade sought us all out in a hurry. We called a meeting for next afternoon, to decide on quick action. “The next afternoon Atanas failed to appear. We searched the city for two days, and even inquired at the prison, to see if he had been arrested. Then came a man who lives in a house behind the gymnasium. He had a horrible story to tell us. His wife had seen Georgie and another terrorist enter the basement of the gymnasium with Atanas. She crept up to one of the windows and when she looked in, she saw Atanas lying on the floor, his throat cut from ear to ear. Georgie and his companion were digging a hole beside the woodpile. The woman ran away and broke into hysterics inside her house. Her man came and she told him all.

“That afternoon Vladimir and several of the teachers went into the basement. They found the woodpile rearranged and lumps of loose dirt trampled on that part of the floor which is paved with concrete. In one corner stood a spade. They removed the wood and found the earth flooring disturbed. One of them began digging a small hole, and presently the spade brought up the end of a piece of cloth. They could not pull the cloth up, but they all recognized it as part of Atanase’s coat. There the body lies buried yet.

“Immediately we sent one of our comrades to Sofia, one who knew Grueff well. We had heard that Grueff was in Sofia. He succeeded. He had taken with him enough proof to convince others beside Grueff. Evidently Sarafoff warned his friends that the Central Committee knew everything, for suddenly the Eagle, Black Peter and Gabriel disappeared. One of our friends saw them boarding the train to Salonica. Later we heard that they were in Sofia, Legusheff remained, but he gave himself up to the Turks; evidently he feared to go to Bulgaria. You can imagine what his arrest amounts to. The man who was too cowardly to fight in the insurrection hasn’t the courage to face the agents of the Central Committee if they should come.

“At the same time most of our people were arrested; Vladi is even now in hiding, and I have found it expedient to be seen little in the streets.”

Here Helen ended her narrative. Of my questions and her answers I need give no record. We had risen and were returning to the city. The girl was waiting for us in the lane, but still we lingered a moment.

"At least," I said, "the committee is crushed."

"You think so," she replied. "No, it still exists, but its executive chamber is now in a Turkish prison. Moses still exists, to obey Legusheff’s orders, and still worse, Georgie, Legusheff’s chief instrument, is free and immune from arrest. You are safe, for if you were harmed, there would be no place on this earth to shelter Legusheff; the foreign consuls at least suspect the true situation. And to betray you to the authorities would only make your evidence public. But the rest of us can only wait, either for the coming of the agents of the Central Committee, with full power to remove Moses and the Jew and Georgie, or, we must remove Georgie; in him lies the remaining power of the committee. Without the terrorists they would be helpless."

We parted in the lane. I saw that every nerve of her none too robust body had suffered from the recital of her grisly story. But before we bade each other good-night, we had arranged another meeting.

%CHAPTER XX.

\chapter{A PICNIC WITH A CLIMAX}

The police raid into Yeni Mahli had been an affair of general principle, not directed against any special person. But I made it the pretext of changing my quarters to the house where Sandy and I had spent a night and a day, the house of Michael H-'s mother. Baba’s house was considered the safest in Yeni Mahli, for the block was too irregular in shape to be effectually surrounded. But in the adjoining houses were also many families in sympathy with the socialist faction.

It seemed to me that baba’s greeting was more cordial than our previous slight acquaintance warranted, for she had housed a dozen illegals since then and might easily have forgotten me. She was alone now with her daughter, her younger son having gone to Bulgaria.

Moses and Georgie came to see me towards evening, and I must admit that in spite of long association with men who had taken human life, close contact with one who could lure his victim into a basement and there murder him in cold blood, afterwards concealing the body, was decidedly disagreeable. Whether I showed my instinctive aversion to him, I cannot say. It was the last time I ever saw Georgie alive.

Hardly had Moses and Georgie gone and the gate been barred for the night when a side door in the garden wall opened and Helen appeared, followed by three other girls, one of them Sasha, Tashko’s eldest daughter. Ten minutes later appeared two youths, both students. One was Vladimir, he to whom Helen had referred in her narrative. I especially liked Vladi; he was about nineteen with a frank, pleasant face, slight in figure; there was nothing about him to indicate the pugnacity he had shown in the election.

It was a pleasant, quiet evening; a social gathering such as I had not yet experienced in Macedonia. Bulgarian at its best is too limited in vocabulary for clear cut expression in discussing a wide range of subjects, and at times I was struck by the incongruity of the purely Macedonian dialect with the foreign words, French, German and Russian, which my new friends made use of. Sasha appeared to me in a new light; I had never thought her especially brilliant, but now I realized that she was more widely read than the average young woman of her age that you are in the habit of meeting in any country. Outside of regular school work they had organized circles for the study of literature and political economy. In the former subject they inclined toward the writers of realism, Zola, Victor Hugo, Maupassant, and especially the Russian writers, whom they could read in the original. With this tendency nothing was more natural than that their political economy should lead to socialism.

The girls left us early in the night, but Vladi remained, and we talked almost till dawn. When I awoke late next morning he was gone. But the visits of my new friends were frequently repeated.

Towards the end of the following week I received a note from the Bel Shisma Mahli, in response to which I went there at once and entered the house of my old acquaintance Itso. On the floor were two men in peasant dress, apparently sleeping, but on my entrance they sprang up to greet me. They were Sandy and Krusty. Krusty had shaved off his beard and was changed into a big, good-natured looking cattle merchant. I remained with them that night. Later Vladi and another of the students joined us, and Krusty, the only one of us who had money, stood treat to a magnificent supper and so much beer that the latter part of the night was quite hilarious.

When I returned to Yeni Mahli next morning, I could not even take Sandy with me, for I had come into town with the ostensible purpose of traveling down to Salonica by rail on a false passport, thence by sea to Bulgaria, an undertaking too risky for two at a time. Sandy and Krusty had come into the city unknown to Moses and Georgie.

By this time I had not only learned that the Deacon was still in town, but what house he sheltered in. I was especially anxious that he should not know of Krusty’s presence in town, for he was quite capable of doing him a mischief. But, just how, none of us ever knew, he did learn that Krusty was in town about a week later, and next day he fled, again in priest’s robes, nor did I ever hear of him again until I arrived in Bulgaria long afterwards and found him enjoying Sarafoff's protection in Sofia.

The days passed quickly; I was busy with my work or else engaged with photography. In the evenings I was rarely alone; indeed, I still remember those gatherings of virile minded young people with reminiscent pleasure.

Often, in the afternoons, Vladi and I, and sometimes the girls, strolled out into the fields. In this there was comparatively little danger, for at that pleasant time of the year the townspeople came out in great numbers. I had been in town almost three weeks when an outing was arranged for the following Saturday; we were to take our lunches with us and spend the entire day under the trees lining the stream flowing down the flanks of Pellister.

Early Saturday morning Sasha came for me; she was the only one of the socialists who could walk the streets quite openly. Helen and Vladi and the others who lived in the neighborhood must cross the town by circuitous routes, through the narrow side streets and alleys. Sasha carried a basket over her arm and with the blond braids down her back she looked innocent enough to deceive the most penetrating police official.

As I have before mentioned, it was understood that an illegal must never be recognized in the streets by the local members. When guiding they walked ahead, the outlaw walking behind at some distance. As we reached the main street I slackened my pace to allow Sasha to go ahead. We walked on so for some distance, when suddenly she swung about and took my arm.

"It’s cowardly," she said, "and I am not afraid." Her trembling hold on my arm belied her words.

"Don’t be foolish," I replied. "Go ahead; everybody does it."

"I am not afraid," she repeated.

"I shan’t go out with you again, Sasha," I said. She laughed, but we continued on together. At the next corner she stopped and bought a large pretzel from a vendor. She broke the pretzel in two halves and gave me one.

"Now," she laughed, "we look idiotic enough to be safe." And we walked the full length of the main street eating pretzel and giggling.
Out beyond the city prison we came into the open fields and ascended the incline towards the hills. Helen and Vladi and the others were there before us, over a dozen in all. Many of the girls and the young fellows I had never met before, but they were all delightful company. The location was even pleasanter than the fields outside Yeni Mahli; here were trees and brushwood in plenty with grassy glades to roam about in.

We were not a noisy crowd; some had even brought books to read. At noon we ate lunch in the grass beside the small torrent of clear mountain water, not yet contaminated by the city refuse and the dogs. That all belonged to the socialist faction was manifest from the general discussion of the incidents with which Helen had already made me acquainted. There was one, a graduate from the gymnasium that year, who had just returned from a visit to Sofia, and he spoke of having seen two members of the Central Committee and explained the true situation. He had been assured that serious measures were being taken to purge Monastir of the Legusheffists.

In the afternoon we became merrier and broke into a folk song and later we joined hands for the horo, accompanying the dancing with singing. Several times I saw strolling passersby on the highway below glance up at us, and some even waved their hands. Towards evening we gathered in a conversational circle in the grass below a huge oak. Some merely listened, lying on their backs watching the sky through the green boughs.

“I thought I heard a shot down in the town,” said somebody. We all looked down toward the distant red tiles, but heard only the rumbling of a cart. Half an hour later we had broken up into strolling groups of twos and threes. I missed Helen. Suddenly I saw her and Vladi approaching in the distance along a path parallel with the hill side and screened from below by trees. Further behind them hurried two peasants.

Helen and Vladi stopped, then beckoned to us violently. The other groups paid no heed, but my two companions and I hurried forward. The two peasants had diverged into the brushwood. Vladi carried two bundles done up in kerchiefs.

“Hurry!” cried Helen, “get into these peasant clothes; Georgie was shot and killed in front of his house half an hour ago. The streets are swarming with police, and every unknown person is being arrested.” Vladi and I dived into the underbrush.

“Dead?” I repeated.

“Dead as a stone,” replied Vladi, “shot between the eyes.”

We changed hurriedly, and came down to the path again where Helen and Sasha and another stood watching the lower fringe of the suburban houses. There was a quick and silent exchange of handgrips, then we two turned and made for uphill. On a by path above stood the two peasants, waiting for us: Sandy and Krusty.

“Hurry,” cried Krusty, “we must reach the cheta before morning.”

%CHAPTER XXI. 

\chapter{“ASKER!”}

With our flight from the city, early in September, ended my association with the affairs of the Monastir provincial organization and its devious intrigues. What happened later I know only from hearsay, some authentic, some mere rumor. A few days after our departure Helen and one other of the girls took train on false passports for Bulgaria, disguised and veiled as Turkish women. They arrived safely in Philipopolis, where I met them the following winter, preparing for the university. Others of the students joined the chetas.

A month after two big chetas, one under command of Michael H-ff, arrived in Monastir vilayet. Ten husky young men entered the city, and with the help of local members, swept out the few remaining partisans of Legusheff. Moses, too was obliged to flee from a power between which and himself his American Presbyterian friends could not intervene.

Legusheff was brought to trial, and to everybody’s astonishment the Turkish procuror rose and dramatically denounced him as a traitor who served both sides and was false to each. They sent him into perpetual exile to one of the walled towns of Asia Minor, where he was probably safest. Later he was released and returned to Monastir, where he was killed in the streets as had been his poor tool, Georgie.

A year later, passing through Paris on my way home, I met a Bulgarian friend from Sofia, a student of medicine. We made an appointment to dine together several days later. On our second meeting he suddenly exclaimed:

“I met a friend of yours in the Bulgarian Club last night. B-ff; he was professor of mathematics in the Monastir gymnasium last year. He sends his greetings. 5 '

“What is he doing in Paris ?" I asked.

“He is attached to the Bulgarian Diplomatic Agency."

Thus, unlike the villains of fiction, the Eagle ended happily, rewarded by his master, an ornament to a service whose ethics are, to us simple folk, at least bewildering.

A last word for Boris Sarafoff, the hero of these intrigues. As was announced last year under big headlines in even the American papers, his miscalculations lead him to a fate similar to that of his obscure instrument in Monastir, Georgie, the terrorist. At a moment when it seemed that his schemes were about to succeed, and the revolutionary field in Macedonia was about to pass under his control, or rather under the control of his master, “Czar" Ferdinand, a prominent leader in Northern Macedonia ended Sarafoff’s career by shooting him in the doorway of his own home.

In justice to Sarafoff let me add that to me, at least, and I knew him quite intimately, it seems doubtful that he fully knew of the double dealings of his creature, Legusheff. Sarafoff was an adventurer, perhaps even a grafter, but his was not a character to harmonize with such rank treachery as LegushefFs. Sarafoff might have killed a political opponent, but he would never have betrayed him to the Turks. Had he not been corrupted by ambitious Prince Ferdinand it is probable that his undoubted ability and energy would have gained him an honorable place in Macedonia’s history.

Vladi, Sandy and I remained about a fortnight with Krusty and his cheta in Ressen. We witnessed the long delayed election, and would have started south at once toward Vodensko, for the time was near when I should meet Luka again to participate in an affair he had planned before I had left him. He and Apostol intended nothing less than a quick march to the Greek frontier and a night attack on the Greek frontier posts, merely to show that they sought Greek fighting men, and not harmless villagers.

But between us and Kostur, the rayon south of Prespa, lay a wide plain to be traversed, and we must wait till the bright full moon waned.

Early one night we started out; Vladi, Sandy, two chetniks, Matthew and Saave and I; the two chetniks to escort us as far as the Korbus Mountains, dividing Prespa from Kostur. Towards midnight, before moonrise, the five of us arrived in a village near the head of Lake Prespa, and there passed the day in the woods.

There still remained a five hours' march to the mountains, across a plain bordering on the lake. The danger here was not so much from a possible meeting with asker as running into the Turkish peasants whose villages were plentiful hereabouts, and who now, during the harvest, often slept out in the fields.

Early in the evening we emerged from the woods and entered our village. After a long discussion we persuaded ten of the local militia to accompany us, husky young fellows badly armed with Greek Grat rifles. However, they made us a force with which Turkish peasants were not likely to dispute the way.
It was a dark, though clear night; I remember we traversed a flat, soggy moor, the lake a silver streak in the distance, stumbling through pools of stagnant water and sometimes wading broad, shallow streams. Naturally we dared not follow the highway by which the peasants usually travelled, and it happened unfortunately that our militia escort became slightly confused, so that we made more of the distance than we need. It must have been some time past midnight, for the moon began lighting the horizon, when we came to the lake and followed a road along the shore.

As the moon rose we made out, in its brilliant light, that we were just below the Turkish villages. The lake was on our right, the tall bulrushes coming up almost to the highway. To our left, not a mile away, rose an abrupt wall of craggy mountains. But between us and them were the villages. We must pass several miles further down the lake shore before we dared cross the fields.

We had been pushing on furiously for the last hour. Saave, one militiaman and I walked ahead, together, the others tailing along behind. “Let’s rest a few minutes here,” said Saave, indicating an unusually thick growth of low rushes; “we’ll have to cross the stubble presently, and that’s heavy work with your feet wet and heavy with mud.” The word was passed along the line and all threw themselves down into the thick vegetation. Five of us were together in one group, conversing in whispers. The militiaman suddenly raised himself to a kneeling position, then rose to his feet, though still bent almost double. A man behind us rose, then all of us were up, crouching. Some one uttered the short, sharp word: “Asker!” At the same moment I saw figures emerging into the moonlight up the road, not a hundred yards away. With one, swift dash we were all scurrying for the fields.

“Durr! Durr! Durr-r-r-rr!” came in several voices, then a flash and a shot. Several of us whipped out our Nagants and fired at the mass of figures behind us. A blast of rifle firing burst after us, the bullets whining over and about us. On we pelted, the militiamen scattering over the fields, some stumbling, one falling. Saave and I remained together at first. The firing from behind continued, intermingling with shouts.

At one time I seemed to be running alone, then I came up to a rubble hedge, leaped it, and saw figures running ahead. The shooting from the rear suddenly ceased, even the shouts died out. Four or five of us were now running in a loose group, our speed slackening. Already the fields were sloping up toward the base of the crags, now less than half a mile away.

We came to a second rubble hedge, passed through a cattle breach in it and halted, panting. Some were already there, waiting, others came straggling after. “How many were they?” gasped some one. “At least fifty, to judge by the firing,” said Saave. “Then they’ll follow us up.” “Let’s wait for them here—behind these stones.” We stood behind the hedge in a silence broken only by the heavy breathing. A distant calt came from our left. “They’ll cut in behind us here,” suggested Sandy. “There’s a rocky hillock under the palisade yonder,” said a militiaman. “We could stand them off better there. The trail is just above; we can make it, perhaps.”

We started off at a trot across the fields in scattered formation. The shouts from behind broke out again; then followed some desultory shooting. We rushed up the jagged side of the hillock and threw ourselves in among the big stones on top; there was crowded shelter for all. The base of the palisade was only a few hundred yards further on. A straight, bare path cut diagonally up the rock wall, turning into a narrow gorge a hundred yards above.

We lay there panting for several minutes; then I made out a broken line of dark figures advancing across the bare stubble fields. They came in short, quick dashes, but with evident caution, crouching between each spurt. Some were already flattening down into position when we opened a brisk, spasmodic fire. The soldiers gave a sharp response, then, as if mutually recognizing the futility of promiscuous shooting, both sides ceased and silence followed. Off in the distance we heard the barking of dogs in the Turkish villages. Suddenly came a call, words were shouted; I could not make them out.

“What’s that?” remarked Saave. “Listen!” After a space the shout was repeated. “What’s he say?” I asked. “To lay down our guns,” replied Sandy, “and they’ll give us fair treatment. There’s an Italian gendarmerie officer with them—he wants to come up and talk to us.” “Answer them something,” I suggested. “Yes; something about pork,” added Saave. A third time the shout came; but now in Bulgarian. “Lay down your guns, comrades. The Padisha offers you quarter. The European with us guarantees it.” Then Saave raised both hands to his mouth and bawled forth; something too coarse to repeat here, but there was reference to pork grease. There were no further shouts.

I do not believe I realized what was happening when running figures darted up from all over the fields before us, until they converged and came on with a swift rush. I remember wondering why Saave should be lighting a cigarette; I saw his profile lighted by the flame in the hollow of his hands. Then we were all firing, with nervous rapidity. The soldiers rushed on, shouting “Allah! Allah! Allah!” though so hoarsely that I failed at the moment to recognize the significance of the words. Here and there one fell flat, or stumbled and pitched forward as though diving into the earth. With a fierce, simultaneous roar of yells they reached the base of our hillock and surged upward; there seemed at least a hundred of them. And just at that point Saave made a violent movement—a bright, fluttering speck of fire shot outward and downward, and then—the night burst into a blaze and the ground rumbled under us. When my eyes recovered from the flare the mass of the soldiers was broken; figures scurried over the fields in various directions and disappeared. Still we continued firing. Once I grasped my carbine by the barrel and dropped it with a start of pain.

The nervous excitement of those few moments was more exhausting than the run across the fields had been. “Wasn’t that fine!” burst from Saave triumphantly. “They won’t try to rush us again for another hour,” and the whole crowd of us cheered; cheer after cheer, mingled with inarticulate yells of exultation. Saave rose and bawled: “Fly, you Mohammedan swine, we’ve greased your tails with pork fat.” The ascendency which pork seemed to have gained over Saave’s active thoughts tickled me immensely, even at that moment.

Silence was worse than the noise; we had distinctly to restrain the desire for indiscriminate firing. “Look!” exclaimed a peasant He pointed toward the base of the palisade. Among the scrub oaks something moved. “They’re cuting us off from the trail,” swore Saave, “we can’t allow that. We’ve got to do it—rush them now—now, while their nerves are still upset.” Nobody answered, but there was a general movement.

“Pile your bomb into them,” whispered Saave to me. He showed me a lighted cigarette in the hollow of his right hand. They evidently did not divine our intentions until we were half across the intervening space. Saave and I ran ahead together. As the soldiers fired we halted, and he held up the cigarette. I applied the fuse; those behind streamed past us for the trail. Then I threw it, and we ran.
The explosion almost threw me down, but I recovered and raced along with the others, scrambling desperately up the rocky path, stumbling again and again, using my carbine as a staff. Higher up, the trail broadened into a ledge and there we dropped, for breath. Some of the militiamen, not weighed down by the ammunition, continued and disappeared above into the gloom of the gorge. And then the bullets from below began sputtering against the rocks about us.

Another desperate effort, and we hurled ourselves upward. The man ahead of me stumbled, fell and rolled under me, tripping me up. I caught one glimpse of Sandy’s white face, made a violent effort to hold him; then he rolled over the edge and disappeared.

“Run, for God’s sake, run!” I heard Saave’s voice, felt him grasp my arm, but I staggered about helplessly, blinded by rock splinters. I felt myself half shoved, half hauled, onward, completely confused, but still struggling. The spattering against the rocks ceased.

"We’re all right now,” came in Saave’s voice. "Let’s take it easy—God, I am spewing blood.” My eyes smarted painfully, but with sheer muscular effort of the lids I could see enough to guide my steps. The firing below ceased.

When I could see plainly we were climbing a narrow gorge, slowly and haltingly. Half an hour later we crawled up on a level, timbered space and threw ourselves down under the trees. There were only six of us: Matthew, Saave, Vladi, two peasants and myself.

"We’ll get you up to the shepherd’s huts,” said one of the peasants, "the others have taken a northern trail, home. We’ll see you through.” We spent the next day high up on an upper ridge. Then, and not before, came the overpowering mental depression. Some had fallen, others had been wounded, we knew, but I could realize only that Sandy was gone.

%CHAPTER XXII. 

\chapter{BACK TO VODENSKO.}

Between the Little Prespa Lake and Kostur rise the barren Korbus Mountains, presumably the Scarbus of antiquity. Vladi and I crossed this range late in September and came into a country differing much in character from any other part of Macedonia I had yet been in. It seemed almost as barren as the mountains we had crossed; rocky and bald of timber. But the floors of the small valleys were golden with grain and green with vineyards.

Our first two days were spent in a village which was almost a town. Some of the houses were two-storied stone structures with ornamental faqades. The church had been bombarded and partly destroyed during the insurrection, but was again built up with a tower almost like that of a western church. Here the fighting had been fiercest during 1904, yet the recovery had been quickest. Most of the men were stone masons who spent the winter months working at their trade in the bigger cities of the Balkan states. The effects of travel were visible in the active interest the villagers showed in political and general current events. One even produced a pamphlet sent him by a relative in America in which I saw for the first time photographs of the ruins of San Francisco after the earthquake.

We entered the church. About a hundred children were gathered there, all seated on the floor, some reading, others writing and one figuring out an arithmetical problem on a blackboard. Two girls were presiding, both in peasant dress, but they addressed us in the Bulgarian of the gymnasia.

“Why don’t you have a schoolhouse?” I asked, “and why don’t you have benches for the children ?”

“Because this is a karaool school,” explained one of the teachers. She took me to a window and pointed to a nearby hilltop. Two children sat there in the shade of a rock. I understood. They were the karaooli watching. Should a file of asker appear along any of the approaching trails their sharp eyes would perceive it in time. Before the soldiers could arrive in the village there would be no traces of the school, and the two teachers would be churning butter or washing clothes. Kostur is one of those districts whose secession from the Greek church the government has not yet recognized, and is therefore still under the temporal jurisdiction of a Greek bishop. Of course he would forbid schools in which Bulgarian would be spoken, and as the people are pure Bulgars, speaking no Greek, it is natural that they would not tolerate teachers who could not even converse with them. The alternative was no schools at all. Still they had them, and it was for the work of teaching in these karaool schools that such girls as Helen and Sasha were training.

Here in Kostur, as in Voden, the efforts of the Greek Church to suppress the revolutionary movement among the people were tremendous. During the insurrection Greek priests accompanied the Turkish troops on punitive expeditions and pointed out the villages to be burned and the villagers to be executed. For this reason the people were now especially bitter against the Greek Church. In this rayon is the village which gave its name to one of the most notorious incidents of recent Macedonian history, known even to Europeans: Zagoritchini. The priests bribed a Turkish officer to repeatedly search the village, thereby causing the villagers to hide their arm beyond immediate reach. Then a note was sent to the elders of the village, signed “a friend,” warning them that another final search was to be made within a few days which would be unusually thorough. Most of the villagers buried their guns.

On the date fixed all the villagers were at church. When they heard a bugle call and saw a body of armed men approach, they thought only that this was the fulfillment of the warning; that asker had come to search for arms.

Suddenly a volley shattered the windows of the church, killing a dozen of the congregation. The worshippers rushed out in a panic and found themselves surrounded by a band of over a hundred soldiers of the church, some in Greek army uniforms. Three of the leaders were in the uniforms of Greek officers.

In all, sixty of the villagers of Zagoritchini were massacred. Five of these were children under fifteen, seven were women, of whom two were advanced in pregnancy, and twenty-two were over sixty years of age. The insurrection had left few of the houses standing, and of these ten were burned with twenty-eight temporary shelters. All this was done in broad day, near enough to a large garrison of troops for the rifle shots to be heard. All this in the name of Christ, instigated by a church to whose head, in Constantinople, representatives of European nations offer deference and consult when the “will of the people” is to be sensed. “I know not how to indict a whole people,” said Burke, but when a whole people acclaim such a deed as this, through a unanimous press and pulpit, as patriotic and heroic, and glorify its perpetrators with public dinners, what is there left to say of such a people? Not much that will be heard probably, for they are the descendants of Socrates, Pericles and Solon (in name, at least), and the chosen people of Byron.

One evening, we were joined by a cheta and one of the rayon voyvodas, Pando Klasheff. Two chiefs divided the administration of Kostur between them. One was Mitre, the Vlach, for fifteen years an outlaw. He was of Rumanian parentage and had been a simple shepherd, but his undoubted ability and integrity raised him even above the animosity of the Sarafoffists. He realized that his abilities were purely of military value, so he invited Pando Klasheff to share his command with him, that he might help in the administrative work, which Mitre considered most important.
Next to Luka, I have never met with a voyvoda better equipped, intellectually, than Pando Klasheff. In appearance he was a clean-cut, aristocratic young fellow, his speech so pure that it approached Russian. He was dressed in simple, close-fitting uniform; nothing except his canvass leggings, leather revolver holster and ammunition belt indicated his vocation. He had studied in the University of Sofia, and had, I believe, his degree of B. A. With him was his lieutenant, Kartchakoff, slightly younger, and also a college boy. No mere fighting men were they; they had definite theories of economic organization. They had established co-operative stores and flour mills, and the karaool schools were of their making. 

When we met they had just executed an affair which will probably give material for the folk legends of generations to come. There lived in the village of Pisoder a Greek priest who ranked next to the bishop, but was even his superior in planning mischief. This Father Stavre was the organizer of the system of espionage which supplied the Church with such accurate information regarding the revolutionary agitators. He first distinguished himself in the insurrection by the energetic help he gave the Turks in seeking out the members of the local committees. Twice he participated in a massacre. Once a Turkish officer refused to burn a certain village at his suggestion; then Father Stavre incited Albanian raiders to do it. The peasants tell that once an old Turkish major refused the father’s company, publicly declaring before his soldiers that “we can disgrace ourselves without the help of this savage.”

Then he endeared himself to the Church by having Lazar Poptrikov betrayed and murdered, one of the brainiest of the first organizers. He brought the head to Kostur and had it photographed, distributing the prints among the peasants. But his master work was the massacre of Zagoritchini.

Now, Father Stavre directed the operations of the terrorist bands. His location was exactly suited to this purpose, for Pisoder was a garrisoned village in the center of a district which was to be evangelized. 

Curiously enough, this monster was a man of superior education and charming manners. He could hypnotize foreign journalists and his power over women was tremendous. For this reason the Church often sent him abroad on diplomatic missions. He was head of the delegation “from the Macedonian people,” which was sent to meet the English King at the Olympic games. During the previous summer he had made a tour of Europe, engaging the sympathies of diplomats and men of affairs for the cause of Greek expansion. Two revolutionary terrorists dogged his steps all over the Continent, but he was too cunning to be caught unguarded. 

One day a peasant woman found a letter on a mountain trail, evidently dropped by a passing courier. She carried it to the cheta. Pando Klasheff opened it and recognized it as a letter in Greek cipher. In his boyhood he had been forced to learn Greek. He began carefully studying it. Appended was a list of several names which Klasheff recognized as those of Turkish gendarmerie officers. The names were in ordinary writing, but opposite each were words in cipher. Working on the rough guess that these might be the officers' stations, he began deciphering, and by this chance discovered the key to the cipher. 

Though anonymous, the letter was evidently from the Greek bishop, and was addressed to Captain Vardas, commander-in-chief in the field. It unfolded a plan for the burning of Boof, a large Bulgar village partly depopulated by emigration to America. The writer then directed Captain Vardas to send a certain Captain Bellus to consult with Father Stavre, who would give the final instructions. 

Klasheff held a consultation with Mitre; they called together all the chetas in the rayon and went to the mountain overlooking Pisoder. Among the minor chiefs was a graduate of the Greek gymnasium in Athens, who, of course, spoke faultless Greek. He was chosen as chief actor. 

A few months before the cheta had taken prisoners several members of a Greek band. They were disarmed and sent to Greece in peasant clothes, their uniforms being kept as trophies. Kouze, the sub-chief, and six of the chetniks dressed themselves in the Greek uniforms. Then a letter was written in the Greek cipher and sent down into Pisoder by a peasant. It requested an interview with Father Stavre on the “Boof business” and was signed “Captain Bellus.”

Several hours later the cheta, from its hiding in the forest, saw two men climbing the mountain from the village. One was in priest’s cassock, the other in the uniform of a Turkish gendarme. When these two arrived in a clearing half way up the mountain, they met a band of seven men in Greek uniform, one of whom stepped forward and introduced himself as Captain Bellus. 

An hour later, when the garrison in Pisoder hurried up the mountain to investigate into the cause of a number of rifle shots, they found Father Stavre and his companion—both dead. 

I regretted not being able to meet Mitre, who had just gone to Monastir for medical treatment. With Klasheff and his cheta, Vladi and I traversed Kostur to Lerinsko, and there parted from them, to continue our way to Vodensko. Several months later all except one of the sub chiefs were killed. But if you should ever come across one of the many little coffee houses Bulgar immigrants have established in America, you will hardly fail to discover a portrait of Pando Klasheff, and sometimes of Kartchakoff, against the wall. One morning we entered the same village in which Sandy and I had first met Tanne, and there I met him again. We spent a whole evening together, and he told me there had been a fight in Vodensko in which Luka’s hand had been wounded. The Greeks had ambuscaded the cheta, but not with much success. 

We hurried on again, traversed Lerinsko, and one morning arrived in the village just below the mountain where I had parted from Luka. The villagers recognized me, and extended a cordial greeting. 

“Where’s Luka?” I asked the president of the local committee. 

“I don’t know,” he replied. “I believe he’s in the swamp.”

“What’s all this about the Greeks ambuscading the cheta? Is it true?” His eyes glistened. 

“It’s true. I helped bury the five Greeks who stayed behind; one was the captain.”

“Were any of our boys hurt?”

“One chetnik—Yani—a Vlach—he was killed. They say Luka was wounded in the shoulder. It happened just above here. The cheta had come to the village. Late that night they went up to the shepherds’ huts. It was bright moonlight. Perhaps the outpost was asleep, but just before dawn the Greeks crawled up to a timbered crest above the huts. They must have crept down on them by the gorge, for our boys didn’t know anything until a volley of Grats ripped through the huts. Out they rushed and began firing, but the Greeks didn’t stay to fight it out. They found Yani inside the hut; he never knew what happened. We sent a pony for Luka; I saw him sitting up against a tree looking very pale and wrapped in a cloak.”

“But where is he now?” I persisted. 

“I don’t know, I tell you. Teodor was in S-v two days ago with ten men; he may be there yet.”

I could not wait till night; early in the afternoon Vladi and I set out over the mountains. At midnight we reached S-v, and leaving Vladi in the forest to wait until I returned, I dived into the village unannounced. Several men rushed out of a house and disarmed me at once. Then they recognized me and gave me back my gun. 

“Where’s Teodor?” I began.
"He’s up toward Rodivo." 

"Do you know where Luka is?" 

"Teodor says he started for the frontier, to get munitions." 

Two men offered to take me to the cheta and, picking Vladi up, we struck a trail up into the mountains again. In an hour we arrived at the edge of a clearing in the center of which was a sheep corral. One of the two men with us whistled; immediately came a response from inside the corral. I made for the entrance; inside the enclosure was a charcoal fire about which lay half a dozen men wrapped in cloaks. Several were raising themselves on their elbows, as if just awakening. 

"Teodor!” I called. A lithe figure sprang up, and next instant Teodor and I were wildly embracing. 

"Where’s Luka?” I demanded, when the greeting was over. For a moment Teodor did not answer. 

"He was wounded,” he said, “in the shoulder and through the abdomen. We tried to carry him away, but-” 

"Where is he now?” 

"He lived just eight hours, then he died, believing it was asker.”

%CHAPTER XXIII. 

\chapter{A ROYAL EMISSARY.}

In the extensive literature on the subject it seems to me there is one phase of actual warfare which has been slighted, or, at best, only lightly touched. Men banded together for fighting purposes, especially if on the defensive, are keenly susceptible to a mutual affection quite as intense as sex love. Perhaps it is only a natural instinct whose purpose is to facilitate close organization, but there it is, a strong, emotional bond whose sudden rupture is one of the sharpest pangs I have ever known. 

Luka’s death had aged Teodor, if not in appearance, at least, mentally. I am confident it caused him more suffering than did the bastinado treatment applied to him by the Salonica police. His old time sense of humor was there, but with a note of bitterness in it. He had always thought warfare a silly survival of savagery, but now, I know, he detested it thoroughly. In the week or two we remained together we were more closely drawn to each other than during the two months’ intimacy in the swamps. Vodensko did indeed seem gloomy to me then, as I know it did to him. The wanton waste of it all seemed most painful. Here, cut down in its prime by a band of ignorant peasant hirelings, had been destroyed a life which, in more favorable environment, would have been a powerful factor in advancing the ideals of real civilization. And yet it was only a typical incident of the greater waste of intelligent young lives over all the country, brought nearer to us by intimate contact in one personality, or, in my case, two, for Teodor had never known Sandy intimately. 

Teodor was anxious that I should remain in Vodensko, and even offered to resign his supreme command of the rayon in my favor, an act which he assured me would be ratified by the electorate. I never, till then, realized his deep regard for me. Had it been the crown of an empire that he offered me, the honor he felt attached to it, and his own self-effacement could have been no greater than in this proposal that I should be Luka’s successor. But Sandy’s death, especially, had set me thinking over the many questions and theories we two had discussed together, and I felt no desire to participate longer in this futile game. The conviction was coming over me that armed force had ceased to be a factor in human progress; that more potent agencies ruled the affairs of men. Certainly armed force and justice seemed never on the same side. I left Vodensko, and had there been an able leader to take his place, I feel certain that Teodor would have come with me. 

After I began the writing of this narrative came a letter to me bearing Teodor’s name to be added to the long list of my old comrades who have fallen in the wanton struggle. One of his own men, discovered in pilfering a peasant, murdered him, then established himself with several others as a brigand of the old fashion. It was Luka’s brother, a quiet merchant in Bulgaria, who gathered about him a score of the Committee’s chetniks, traversed Macedonia and quickly avenged Teodor’s murder. Then he returned to Bulgaria to resume his commercial pursuits, just as the Macedonian constitution was declared. 

I had a vague plan of getting Vladi down to Salonica that he might be smuggled through to some steamer bound for Black Sea ports, to enable him to reach Sofia in time for the beginning of the university term. This brought us down to the swamps and to my old friend Apostol, who had extended his domains several kilometers further east by occupying another island. We learned then that a series of arrests in Salonica had broken up the provincial committee, so our scheme was not feasible. But I was glad to spend a week with Apostol and his motely collection of fighting men and refugees. 

It was then that occurred an incident which I consider one, if not the most, significant of those I observed in all my Macedonian experience. 

As I have stated before, Apostol was a talkative old gossip, and on our first day together we lay out in the sun talking over old times and relating our several experiences since then. He had “precipitated” several affairs and I listened; only lazily attentive. Gradually my interest was becoming keenly excited. 

"What’s that?” I interrupted. “I don’t quite catch it—tell me again.” 

"Why, he’s a big gun from Stamboul, cousin to the heir apparent—of the Imperial Family—the real Blood. I got his letter through one of our local beys. ‘In the name of His Imperial Majesty the Padisha, I wish to confer with you. I will meet you wherever you wish—in the mountains, or in the swamp.’ That’s what he says. Think of it! The Europeans call me a brigand—but a member of the Imperial Family wants to meet me—trusts me enough to come out here alone. What you think of that, eh ? He doesn’t think me a brigand.” 

"And what did you answer him?” I demanded, sitting up. 

"I haven’t answered—what for? He only wants to bribe me—like they tried once through my wife. What’s the use of talking with him? Then some of the comrades in other rayons will say, ‘there’s Apostol showing off—thinks he’s the Big Chief.’ It might even hurt my reputation.” 

"Apostol,” I said, solemnly, “we must see this royal Turk. I am here to vouch for your reputation. We must meet him—let him come—he’s excellent copy for my correspondence if nothing else. Think of that—to write that a delegate from the Sultan meets you in your own stronghold is the best testimony in the world to what you and the chetas really are.” 

We awakened Vladi from a nap to write a letter at once; he knew Turkish fairly well. The missive was sent off that evening to the bey who had acted as intermediary. Then we waited—two days. 

A courier arrived with a letter; the inscription on the envelope was in Turkish. We tore it open, and Vladi translated: 

"To the Voyvoda Apostol, chief of chetas, greetings from Selianikde Mekteb Hamidis Houssinde Hussein Konainda; Sheik Achmed Kemal Bey; I shall arrive at the appointed place to-morrow, to enjoy the supreme pleasure of an evening's conversation with you and your associates.” 

How much of that inscription is name, how much title, I do not know; I suspect it even includes an address. Just so I have it down in my diary, written by the honourable Sheik Kemal himself and translated underneath by Vladi. 

Next morning a suppressed excitement possessed us. We had sent two chetniks in a boat to the landing, and then we waited, listening for the challenge of a distant outpost. Even Apostol was infected by my spirit.

Shortly past noon we heard the challenge. Ten minutes later the boat shot into the lake and approached our landing. In the bow sat a man in Norfolk shooting jacket, an English fowling piece across his knees. His tall fez was escaped in a neat fitting muslin covering which dropped down over his shoulders, protecting his neck from the sun. Beside him was a beautiful Irish setter. 

As the boat touched he rose and stepped ashore. He brushed aside the sunshade and revealed a youthful face; a reddish beardlet on either side of his pointed chin, light brown hair, high, though not prominent, cheekbones and bright hazel eyes that smiled in unison with a faint wrinkle on either side of his small mouth. It was a pleasant face, though strongly Ottoman. My lasting impression is of the humorous twinkles in the hazel eyes. 

The rest of us remained in the background while Apostol, with a dignity I had never suspected in him, advanced and solemnly exchanged the kiss of greeting. Then Vladi, Apostol's lieutenant, and I were introduced and similarly greeted. We adjourned into the big hut, the chetniks saluting the visitor, Turkish fashion, as we passed. We seated ourselves on the rug of state; a chetnik immediately served guest coffee, and the Sheik offered us each a present of sweets. 

To one who has had experience of Turkish psychology it is unnecessary to say that the object of the Sheik’s visit was not referred to that day. We conversed on general matters, one of the most interesting discussions I have ever enjoyed. The Sheik spoke no Bulgarian, and a little French. I spoke neither French nor Turkish, but Vladi and Apostol both spoke Turkish. The conversation was therefore in that language, but I lost little by it. Apostol had a way of referring to me and insisting on the translation of every word spoken, for my benefit, which probably impressed the Sheik with my rank in the organization. But Vladi was a clever interpreter. Occasionally the Sheik added something in French which I understood, though I could not frame an answer. 

To judge by his words, Kemal Bey was by no means a reactionary, though he produced royal credentials. I suspected even then that he must be a Young Turk. 

“Why take up arms?” he argued. “Armed revolution is obsolete, futile. Do like the Young Turks; accept things as they are for the time being, but agitate. Join them, and make the progressive movement a solid force. . . . The majority, Turks as well as Christians, are progressive, but the majority are for peaceful means. . . . You are destructive, wasteful—why not create?” 

At times the argument became warm, but never bitter. Apostol was the most decorous. Vladi and I were more Western and even the Sheik became animated. But just at those moments when he seemed about to utter some impulsive exclamation, the humorous smile expanded, and he passed off some pleasantry that set us all laughing. He was a charming talker. 

I believe he suspected who I was early in the day, but he asked no personal questions. Apostol had simply introduced me as a comrade. Finally Vladi told him. His interest was manifest, but he showed no curiosity regarding my motives. 

So we talked till late that night. We showed him around the farm, the beehives, the chicken and turkey run, the duck pond and the sheep enclosure, and he expressed opinions on practical farming. 

Near midnight we retired; the four of us stretched out together on the rug before the fire. Just before dawn I was awakened by Apostol sitting up beside me. It was chilly. The Sheik slept peacefully beside us, curled up under a cloak. Apostol’s eye caught mine. 

“He isn’t used to this,” he whispered. He rose, spread his cloak over the young Turk and crawled under my covering with me. 

After coffee next morning we went out into the open. I suggested to Vladi that we leave the two together to discuss business, but both the Sheik and Apostol insisted that we be present. Then the object of the visit was revealed to us. 

It was that Apostol should come to Constantinople to be the guest of the Sultan for one month. His Majesty recognized in Apostol a man of wonderful military and administrative ability, and a leader of the people. His Majesty desired to learn in detail the desires of the people directly from their leader. During the visit all asker would be withdrawn from Apostol’s rayon, and he might communicate with his sub-chiefs daily by his own special couriers. Apostol would be received and entertained with royal honors. 

I do not believe the Sheik was greatly surprised when Apostol, after silently listening through the entire proposal, decidedly refused it. He probably knew the situation better than the Sultan. Though I were to read ten books to the contrary, each by a noted diplomat, I shall always believe that the much vaunted knowledge of Abdul Hamid regarding his own empire is untrue. His grasp of the revolutionary movement must have been hazy at least when he thought Apostol an important factor in it. 

The Sheik then asked that Apostol affirm his refusal in writing, a letter addressed to His Majesty, acknowledging that the Sheik had performed his mission. Apostol did so, and to the letter was added “if all Turkish officials were like Sheik Kemal there would be fewer difficulties between the people and their sovereign.” 

It was then that I showed my kodak to the Sheik, and he became at once excited. 

“Photograph Apostol and me together!” he exclaimed. “That will be to me a valuable souvenir of my visit.” I did so, the two together, sitting arm in arm. Then I discovered that I had not enough developing fluid. The Sheik seemed so downcast that I said: 

“I'll give you the undeveloped negative, but on the condition that you send me a print. Here is my address in Bulgaria.” He promised. I then wrapped the negative in red paper and gave it to him. 

He went early in the afternoon. “If you should ever come to Constantinople,” he said to me, “look me up. Yes, I know you are outlawed,—but I am not a police official. I would not present you at court, but I would entertain you royally in my house.” 

As he disappeared into the rushes in the boat, the chetniks gave a cheer. He rose, touched his breast, mouth and forehead; then suddenly he took off his fez and waved it. 

The following winter, when I arrived in Sofia, I found a letter addressed to me with a Turkish postmark. Inside was a mounted print of the famous negative, not very sharp, but quite plain. Enclosed was a letter which ended with “may God protect you in your various enterprises.” On the back of the photograph was written simply, “Kemal”. 

Later I read in the papers that he was appointed kaimakam of Enedji Vardar caza. Since the recent successes of the Young Turkey Party I have often scanned the papers, but though I have several times seen mention of some of his names, they were never arranged in the order in which they are written in my diary. 

%CHAPTER XXIV. 

\chapter{CONFESSION OF THE MELANCHOLY BRIGAND}

After leaving Apostol we began gradually shaping our course northward. Crossing the River Vardar we came into Kukush, and there, in the extensive cane brakes of Lake Amatov, we met and spent several days with Damion Grueff. He was on a tour arranging the elections of delegates for the coming annual general congress, to be held up close to the Bulgarian frontier. I took, then, a photograph of this prominent leader; I little thought then of the value this portrait would have to Macedonians. A little more than a month later Grueff was killed in a skirmish with asker.

There were just six of us; and for five days we had been dodging military patrols, which is no college sport under a hundred rounds of ammunition, a Manlicher rifle and a twenty-pound, goats’ hair cloak. On the sixth day we came to the lower slopes of the Strumitza Mountains, in Northern Macedonia. We wiped the perspiration out of our eyes as we gazed up at the cool, blue ridges.

“We will go up there,” I said, and my army of five men exclaimed unanimously that I uttered the wisdom of a great general.

But apart from the promise of security, those mountains held in them another object of keen interest to me, for through their forests wandered a man whose anonymous fame once spread over all Europe and America. I longed to meet him and hear the other side of a story which for six months filled the leading columns of American newspapers and many pages of American magazines.

We waited up in a timbered gorge, and on the third day came a spider-legged courier clambering down the rocks, with an answer to my letter. It read: “I shall wait for you below the White Oak Peak. Impress the courier into your service.”

We traveled hard all night, and when a misty morning broke, we were toiling laboriously up a steep mountain side, through dense oak saplings whose leaves had already been turned dark brown by the first frosts of the coming winter. A shrill whistle from above encouraged the file of us to another spurt of effort, and we came up on a level space among a crowd of husky chetniks. There was the usual silent exchange of the kiss of brotherhood, and we all dropped down in a wide circle about a fire. A tall, garrulous man was holding forth; he talked like one who might have ideas to expound.

“Is that Hristo Tchernopeef?” I asked my neighbor, a quiet little man who as yet had not spoken.

“No,” he said, in very clear, crisp Bulgarian, smiling humorously, “I am Hristo Tchernopeef.” I turned on him instantly.

“Then you,” I said, “are—it was you, was it—who kidnapped Miss Stone?” He nodded, with a grim smile.

“It was I,” he admitted. “But don’t condemn me without a hearing. You, as a brother brigand, should be more just. I want you to understand how it was—for them, I don’t care a damn.” He waved his hand toward the frontier; I knew he meant all Europe.

He was a small man with a face which, when in repose, was that of a peasant; straight, brown, wiry hair, cut short, sticking up obstinately; round features, dark grey eyes under heavy eyebrows, and a small sandy mustache. It was, as I say, a peasant’s face, but when he smiled—in that smile was all that was super-peasant in Hristo Tchernopeef.

For two days we talked on other matters; but it was these discussions on the relative merits of Manlicher and Mauser, Evolution versus Revolution, and whether there is a proletariat in the Balkan countries, a question on which good socialists may differ, that begot that degree of intimacy which brought the story on of itself.

We had strolled away from the camp to an open glade on the mountainside from which we could clearly see the Rilo peaks, dividing Macedonia from Bulgaria. In the intervening distance lay the theater of the famous incident. And so the subject came up again.

“Yes,” he said, “it was Yani Sandanski and I, with eighteen good, husky lads who did it. God! Who would have thought it was going to last five months. No, I don’t care who knows it now—you might call it the “Confession of a Melancholy Brigand.” They, the editors and the diplomats, will vouch for the brigand, while you have it on my authority that the adjective fits.”

“Yes, it must have been dangerous business,” I said, sympathetically.

“Dangerous!” he repeated, contemptuously. “You don’t understand me. I wasn’t referring to the danger. If you were older, and long married— have you ever found yourself in a position of strong opposition to a middle-aged woman with a determined will, all her own? She assuming the attitude that you are a brute, and you feeling it? Firm opposition, not with physical violence; that would be a relief, hour by hour, day by day-”

“But that isn’t the story,” I objected.

“Perhaps not. I am only creating an atmosphere; that comes from reading the literary supplement of our revolutionary organ. I only wanted to give you the right aspect from my point of view. I want you to understand this,” and he raised his hand to the grey hairs on his temple. He thought a moment, then plunged abruptly into his story.

“It was after the downfall of Sarafoff, and Prince Ferdinand had captured the machinery of our committee of representatives in Sofia by putting General Tsoncheff into it. Tsoncheff, the prince's friend. Of course, we repudiated him. We, in the interior, were not going to recognize as our representative a Bulgarian general appointed by Prince Ferdinand.

“But Tsoncheff not only insisted that he was our representative, but that he should govern the whole Macedonian revolutionary organization. Fancy a German admiral coming over to your United States and declaring himself your prime minister. You would either kick him out, or laugh at him. But Tsoncheff's rank impudence was backed by Ferdinand's gold, which bought men, guns and ammunition. And with the pretence of revolution he began sending big armed bands across the frontier, to oust us out of our rayons.

“Of course, we resisted. But just then happened the Salonica betrayal, and the whole Central Committee and dozens of other able leaders, on whom we depended for the financing of supplies, were arrested and sent into close exile into Asia Minor. The organization collapsed; in all Northern Macedonia only Sandanski and I were left. It was then that Tsoncheff began running his bands across the frontier to conquer the revolutionary field.

“We met them, first with protests, then with armed force. Men we had in plenty, for the population was behind us, but empty handed men aren't much good in such work. There was no Central Committee to assist us, even with advice. And the means of appealing for help to the Macedonian immigrants in Bulgaria was denied us. You see, the Macedonians in Bulgaria hardly knew how things stood, for our revolutionary organ had fallen into the hands of Tsoncheff from Sarafoff. So the people got their version of it, and continued sending in their contributions to Tsoncheff.

“It was a desperate situation. It looked as if we and the whole organization would be swept out of existence and Prince Ferdinand’s hirelings would possess themselves of the field, to do with it as they liked. To add to the aggravation, Tsoncheff hired and sent over an old brigand who had operated in the Rilo Mountains in the early days before the organization had driven him out, old Dontcho, who captured Christians and Turks alike for ransom and kept the money for himself. The people detested him.
"Sandanski and I were together. We were now so poorly equipped that we didn’t even dare to meet Tsoncheff’s bands; we had to run from them, as if they were asker. We needed money. So we determined to capture some wealthy Turk and get a few thousand liras ransom. Once we tried and failed. At that time there came to us a chetnik who had been a student in the American school in Samakov. 'Capture one of the missionaries,' he suggested, ‘and the Turkish government will pay the ransom immediately to avoid complications.'

“The idea took us with fever heat. You understand, it wasn’t pleasant to contemplate—we had never even captured Turks for ransom. But Tsoncheff’s bands were pouring in on us. When we heard that Dr. House was coming across the country, we decided to take him. Dr. House has always been a friend of the peasants; when we heard that he had decided not to come our way, I, for one, only half regretted it.

“A few days later we heard that Miss Stone was in Bansko, and would be traveling south in a few days. Down we rushed to Bansko. I didn’t mind Miss Stone so much. She often preached against us, telling the poor peasants that God would right their troubles, and not the “brigands.” All harmless stuff—nobody took it seriously, but it made the business less difficult for us to gulp down.

“There was a garrison in Bansko, and the villagers couldn’t even get food out to us. But for two days Sandanski and I were in the village, dressed as peasants, watching Miss Stone and arranging plans. It was the villagers who persuaded us not to do it in Bansko; they feared reprisals. The courier who afterwards was guide to the party was our man; he took them to us.

“You will remember how we dropped down on them as they passed, all of us disguised as bashi bazouks, but so famished that we hadn’t the presence of mind to refrain from pork when we tore open the lunch hampers.

“Sandanski and I had decided to take a Bulgar woman with us as Miss Stone’s companion. We really wanted to be as decent to her as was possible. But the elderly woman we had chosen was taken so ill she couldn’t be moved.

“There were a lot of young girls in the party, but we were afraid of the gossips. ‘There’s Mrs. Tsilka,” said the guide. ‘She’s married.’ We liked her looks; she wasn’t too young, and she looked matronly. But if we had known what was coming—the baby, you know—we’d have taken our chances with an unmarried woman. Or we’d have done without a chaperon. We paid heavily for conventions.”

“How about the Turk you killed?” I put in.

“Oh, the one we used for making an “impression?” he replied with sarcastic bitterness. “To make them realize that we meant business? The papers had it that way. Took an innocent life to create an artistic effect. No. We don’t have to strain after effect.

“That Turk—Albanian, rather—was a becktchee —steward to a landlord who squeezed the villagers. They came to us—long before this—and said that if we didn’t kill this man they couldn’t see what good the organization was to them. You know becktchees—the good ones, and the bad ones—but we kill neither, for mere squeezing. But this swine raped two peasant girls—one after the other. That's all there was to him.”

“Were the two women frightened ?” I asked.

“Naturally. That first night's march took the breath out of them. But afterwards—well, we were inexperienced. We gave them a month, believing we should have the money from Constantinople in a week. Of course, we wanted them to take it seriously. Those missionaries are different from us, but we know that some of them are in earnest. We had one fear—she might decide to martyr herself. Fortunately she didn’t.

“So we arranged dramatic scenes. I was best at them—that’s why I am the Bad Man. But Yani Sandanski has the instincts of a French dancing master. I’ve seen the perspiration stand out on his bald head, with winter frost about us. I've seen him go off by himself among the trees and clench those big hands of his and grind his teeth. Well, he got his reward. He was handed down to history as the Good Man.”

“But what happened when the first term expired ?"

“Ah, that was it. Bluff never pays, nor were we used to bluff. We tried to keep it up. But what can you do with an angry, elderly and very respectable woman glaring at you? Once she made a sudden move with her umbrella—she always carried that umbrella—and her Bible and the old bonnet—well, it may have been imagination on my part, that move with the umbrella, but I stumbled backward through the doorway of the hut, to save my dignity. But I didn’t save much of it.

“She wouldn’t allow smoking. She didn’t forbid it by actual injunction, you know, but so: 'Have you human hearts, or have you absolutely no regard for helpless women?’ In a shrill voice, you know. You couldn’t smoke in her presence after such a scene.”

“In Miss Stones’ narrative,” I put in, “she once refers to the superstitious fears of one of the chetniks; altogether you have the impression of very ignorant peasants. Who were the chetniks?”

His lips curled as he answered:

“There was Krusty Asenov; you’ve heard of him. He was a school teacher with a college training. A big, strong fellow, whom Miss Stone refers to as Metchkato, “the Bear.” We all had pseudonyms. Poor Krusty, he was killed in the insurrection.

“Then there was “Tchaoush.” He was Alexander Eleav, also a school teacher, with a half finished university training. Dontcho killed him, with an axe, while he was sleeping. And there was the doctor. He had studied medicine, in Paris, I believe. That was Petrov. Saave Michaelov was with us; you’ve see him, tall and aristocratic—superstition didn’t bother him much. And Peter Kitanov, voyvoda of Djumaysko, with us now, he with whom you were discussing Ibsen’s “The Lady from the Sea” last night. School teachers, most of them, turned out of their position for their radicalism.

“I’ve no doubt that Miss Stone’s attitude was sincere, but it’s amusing, considering a little incident that comes to my mind now. She got after Krusty Asenov with her Bible—wanted him to read a marked chapter. He said, 'I'll read your chapter if you will read a pamphlet I have. You look into my creed, and I’ll look into yours.' She agreed; he took the Bible, and gave her some socialist pamphlet, by Kautsky, I believe. Next day he asked her if she was ready to exchange views. I believe she really tried, but she couldn’t understand it. Perhaps she wasn’t used to the terminology. Krusty recited some of the verses of her chapter, by memory. ‘You see,' he said, ‘it’s easier for me to learn your creed than it is for you to learn mine.'

“Then came the baby. It was about then that my hair turned grey. Fancy, a newborn baby on the trail with you! How often haven’t you had to put your head into a cloak to muffle a sneeze. Then think of a healthy, whooping baby with you, the country teeming with asker and Tsoncheff’s brigands. But—it’s strange how a helpless baby acts on you, especially if you have been away from women and children long. It was the death of our authority then. I think, perhaps unconsciously, Miss Stone, as well as the mother, came to regard us a little more humanly after that. I am sure, too, that our fear of asker and Dontcho wasn’t a purely selfish one. When the fight with Dontcho did come off, how the boys did jam those women down into a hole. I wasn’t there at the time; I was on Dontcho’s flank, while the others got out of the way with the women, and Sandanski tells me they clung to him as if they didn’t relish that kind of a rescue."
"You see, Tsoncheff had Dontcho after us, to 'rescue’ those captives. He didn’t want us to get that money. It would have served his purpose as well to have those women killed on our hands. You know that Macedonia would have been turned upside down and raked with a fine comb if anything had really happened to them. And if, by any rare chance, we had escaped that, Grueff and the rest of them would have killed us when they came out of prison. As it was they disapproved when they heard of it, but then we already had the money. I tell you the various kinds of danger we were in set me perspiring many a night.

Finally we were so hard pressed that we took a big chance. We crossed the frontier into Bulgaria. We were there quite a while, and it was a period of rest. We were over there a few miles out of Kustendil, where the barracks are, but nobody dreamed we were in that neighborhood. It was then I went to Sofia and saw Mr. Dickenson, the American representative. He thought me a common peasant, hired by the brigands; Peter Kitanov’s brother, Sandy, acting as interpreter, in French. I didn’t look as if I understood even Bulgarian, the dress I was in. 'Ten thousand francs,’ he said, 'not a sou more.’ We couldn’t come to terms.

“When I was in Samakov,” I said, “one of the missionaries told me that when the committee met the ‘brigands/ there was such a remarkably intelligent man there that they thought it must have been Grueff’s comrade, Gotze Deltcheff. Who was he?”

“That was only Krusty Asenov. Deltcheff—he was the only one of the big leaders not in prison—was down in Monastir at the time, and knew nothing of the affair till it was over. And there was the story that Sarafoff was in it—asininity—Sarafoff, Prince Ferdinand’s creature, fellow creature to Tsoncheff.”

“What became of the money when you got it?”

“A committee took charge of it: D-S-in Sofia now, and old M-in Dubnitza, you know them both, men whose integrity is above suspicion. The third was Gotze Deltcheff. Tsoncheff knows what became of part of it—to his cost. But most of it financed the insurrection of 1904 in Monastir. And then,” here he smiled at me, “I got some of it. They gave me five liras to make a Christmas visit home. The others got nothing.”

We remained silent a long time, he meditating. I had heard much of this story before, from others.

“And Dontcho,” I remarked, half to myself, “is now a prosperous, honored citizen, living in his own house in Dubnitza.”

“I know,” he answered, murkily. “They lionized him in the press. Though he’s collected more ransoms in his time than ten Miss Stones would have brought. But we’re the outlaws. I am indifferent for myself—but, the others—most of them died for their ideas—never had so much as a lira in their ragged pockets. But they were only brigands. God! What greasy hypocrites they are! The smug diplomats and editors and the clergy, with their hanging jowls and rotund bellies. Yes, brigands, we are. They allow our women and small babies to be outraged and slaughtered, and when we ask them for help, only to stop it, in the name of Christ, they give us soft, lying words. And then, when we give one of their women a few months’ worry and discomfort, which we more than share with her, only to give us the means to save a million women from death, or worse, we are brigands. Because it was one of their women, they didn’t worry about poor Mrs. Tsilka, no, it was only Miss Stone. For that we are brigands, outlaws, criminals. No, damn such a civilization. It isn’t real.”

So he would usually express his bitterness, for we had many more talks on the subject. He impressed me strongly, as he did all those young teachers who collected around him, though he was not a man of much school made education himself.

I often regret that I did not make the short detour necessary to meet Sandanski in Razlog. I feel that his was the leading mind. He it was who ended the last of Prince Ferdinand’s intrigues in Macedonia by removing Sarafoff from the field of activity. He and Tchernopeef are the leaders of the socialist wing in Macedonia, who would have substituted economic action for armed force. When the Young Turks declared the constitution, these two and their associates were the first to respond by laying down their arms. The Young Turks received them with open arms in Salonica. Nor is this to be wondered at, for the Young Turk movement is as much begotten of socialism as is the revolutionary movement in Russia. Today chetas in Macedonia are things of the past.

Today, as I write this, I read in a newspaper correspondent’s despatch that Sandanski led the vanguard of the Young Turk army to the gates of Constantinople with a company of one hundred Bulgars, followed by mixed battalions of Greeks, Jews and Turks.

The success of the Young Turks has, indeed, saved me the usual prophetic utterances which are proper at the end of any book of this kind. I would not, when I began this narrative, have predicted the things that have come to pass since. That Turks, Bulgars and Greeks should march shoulder to shoulder on Stamboul to depose the Sultan, as they are now doing, would then have been a prophecy to be laughed at by all sane men of political understanding. Already these good, sane men of ponderous understanding are intriguing to turn back the tide of evolution. But there is an ideal behind Young Turkey to which Turkey is only incidental.

On the night of November 9th, Vladi and I, with a dozen others, scurried across the frontier, and my Macedonian experiences were ended.

\backmatter

\end{document}
